% vim: set foldmethod=marker foldlevel=0:

\documentclass[a4paper]{article}
\usepackage[UKenglish]{babel}

\usepackage[hidelinks]{hyperref}

\usepackage{preamble}

\fancyhead[L]{MA270 Assignment 1}
\title{MA270 Analysis 3, Assignment 1}
\colorlet{questionbodycolor}{blue!50}

\begin{document}

\maketitle

\setlength{\parindent}{0em}
\setlength{\parskip}{1em}

% {{{ Q1
\question{1}

\begin{questionbody}
Let $f : [0, \infty) \to \R$ be a function such that $\llim{x \to \infty} f(x) = 0$, $f(0) = 0$, and $f(1) = 1$. For each $n \in \N$, let $f_n(x) := f(nx)$ for every $x \in [0, \infty)$. % chktex 9
%
\begin{enumerate}[(a)]
    \item Show that the sequence $(f_n)$ converges to 0 pointwise but not uniformly.
    \item Show that for any $a > 0$, we have $\sup\limits_{x \in [a, \infty)} |f_n(x)| \to 0$ as $n \to \infty$ (in such case, we say that $f_n$ converges uniformly to 0 on $[a, \infty)$). % chktex 9
\end{enumerate}
\end{questionbody}

\subsection{~} % 1.a

We will first show that $f_n \to 0$ pointwise. If $x = 0$ then $f_n(x) = 0$ for any $n$, so consider an arbitrary $x_0 \in (0, \infty)$. For any $\varepsilon > 0$, we want to choose an $N$ such that $\forall n \ge N$, $|f(n x_0)| < \varepsilon$.

Since $\llim{x \to \infty} f(x) = 0$, we know that $\forall \varepsilon > 0$, $\exists N_0 \in \N$ such that $\forall x \ge N_0$, $|f(x)| < \varepsilon$.

To apply this above, we just want to choose $N$ such that $n x_0 \ge N_0$, so we choose $N = \lceil N_0 / x_0 \rceil$. Then $\forall n \ge N$, we have $|f_n(x_0)| = |f(n x_0)| < \varepsilon$ as required. Therefore $f_n \to 0$ pointwise.

However, $f \not\rightrightarrows 0$ because no matter which $n$ we choose, we can find some $x$ such that $f_n(x) \ne 0$. In this case, $x = \f1n$, and $f_n(x) = f\l(n \f1n\r) = f(1) = 1$. Therefore $f$ does not converge uniformly to 0.
\hfill$\square$

\subsection{~} % 1.b

Fix some arbitrary $a > 0$. We want to show that $\forall \varepsilon > 0$, $\exists N \in \N$ such that $\forall x \in [a, \infty)$, $\forall n \ge N$, $| f(nx) | < \varepsilon$. Since $\llim{x \to \infty} f(x) = 0$, we know that $\forall \varepsilon > 0$, $\exists N_0 \in \N$ such that $\forall x \ge N_0$, $|f(x)| < \varepsilon$. % chktex 9

So we just choose our $N$ such that when $n > N$, $nx \ge N_0$ for all $x \in [a, \infty)$. So we choose $N = \lceil N_0 / a \rceil$. Then the condition is met and therefore $f_n$ converges uniformly on $[a, \infty)$. % chktex 9
\hfill$\square$

% }}}

% {{{ Q2
\newquestion{2}

\begin{questionbody}
Study the pointwise and uniform convergence of the sequence $f_n : (0, \infty) \to \R$ defined by $f_n(x) = \df{\sin(nx)}{n \sqrt x}$.

\textbf{Hint}: For uniform convergence, given $n \in \N$, one can analyse the behaviour of $f_n(x)$ separately for $x \in \l[ \f1n, \infty \r)$ and $x \in \l( 0, \f1n \r)$. For the analysis in the latter interval, one can consider the function $h : [0, \infty) \to \R$ defined by $h(x) = \df{\sin x}x$ for $x \ne 0$ and $h(0) = 1$, show that $h$ is bounded on $[0, \infty)$, and make use of this. % chktex 9
\end{questionbody}

I shall conjecture that $f_n$ converges pointwise to 0. To show this, we need to show that \[
\forall \varepsilon > 0,\ \forall x \in (0, \infty),\ \exists N \in \N \text{ such that } \forall n \ge N,\ |f_n(x)| < \varepsilon .
\]
So consider some arbitrary $\varepsilon$ and $x$. Then \begin{align*}
|f_n(x)| &= \l| \f{\sin(nx)}{n \sqrt x} \r| \\[0.5ex]
&= \f{|\sin(nx)|}{|n \sqrt x|} \\[0.5ex]
&= \f{|\sin(nx)|}{n \sqrt x} \\[0.5ex]
|f_n(x)| &< \varepsilon \\[0.5ex]
\implies |\sin(nx)| &< \varepsilon n \sqrt x
\end{align*}

The maximum value of $|\sin(nx)|$ is 1, so as long as $\varepsilon n \sqrt x > 1$, we have the desired inequality. That means we can just choose $N = \l\lceil \df1{\varepsilon \sqrt x} \r\rceil$.

Since this value of $N$ depends on $x$, $f_n$ does not converge uniformly to 0 in general. We shall consider $f_n$ separately on the intervals $\l( 0, \f1n \r)$ and $\l[ \f1n, \infty \r)$. % chktex 9

First, consider $f_n$ only on the interval $\l( 0, \f1n \r)$. We shall first consider $h : [0, \infty) \to \R$ defined by \[ % chktex 9
h(x) = \begin{cases}
1 & \text{if } x = 0 \\
\df{\sin x}x & \text{otherwise}
\end{cases}
\]

The function $h$ is bounded above and below by $\pm 1$. To show this, we want to show that $\forall x \in [0, \infty)$, % chktex 9
\begin{align*}
|h(x)| &\le 1 \\[0.5ex]
\l| \f{\sin x}x \r| &\le 1 \\[0.5ex]
\f{|\sin x|}{|x|} &\le 1 \\[0.5ex]
|\sin x| &\le |x|
\end{align*}
This is a well-known fact for $x \ge 0$, which is exactly where $h$ is defined. Therefore $-1 \le h(x) \le 1$.

This should help with studying the uniform convergence of $f_n$. We can note that \[ f_n(x) = h(nx) \f x{\sqrt x} \] but unfortunately, I don't know where to go from here.

% TODO: Now we will consider $f_n$ on $\l[ \f1n, \infty \r)$.

% }}}

% {{{ Q3
\newquestion{3}

\begin{questionbody}
For each $n \in \N$, let $f_n : [0, 1] \to \R$ be a Riemann integrable function. Suppose that the sequence $(f_n)$ converges uniformly to a function $f : [0, 1] \to \R$. Let $(g_n)$ be the sequence defined by $g_n = f_{n+3} - f_n$. Prove that the limit $\llim{n \to \infty} \int_0^1 g_n$ exists and calculate this limit.
\end{questionbody}

% TODO: Do I need to prove that the limit exists?
% Why does integrability matter? In my current answer, we could just omit the integral signs

\begin{align*}
\llim{n \to \infty} \int_0^1 g_n &= \llim{n \to \infty} \intlim 01 {( f_{n+3}(x) - f_n(x) )} x \\[0.5ex]
&= \llim{n \to \infty} \intlim 01 {f_{n+3}(x)} x - \llim{n \to \infty} \intlim 01 {f_n(x)} x \\[0.5ex]
&= \intlim 01 {\llim{n \to \infty} f_{n+3}(x)} x - \intlim 01 {\llim{n \to \infty} f_n(x)} x \tag{*}\label{eqn:Q3-lim-int-swap} \\[0.5ex]
&= \intlim 01 {f(x)} x - \intlim 01 {f(x)} x \\[0.5ex]
&= 0
\end{align*}
Note that we're only allowed to move the limits inside the integrals on line~\eqref{eqn:Q3-lim-int-swap} because $f_n$ converges to $f$ uniformly. A weaker notion of convergence would not allow for this.

% }}}

% {{{ Q4
\newquestion{4}

\begin{questionbody}
Let $g : [0, 1] \to \R$ be a continuous function. For each $n \in \N$, let $g_n : [0, 1] \to \R$ be defined by $g_n(x) = \df x{1 + nx^2} \, g(x)$. Show that $\ds \intlim 01 {g_n(x)} x \to 0$ as $n \to \infty$.

\textbf{Hint}: One can start by showing that the sequence of functions $h_n : [0, 1] \to \R$ defined by $h_n(x) = \df x{1 + nx^2}$ converges uniformly to 0.
\end{questionbody}

Consider the sequence of functions $h_n : [0, 1] \to \R$ defined by $h_n(x) = \df x{1 + nx^2}$. For $h_n$ to converge uniformly to 0, we need \[
\forall \varepsilon > 0,\ \exists N \in \N \text{ such that } \forall n \ge N,\ \forall x \in [0, 1],\ |h_n(x)| < \varepsilon .
\]

A bit of simple manipulation tells us that \begin{align*}
|h_n(x)| &< \varepsilon \\[0.5ex]
\l| \f{x}{1 + nx^2} \r| &< \varepsilon \\[0.5ex]
\f{|x|}{|1 + nx^2|} &< \varepsilon \\[0.5ex]
\f{|x|}{1 + nx^2} &< \varepsilon \\[0.5ex]
|x| &< \varepsilon (1 + nx^2) \\[0.5ex]
|x| &< \varepsilon + \varepsilon nx^2 \\[0.5ex]
% |x| - \varepsilon &< \varepsilon nx^2 \\[0.5ex]
\f{|x| - \varepsilon}{\varepsilon x^2} &< n
\end{align*}
needs to hold for all $x$. Since $x \in [0, 1]$, the left hand side is maximised when $x=1$, so we need $n > \df{1-\varepsilon}{\varepsilon}$ and therefore we choose $N = \l\lceil \df{1 - \varepsilon}{\varepsilon} \r\rceil$. Therefore $h_n \rightrightarrows 0$.

Now we want to consider $\ds \llim{n \to \infty} \intlim 01 {g_n(x)} x$. Since $h_n$ converges uniformly and $g$ is continuous, $g_n$ converges uniformly. Therefore we can move the limit inside the integral and instead consider \begin{align*}
\intlim 01 {\llim{n \to \infty} g_n(x)} x &= \intlim 01 {\llim{n \to \infty} (h_n(x) g(x))} x \\[0.5ex]
&= \intlim 01 {\l( \llim{n \to \infty} h_n(x) \r) g(x)} x \\[0.5ex]
&= \intlim 01 {0 g(x)} x \\
&= 0
\end{align*}

% }}}

\end{document} % chktex 17
