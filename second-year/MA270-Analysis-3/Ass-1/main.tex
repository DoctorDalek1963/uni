% vim: set foldmethod=marker foldlevel=0:

\documentclass[a4paper]{article}
\usepackage[UKenglish]{babel}

\usepackage{preamble}

\fancyhead[L]{MA270 Assignment 1}
\title{MA270 Analysis 3, Assignment 1}
\colorlet{questionbodycolor}{blue!50}

\begin{document}

\maketitle

\setlength{\parindent}{0em}
\setlength{\parskip}{1em}

% {{{ Q1
\question{1}

\begin{questionbody}
Let $f : [0, \infty) \to \R$ be a function such that $\llim{x \to \infty} f(x) = 0$, $f(0) = 0$, and $f(1) = 1$. For each $n \in \N$, let $f_n(x) := f(nx)$ for every $x \in [0, \infty)$. % chktex 9
%
\begin{enumerate}[(a)]
    \item Show that the sequence $(f_n)$ converges to 0 pointwise but not uniformly.
    \item Show that for any $a > 0$, we have $\sup\limits_{x \in [a, \infty)} |f_n(x)| \to 0$ as $n \to \infty$ (in such case, we say that $f_n$ converges uniformly to 0 on $[a, \infty)$). % chktex 9
\end{enumerate}
\end{questionbody}

\subsection{~} % 1.a

We will first show that $f_n \to 0$ pointwise. If $x = 0$ then $f_n(x) = 0$ for any $n$, so consider an arbitrary $x_0 \in (0, \infty)$. For any $\varepsilon > 0$, we want to choose an $N$ such that $\forall n \ge N$, $|f(n x_0)| < \varepsilon$.

Since $\llim{x \to \infty} f(x) = 0$, we know that $\forall \varepsilon > 0$, $\exists N_0$ such that $\forall x \ge N_0$, $|f(x)| < \varepsilon$.

To apply this above, we just want to choose $N$ such that $n x_0 \ge N_0$, so we choose $N = \lceil N_0 / x_0 \rceil$. Then $\forall n \ge N$, we have $|f_n(x_0)| = |f(n x_0)| < \varepsilon$ as required. Therefore $f_n \to 0$ pointwise.

However, $f \not\rightrightarrows 0$ because no matter which $n$ we choose, we can find some $x$ such that $f_n(x) \ne 0$. In this case, $x = \f1n$, and $f_n(x) = f(n \f1n) = f(1) = 1$. Therefore $f$ does not converge uniformly to 0.

\subsection{~} % 1.b

Fix some arbitrary $a > 0$.

% }}}

% {{{ Q2
\newquestion{2}

\begin{questionbody}
Study the pointwise and uniform convergence of the sequence $f_n : (0, \infty) \to \R$ defined by $f_n(x) = \df{\sin(nx)}{n \sqrt x}$.

\textbf{Hint}: For uniform convergence, given $n \in \N$, one can analyse the behaviour of $f_n(x)$ separately for $x \in \l[ \f1n, \infty \r)$ and $x \in \l( 0, \f1n \r)$. For the analysis in the latter interval, one can consider the function $h : [0, \infty) \to \R$ defined by $h(x) = \df{\sin x}x$ for $x \ne 0$ and $h(0) = 1$, show that $h$ is bounded on $[0, \infty)$, and make use of this. % chktex 9
\end{questionbody}

Answer

% }}}

% {{{ Q3
\newquestion{3}

\begin{questionbody}
For each $n \in \N$, let $f_n : [0, 1] \to \R$ be a Riemann integrable function. Suppose that the sequence $(f_n)$ converges uniformly to a function $f : [0, 1] \to \R$. Let $(g_n)$ be the sequence defined by $g_n = f_{n+3} - f_n$. Prove that the limit $\llim{n \to \infty} \int_0^1 g_n$ exists and calculate this limit.
\end{questionbody}

Answer

% }}}

% {{{ Q4
\newquestion{4}

\begin{questionbody}
Let $g : [0, 1] \to \R$ be a continuous function. For each $n \in \N$, let $g_n : [0, 1] \to \R$ be defined by $g_n(x) = \df x{1 + nx^2} \, g(x)$. Show that $\intlim 01 {g_n(x)} x \to 0$ as $n \to \infty$\dots

\textbf{Hint}: One can start by showing that the sequence of functions $h_n : [0, 1] \to \R$ defined by $h_n(x) = \df x{1 + nx^2}$ converges uniformly to 0.
\end{questionbody}

Answer

% }}}

\end{document} % chktex 17
