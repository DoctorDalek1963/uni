% vim: set foldmethod=marker foldlevel=0:

\documentclass[a4paper]{article}
\usepackage[UKenglish]{babel}

\usepackage{preamble}

\fancyhead[L]{MA270 Assignment 2}
\title{MA270 Analysis 3, Assignment 2}
\colorlet{questionbodycolor}{blue!50}

\begin{document}

\maketitle

\setlength{\parindent}{0em}
\setlength{\parskip}{1em}

% {{{ Q1
\question{1}

\begin{questionbody}
\begin{enumerate}[(a)]
\item For each $n \in \N$, let $f_n : \R \to \R$ be a function and $|f_n|$ be the function $\R \to \R$ obtained by composing with absolute value (i.e. $|f_n| : x \mapsto |f_n(x)|$). Show that if the series $\sum |f_n|$ converges pointwise then $\sum f_n$ converges pointwise.
% (\textit{You may use this fact elsewhere in this assignment and this module.})

\item Find an example of a sequence $(f_n)$ of functions $\R \to \R$ such that $\sum f_n$ converges pointwise but $\sum |f_n|$ does not converge pointwise.
\end{enumerate}
\end{questionbody}

\subsection{~} % 1.a

We can apply the theorem from first year analysis for series of real numbers. Let $(a_n)$ be a sequence. If $\sum |a_n|$ converges then $\sum a_n$ converges. This was proved in MA141.

Then we can define a set of sequences $(a_{x, n})$ for all $x \in \R$ where $a_{x, n} = f_n(x)$. Then what we want to show is equivalent to saying that if $\sum |a_{x, n}|$ converges then $\sum a_{x, n}$ converges. This is evident from the theorem above. Therefore if $\sum |f_n|$ converges pointwise then $\sum f_n$ converges pointwise.

\hfill $\square$

\subsection{~} % 1.b

Let $f_n(x) = {(-1)}^n x$. Then $\sum f_n = -x + x - x + x - \dotsb = 0$ so converges pointwise, but $\sum |f_n| = x + x + x + \dotsb$ so does not converge pointwise.

% }}}

% {{{ Q2
\newquestion{2}

\begin{questionbody}
For each integer $n \ge 1$ and $x \in \R$, let $f_n(x) = \df{x}{x^2 + n^2}$.
\begin{enumerate}[(a)]
\item Show that the series $\sum_{n=1}^\infty f_n(x)$ converges pointwise. (\textbf{Hint}: You may use \textbf{Q1~(a)}.)

\item Does the series $\sum_{n=1}^\infty f_n(x)$ converge uniformly? Justify your answer. (\textbf{Hint}: Try to give a lower bound to $\sum_{k=n+1}^{2n} f_k(x)$ evaluated at ${x = 2n}$.)

\item Is the function $\sum_{n=1}^\infty f_n(x)$ continuous? (\textbf{Hint}: Fix $t > 0$ and restrict the functions $f_n$ to the interval $[-t, t]$ and analyse the convergence of associated series $\sum f_n$.)
\end{enumerate}
\end{questionbody}

\subsection{~} % 2.a

% TODO: This is just waffle
We shall consider instead $\sum_{n=1}^\infty |f_n|(x)$. This converges pointwise by the comparison test, so $\sum_{n=1}^\infty f_n(x)$ converges by \textbf{Q1~(a)}.

\subsection{~} % 2.b

We shall consider $\sum_{k = n+1}^{2n} f_k(2n)$. When $n=1$, this sum is $f_2(2) = \f28 = \f14$. As $n$ increases, the sum increases, so $\f14$ is a lower bound. % TODO: Prove

The series $\sum_{n=1}^\infty f_n(x)$ does converge uniformly because $f_n(x)$ converges to 0.

\subsection{~} % 2.c

Yes it is continuous. % TODO: Prove

% }}}

% {{{ Q3
\newquestion{3}

\begin{questionbody}
Let $f : \R^2 \setminus \{(x, y) \in \R^2 : x = y\} \to \R$ be the function defined by $f(x, y) = \df{\sin x - \sin y}{x - y}$. Show that $f$ extends by continuity to $\R^2$. (\textbf{Hint}: Use the Mean Value Theorem.)
\end{questionbody}

We want to extend $f(x, y)$ to points where $x=y$. Fix some arbitrary $x$ and $0 < \varepsilon \ll 1$, then we will consider $y = x \pm \varepsilon$.

We get \[ \cos\l( \f{\sqrt2}2 \sqrt{x^2 + y^2} \r) \] on the line $x=y$.
% TODO

% }}}

% {{{ Q4
\newquestion{4}

\begin{questionbody}
Do the following functions defined on $\R^2 \setminus \{(0, 0)\}$ admit a limit at $(0, 0)$? Justify your claims.
\begin{enumerate}[(a)]
\item $f(x, y) = \df{x^2 - y^2}{x^2 + y^2}$.

\item $f(x, y) = \df{x^3 + y^3}{x^2 + y^2}$.

\item $f(x, y) = \df{\sin x^2 + \sin y^2}{\| (x, y) \|}$. (\textbf{Hint}: Show that $|\!\sin x| \le |x|$ for all $x \in \R$.)

\item $f(x, y) = (3x + 2y) \sin \l( \df 1{2x + 3y} \r)$.
\end{enumerate}
\end{questionbody}

\subsection{~} % 4.a

We can use separate continuity. When we consider the line $y=0$, we get $f(x, 0) = \df{x^2}{x^2} = 1$ for all $x \ne 0$, and it is easy to see the limit as $x \to 0$ is 1. On the line $x=0$, we get $f(0, y) = \df{-y^2}{y^2} = -1$. Like in the previous case, the limit as $y \to 0$ is now $-1$. These limits disagree, so $f$ has no limit at $(0, 0)$.

\subsection{~} % 4.b

The limit at $(0, 0)$ is 0. We want to show that $\forall \varepsilon > 0$, $\exists \delta > 0$ such that $\| (x, y) - (0, 0) \| < \delta \implies | f(x, y) - 0 | < \varepsilon$. That is, we want $\sqrt{x^2 + y^2} < \delta \implies | f(x, y) | < \varepsilon$. Let us assume $\delta < 1$.
%
\begin{align*}
\sqrt{x^2 + y^2} &< \delta \\[0.5ex]
\qquad\qquad \implies x^2 + y^2 &> \delta^2 \qquad \text{sign flips because } 0 < \delta < 1 \\[0.5ex]
\implies \f1{x^2 + y^2} &< \f1{\delta^2} \\[0.5ex]
\implies \f{|x^3 + y^3|}{x^2 + y^2} &< \f{|x^3 + y^3|}{\delta^2} \\[0.5ex]
\implies \l|\f{x^3 + y^3}{x^2 + y^2}\r| &< \f{|x^3 + y^3|}{\delta^2} \\[0.5ex]
\therefore \varepsilon &= \f{|x^3 + y^3|}{\delta^2} \\[0.5ex]
\therefore \delta &= \sqrt{\f{|x^3 + y^3|}{\varepsilon}}
\end{align*}
Therefore $f(x, y) \to 0$ as $(x, y) \to (0, 0)$.

\subsection{~} % 4.c

The limit at $(0, 0)$ is 0. % TODO

% We will first show that $|\!\sin x| \le |x|$ for all $x \in \R$. Clearly $0 \le |\!\sin x| \le 1$ for all $x$, so we only need to consider $-1 \le x \le 1$ carefully.

\subsection{~} % 4.d

The limit at $(0, 0)$ is 0. Clearly $\sin\l(\df1{2x+3y}\r)$ doesn't have a limit at the origin, but it will oscillate between $-1$ and $1$ with increasing frequency as $(x, y) \to (0, 0)$. However, $3x+2y$ converges linearly to 0 as $(x, y) \to (0, 0)$, and so this term will dominate $f$. Therefore $f(x, y) \to 0$ as $(x, y) \to (0, 0)$.
% TODO: Explain better

% }}}

\end{document}
