% vim: set foldmethod=marker foldlevel=0:

\documentclass[a4paper]{article}
\usepackage[UKenglish]{babel}

\usepackage{preamble}

\newcommand{\opnorm}[1]{{\ensuremath{} {\| #1 \|}_{\mathrm{op}}}}

\fancyhead[L]{MA270 Assignment 3}
\title{MA270 Analysis 3, Assignment 3}
\colorlet{questionbodycolor}{blue!50}

\begin{document}

\maketitle

\setlength{\parindent}{0em}
\setlength{\parskip}{1em}

% {{{ Q1
\question{1}

\begin{questionbody}
Let $n, k \ge 1$ be two integers. In the following, the vector space $L(\R^n, \R^k)$ of linear maps from $\R^n$ to $\R^k$ is endowed with the operator norm $\opnorm \cdot$. Let $(A_m)$ be a Cauchy~sequence in the normed vector space $L(\R^n, \R^k)$.
\begin{enumerate}[(a)]
\item Show that $(A_m)$ is a bounded sequence, i.e.\ $\exists M > 0$ such that $\forall m \ge 1$, $\opnorm{A_m} \le M$.

\item Show that $\forall v \in \R^n$, the sequence $(A_m v)$ in $\R^k$ converges.

\item Define a function $A : \R^n \to \R^k$ by setting $A v = \llim{m \to \infty} A_m v$. Show that $A \in L(\R^n, \R^k)$.

\item Show that $\opnorm{A_m - A} \to 0$ as $m \to \infty$.

% \textbf{Hint}: Show that if this is not true then $\exists \varepsilon_0 > 0$ and a sequence $m_k$ in $\N$ and a sequence $x_k$ of elements of the unit sphere $\{ x \in \R^n : \| x \| = 1 \}$ such that $\| (A_{m_k} - A) x_k \| \ge \varepsilon_0$. Justify using~\textbf{(a)} that then $\exists M > 0$ such that
% \begin{align*}
% \varepsilon_0 &\le \| (A_{m_k} - A) x_k \| \\[0.5ex]
% &\le \| (A_{m_k} - A) (x_k - x) \| + \| (A_{m_k} - A) x \| \\[0.5ex]
% &\le M \| x - x_k \| + \| (A_{m_k} - A) x \|.
% \end{align*}
% Deduce a contradiction from this.

% \textit{Remark}: This problem gives an alternative proof of the fact that $L(\R^n, \R^k)$ endowed with the operator norm is a complete normed vector space. This was left as an exercise in lectures, and an alternative (easier) method was suggested: One can use the linear isomorphism $\mu : L(\R^n, \R^k) \to \R^{k,n}$ together with the inequalities between the Frobenius norm and the operator norm discussed in the lecture to prove the completeness. You do not need to include this proof in your assignment.
\end{enumerate}
\end{questionbody}

\subsection{~} % 1.a

By the definition of a Cauchy sequence, $\forall \varepsilon > 0$, $\exists N \in \N$ such that $\forall n, m \ge N$, $\opnorm{A_m - A_n} < \varepsilon$.

Take $\varepsilon = 1$ and $n = N$ in this definition. Then we get that $\exists N \in \N$ such that $\forall m \ge N$, $\opnorm{A_m - A_N} < 1$.

Since the operator norm satisfies the triangle inequality, we have that for $m \ge N$,
\begin{align*}
\opnorm{A_m} &\le \opnorm{A_m - A_N} + \opnorm{A_N} \\[0.5ex]
&< 1 + \opnorm{A_N}.
\end{align*}
% TODO: Does this step need more justification?
Therefore $\forall m \in \N$, $\opnorm{A_m} \le \max \{ \opnorm{A_1}, \dotsc, \opnorm{A_{N-1}}, \opnorm{A_N} + 1 \}$, so $(A_m)$ is bounded.

\hfill $\square$

\subsection{~} % 1.b

Consider an arbitrary $v \in \R^n$. Then ${(A_m v)}_m$ is a Cauchy sequence because $\forall m, n \in \N$, \[ \|A_m v - A_n v \| = \| (A_m - A_n) v \| \le \opnorm{A_m - A_n} \, \| v \|. \]
Of course $A_m v \in \R^k$ and we know that $\R^k$ is complete for any $k$. That means that any Cauchy sequence in $\R^k$ converges to an element of $\R^k$. Since ${(A_m v)}_m$ is a Cauchy sequence in $\R^k$, it converges.

\hfill $\square$

\subsection{~} % 1.c

% TODO: Is this rigorous enough?
Clearly $A$ must be a map $\R^n \to \R^k$, since no other domain and codomain would make sense for the limit, so we just have to show that it's linear. Since all $A_m$ are linear,
\begin{align*}
A(\lambda v + u) &= \llim{m \to \infty} A_m (\lambda v + u) \\[0.5ex]
&= \llim{m \to \infty} \l( \lambda A_m v + A_m u \r) \\[0.5ex]
&= \llim{m \to \infty} \lambda A_m v + \llim{m \to \infty} A_m u \\[0.5ex]
&= \lambda \llim{m \to \infty} A_m v + \llim{m \to \infty} A_m u \\[0.5ex]
&= \lambda A v + A u
\end{align*}
Therefore $A \in L(\R^n, \R^k)$.

\hfill $\square$

\subsection{~} % 1.d

% \textbf{Hint}: Show that if this is not true then $\exists \varepsilon_0 > 0$ and a sequence $m_k$ in $\N$ and a sequence $x_k$ of elements of the unit sphere $\{ x \in \R^n : \| x \| = 1 \}$ such that $\| (A_{m_k} - A) x_k \| \ge \varepsilon_0$. Justify using~\textbf{(a)} that then $\exists M > 0$ such that
% \begin{align*}
% \varepsilon_0 &\le \| (A_{m_k} - A) x_k \| \\[0.5ex]
% &\le \| (A_{m_k} - A) (x_k - x) \| + \| (A_{m_k} - A) x \| \\[0.5ex]
% &\le M \| x - x_k \| + \| (A_{m_k} - A) x \|.
% \end{align*}
% Deduce a contradiction from this.

Suppose $\opnorm{A_m - A} \not\to 0$ as $m \to \infty$. That means that $\exists \varepsilon_0 > 0$ such that $\forall m$, $\opnorm{A_m - A} \ge \varepsilon_0$. Using the equivalent definition of the operator norm as $\opnorm A := \sup_{x = \| 1 \|} \| Ax \|$, we can deduce that there exists a sequence $m_k$ in $\N$ and a sequence $x_k$ of elements of the unit sphere $\{ x \in \R^n : \| x \| = 1 \}$ such that $\forall k$, $\| (A_{m_k} - A) x_k \| \ge \varepsilon_0$. Let $x$ be the limit of $(x_k)$.

By part~\textbf{(a)}, we know that $(A_m)$ is bounded, so $\exists M > 0$ such that $\opnorm{A_m - A} \le M$. Then
\begin{align*}
\varepsilon_0 &\le \| (A_{m_k} - A) x_k \| \\[0.5ex]
&\le \| (A_{m_k} - A) (x_k - x) \| + \| (A_{m_k} - A) x \| \\[0.5ex]
&\le M \| x - x_k \| + \| (A_{m_k} - A) x \|.
\end{align*}

% TODO: This doesn't use the hint properly, so it's almost certainly wrong.
% It's probably not true (or at least not useful) that $A_{m_k} - A \to 0$,
% since that observation makes the hint unnecessary
As $k \to \infty$, $\| x - x_k \| \to 0$ and $A_{m_k} - A \to 0$. Therefore $\exists \varepsilon > 0$ such that $\varepsilon_0 \le \varepsilon$. But $\varepsilon_0$ is chosen first, so we can always choose a $k$ that makes $\varepsilon < \varepsilon_0$, which is a contradiction.
Therefore $\opnorm{A_m - A} \to 0$ as $m \to \infty$.

\hfill $\square$

% }}}

% {{{ Q2
\question{2}

\begin{questionbody}
Let $K \subset \R^n$ be a sequentially compact subset and $f : K \to K$ a continuous function such that $\| f(x) - f(y) \| < \| x - y \|$ for every $x, y \in K$ such that $x \ne y$.
\begin{enumerate}[(a)]
\item Show that the function $K \to \R$, defined by $x \mapsto \| f(x) - x \|$ attains a minimum in $K$ (i.e.\ $\exists x_* \in K$ such that $\| f(x_*) - x_* \| \le \| f(x) - x \|$ for every $x \in K$).

\item Show that $f$ admits a unique fixed point in $K$ (i.e.\ a point $y_0 \in K$ such that $f(y_0) = y_0$).

% \item \textit{Optional challenge (not marked)}: Denote the unique fixed point of $f$ by $x_*$. Show that $\forall x \in K$, the sequence $(x_m)$ defined by induction $x_0 = x$ and $x_{m+1} = f(x_m)$ converges to the unique fixed point of $f$ in $K$.

% \textbf{Hint}: Note that $\| x_k - x_* \|$ is a non-negative monotonic decreasing sequence. Therefore it converges. Let $\ell \ge 0$ be its limit. If $\ell = 0$, we are done. Otherwise, $\ell > 0$. Now by sequential compactness, show that we can find a subsequence of $x_{m_k}$ that converges to some $y \ne x_*$ such that $\| y - x_* \| = \ell$. Then consider the sequence $f(x_{m_k}) = x_{m_k + 1}$. It converges to $f(y)$ and we have $\| f(y) - x_* \| < \ell$. Deduce a contradiction.
\end{enumerate}
\end{questionbody}

% If $f \in C^1$, then $\forall x \in K$, $|f'(x)| < 1 \forall x$.

\subsection{~} % 2.a

% Is it true that $f$ must have a constant second derivative? So it's strictly monotonic.

Since $K$ is sequentially compact, it is bounded. Since $f$ is continuous, the function ${g : x \mapsto \| f(x) - x \|}$ is continuous. By the Extreme Value Theorem, $g$ is bounded and attains its bounds, so $\exists x_* \in K$ such that $g(x_*) = \inf_{x \in K} g(x)$. Therefore $\forall x \in K$, $\| f(x_*) - x_* \| \le \| f(x) - x \|$.

\hfill $\square$

\subsection{~} % 2.b

Clearly $y_0$ minimises $g$. We have to prove that $g(y_0) = 0$ and $y_0$ is unique. Unfortunately I don't know how to do this.

% We know there must exist an $x_* \in K$ from part~\textbf{(a)}. Let $(x_j)$ be a sequence in $K$ which converges to $x_*$. % TODO: Finish

% To prove uniqueness, suppose we have multiple minimising points, so $\exists y_1, y_2 \in K$ such that $y_1 \ne y_2$ and $f(y_1) = f(y_2) = f(x_*)$.

% \subsection{~} % 2.c

% \textbf{Hint}: Note that $\| x_k - x_* \|$ is a non-negative monotonic decreasing sequence. Therefore it converges. Let $\ell \ge 0$ be its limit. If $\ell = 0$, we are done. Otherwise, $\ell > 0$. Now by sequential compactness, show that we can find a subsequence of $x_{m_k}$ that converges to some $y \ne x_*$ such that $\| y - x_8 \| = \ell$. Then consider the sequence $f(x_{m_k}) = x_{m_k + 1}$. It converges to $f(y)$ and we have $\| f(y) - x_* \| < \ell$. Deduce a contradiction.

% }}}

% {{{ Q3
\newquestion{3}

\begin{questionbody}
Let $A$ and $B$ be two subsets of $\R^n$ such that $A$ is closed and $B$ is sequentially compact.
\begin{enumerate}[(a)]
\item Show that the set ${A + B} := \{ a + b : a \in A \text{ and } b \in B \}$ is closed.

\item Find an example of $n \ge 1$, closed sets $A$ and $B$ in $\R^n$ such that ${A + B}$ is not closed.
\end{enumerate}
\end{questionbody}

\subsection{~} % 3.a

% Adapted from https://math.stackexchange.com/a/3945602
For ${A+B}$ to be closed, we need any sequence $(x_n)$ in ${A+B}$ to converge to some element $x \in {A+B}$.

The sequence $(x_n)$ must be the sum of two sequences $(a_n)$ in $A$ and $(b_n)$ in $B$. Since $B$ is sequentially compact, there exists a convergent subsequence $b_{n_k} \to b \in B$. Note that since $A$ is only known to be closed, we can't assume that $(a_n)$ converges.

Since $b_{n_k} \to b$, we have $x_{n_k} \to x$, which means $a_{n_k}$ converges to $x - b$. Since $A$ is closed, this limit $a = x - b \in A$. Therefore $x = a + b \in {A+B}$, so ${A+B}$ is closed.

\hfill $\square$

\subsection{~} % 3.b

% Taken from https://math.stackexchange.com/a/137438
Let $A = \N$ and $\ds B = \l\{ \f1{n+1} - n : n \in \N \r\}$. They are both closed but $A + B$ contains the sequence ${\l( \df1{n+1} \r)}_{n \in \N}$, whose limit is zero, but $0 \notin {A + B}$.

% }}}

% {{{ Q4
\newquestion{4}

\begin{questionbody}
Let $n \ge 1$ be an integer and $A \in L(\R^n, \R^n)$ be a linear map. Suppose that $\opnorm A < 1$.
\begin{enumerate}[(a)]
\item Show that the sequence $\ds {\l( \sum_{k=0}^m A^k \r)}_m$ converges in the normed vector space $\big( L(\R^n, \R^n), \opnorm \cdot \big)$ (i.e.\ $\exists B \in L(\R^n, \R^n)$ such that \\
${\ds {\l\| \sum_{k=0}^m A^k - B \r\|}_\mathrm{op} \to 0}$ as $m \to \infty$).

\item Show that if $(C_m)$ is a sequence in $L(\R^n, \R^n)$ converging to $D \in L(\R^n, \R^n)$, then $\forall E \in L(\R^n, \R^n)$, we have $\opnorm{C_m E - D E} \to 0$ as $m \to \infty$.

\item Let $\mathrm{Id} : \R^n \to \R^n$ denote the identity linear operator ($x \mapsto x$) and let $B \in L(\R^n, \R^n)$ denote the limit of the sequence $\ds {\l( \sum_{k=0}^m A^k \r)}_m$. Show that $(\mathrm{Id} - A) B = \mathrm{Id}$.
\end{enumerate}
\end{questionbody}

\subsection{~} % 4.a

Since the operator norm is a norm, $\opnorm{A^{n+1}} = \opnorm A \opnorm{A^n}$ for any $n \in \N$. Therefore $\opnorm{A^n} = {\opnorm A}^n$. We know that $\opnorm A < 1$, so $\opnorm{A^n} \to 0$ as $n \to \infty$. We define $B$ as $\ds \sum_{k=0}^\infty A^k$ so that
\begin{align*}
{\l\| \sum_{k=0}^m A^k - B \r\|}_\mathrm{op} &= {\l\| \sum_{k=0}^m A^k - \sum_{k=0}^\infty A^k \r\|}_\mathrm{op} \\[0.5ex]
&= {\l\| - \sum_{k = m+1}^\infty A^k \r\|}_\mathrm{op} \\[0.5ex]
&= |-1| {\l\| \sum_{k = m+1}^\infty A^k \r\|}_\mathrm{op} \\[0.5ex]
&= {\l\| \sum_{k = m+1}^\infty A^k \r\|}_\mathrm{op} \\
\intertext{Then by the triangle inequality,} \\
&\le \opnorm{A^{m+1}} + {\l\| \sum_{k = m+2}^\infty A^k \r\|}_\mathrm{op} \\[0.5ex]
&\le \opnorm{A^{m+1}} + \opnorm{A^{m+2}} + {\l\| \sum_{k = m+3}^\infty A^k \r\|}_\mathrm{op} \\[0.5ex]
&\le \sum_{k = m+1}^\infty {\l\| A^k \r\|}_\mathrm{op}
\end{align*}
And we know that $\opnorm{A^k} \to 0$ as $k \to \infty$, so the sum above goes to $0$ as $m \to \infty$. Therefore $\ds {\l\| \sum_{k=0}^m A^k - B \r\|}_\mathrm{op} \to 0$ as $m \to \infty$.

\hfill $\square$

\subsection{~} % 4.b

Since $C_m \to D$, we have $\opnorm{C_m - D} \to 0$ by definition. Since $L(\R^n, \R^n)$ can be represented by matrices, these linear maps are distributive, so $C_m E - D E = (C_m - D) E$. Therefore we have
\begin{align*}
\opnorm{C_m E - D E} &= \opnorm{(C_m - D) E} \\[0.5ex]
&= \opnorm{C_m - D} \, \opnorm E \\[0.5ex]
&= 0 \, \opnorm E \quad\qquad\text{as } m \to \infty \\[0.5ex]
&= 0
\end{align*}

\subsection{~} % 4.c

\begin{align*}
(\mathrm{Id} - A) B &= (\mathrm{Id} - A) \sum_{k=0}^\infty A^k \\[0.5ex]
&= \sum_{k=0}^\infty A^k - A \sum_{k=0}^\infty A^k \\[0.5ex]
&= \sum_{k=0}^\infty A^k - \sum_{k=1}^\infty A^k \\[0.5ex]
&= A^0 \\[0.5ex]
&= \mathrm{Id}
\end{align*}

% }}}

\end{document}
