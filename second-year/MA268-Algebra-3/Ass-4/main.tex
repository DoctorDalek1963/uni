% vim: set foldmethod=marker foldlevel=0:

\documentclass[a4paper]{article}
\usepackage[UKenglish]{babel}

\usepackage{preamble}

\renewcommand{\thesubsection}{Q\arabic{section}~(\roman{subsection})}

\fancyhead[L]{MA268 Assignment 4}
\title{MA268 Algebra 3, Assignment 4}
\colorlet{questionbodycolor}{violet!50}

\begin{document}

\maketitle

\setlength{\parindent}{0em}
\setlength{\parskip}{1em}

% TODO: Only submit Q6

% {{{ Q1
% \question{1}

% \begin{questionbody}
% \[
% A = \begin{pmatrix}
% 6 & 21 \\
% 4 & 14
% \end{pmatrix}.
% \] Determine the Smith Normal Form of $A$.
% \end{questionbody}

% Answer

% }}}

% {{{ Q2
% \newquestion{2}

% \begin{questionbody}
% Let $R = \mathbb F_2(X)$ and recall that this is a Euclidean ring where the Euclidean function is the degree: $\partial(f) = \deg(f)$. Compute the Smith Normal Form of the matrix \[
% A = \begin{pmatrix}
% X^3 + X + 1 & X^2 \\
% X^2 + X & X^2 + X
% \end{pmatrix}.
% \]
% \end{questionbody}

% Answer

% }}}

% {{{ Q3
% \newquestion{3}

% \begin{questionbody}
% Recall that $\R^2$ is an $M_2(\R)$-module. Show that $\op{Span}_{M_2(\R)}(\mathbf{e}_1) = \R^2$ where $\ds \mathbf{e}_1 = \begin{pmatrix} 1 \\ 0 \end{pmatrix}$.
% \end{questionbody}

% Answer

% }}}

% {{{ Q4
% \newquestion{4}

% \begin{questionbody}
% Let $R$ be a ring and $M$ an $R$-module. Show that a finite subset $\{ x_1, x_2, \dotsc, x_n \}$ of $M$ is an $R$-basis if and only if every element $x \in M$ can be written as a sum \[
% x = a_1 x_1 + a_2 x_2 + \dotsb + a_n x_n
% \] where the $a_i \in R$ are unique.
% \end{questionbody}

% Answer

% }}}

% {{{ Q5
% \newquestion{5}

% \begin{questionbody}
% \begin{enumerate}[(i)]
% \item Is $\Z[i]$ free as a $\Z$-module? If so, what is its rank?

% \item Let $n \ge 2$. Is $\R^n$ free as an $M_n(\R)$-module? Prove or disprove.
% \end{enumerate}
% \end{questionbody}

% \subsection{~} % 5.i

% Answer

% \subsection{~} % 5.ii

% Answer

% }}}

% {{{ Q6
\question{6}

\begin{questionbody}
Let $\R[X]$ be the $\R[T]$-module where multiplication is given by
\begin{align*}
(a_0 + a_1 & T + \dotsb + a_n T^n) \cdot f(X) \\
&= a_0 f(X) + a_1 f'(X) + a_2 f''(X) + \dotsb + a_n f^{(n)}(X)
\end{align*}
where $f^{(n)}(X)$ denotes the $n$-th derivative of $f(X)$ with respect to $X$.
\begin{enumerate}[(i)]
\item Compute $(1 + T - T^2 - 3T^5) \cdot (X + 3X^2)$.

\item Show that $\op{Span}_{\R[T]}(X^n) = \op{Span}_\R(1, X, \dotsc, X^n)$.

\item Show that $\R[X]$ is not free as an $\R[T]$-module.
\end{enumerate}
\end{questionbody}

\subsection{~} % 6.i

\begin{align*}
(1 + T - T^2 - 3T^5) \cdot (X + 3X^2) &= (X + 3X^2) + \diff X (X + 3X^2) \\
    & \qquad - \ddd 2{}X (X + 3X^2) - 3 \ddd 5{}X (X + 3X^2) \\[0.5ex]
&= (X + 3X^2) + (1 + 6X) - 6 - 3(0) \\[0.5ex]
&= -5 + 7X + 3X^2
\end{align*}

\subsection{~} % 6.ii

Answer

\subsection{~} % 6.iii

Answer

% }}}

\end{document}
