% vim: set foldmethod=marker foldlevel=0:

\documentclass[a4paper]{article}
\usepackage[UKenglish]{babel}

\usepackage{preamble}

\fancyhead[L]{MA268 Homework 1}
\title{MA268 Algebra 3, Homework 1}
\colorlet{questionbodycolor}{violet!50}

\begin{document}

\maketitle

\setlength{\parindent}{0em}
\setlength{\parskip}{1em}

% {{{ Q1
\question{1}

\begin{questionbody}
With the help of Euclid's algorithm, compute the multiplicative inverse of $\overline{11}$ in $\Z / 101\Z$. (Note that $\overline{11} = [11]_{101}$ in the notation of Algebra 1).
\end{questionbody}

TODO
% }}}

% {{{ Q2
\newquestion{2}

\renewcommand{\thesubsection}{Q\arabic{section}~(\roman{subsection})}

\begin{questionbody}
Let $\rho$ and $\tau$ be the following permutations: \[
\rho = \begin{pmatrix}
1 & 2 & 3 & 4 & 5 \\
2 & 3 & 5 & 1 & 4
\end{pmatrix}, \qquad \tau = \begin{pmatrix}
1 & 2 & 3 & 4 & 5 \\
3 & 1 & 2 & 5 & 4
\end{pmatrix}
\]
\begin{enumerate}[(i)]
\item Write $\rho$ and $\tau$ as products of disjoint cycles.
\item Write $\rho$ and $\tau$ as products of transpositions and state if they're even or odd.
\end{enumerate}
\end{questionbody}

\subsection{~} % 2.i

\[ \rho = (1, 2, 3, 5, 4), \qquad \tau = (1, 3, 2) (4, 5) \]

\subsection{~} % 2.ii

\[ \rho = (3, 4) (4, 5) (2, 3) (1, 2), \qquad \tau = (1, 2) (2, 3) (4, 5) \]

So $\rho$ is even and $\tau$ is odd.

% }}}

% {{{ Q3
\question{3}

\begin{questionbody}
Let $\rho = (1, 2, 3) (4, 5)$ and $\tau = (1, 2, 3, 4)$. Write the following in cycle notation (i.e. as a product of disjoint cycles): $\rho^{-1}, \tau^{-1}, \rho \tau, \tau \rho^2$.
\end{questionbody}

% We will first write $\rho$ and $\tau$ in cycle notation: \[ \rho = (2, 3) (1, 2) (4, 5), \qquad \tau = () \]
\begin{align*}
\rho^{-1} &= (3, 2, 1) (4, 5) \\
\tau^{-1} &= (4, 3, 2, 1) \\
\rho \tau &= (1, 3, 5, 4, 2) \\
\tau \rho^2 &= (1, 4)
\end{align*}
% }}}

% {{{ Q4
\newquestion{4}

\renewcommand{\thesubsection}{Q\arabic{section}~(\alph{subsection})}

\begin{questionbody}
\[ H = \l\{ \begin{pmatrix} 1 & r \\ 0 & s \end{pmatrix} \colon r \in \R, s \in \R^* \r\} \]
\begin{enumerate}[(a)]
\item Show that $(H, \cdot)$ is a group. (\textbf{Hint}: show that it's a subgroup of $\mathrm{GL}_2(\R)$).
\item Show that $H$ is non-abelian by giving a non-commuting pair of elements.
\item How many elements of order 2 does $H$ have? Explain your answer.
\end{enumerate}
\end{questionbody}

\subsection{~} % 4.a

Let $G = \mathrm{GL}_2(\R)$. We need to show three things to show that $H$ is a subgroup of $G$:
\begin{enumerate}[1)]
\item $1_G \in H$.
\item Whenever $a, b \in H$, $ab \in H$ using multiplication in $G$.
\item Whenever $a \in H$, $a^{-1} \in H$ using inverses from $G$.
\end{enumerate}

Let $r=0$ and $s=1$. Thus $1_G \in H$.

Consider $\begin{pmatrix} 1 & r_1 \\ 0 & s_1 \end{pmatrix}, \begin{pmatrix} 1 & r_2 \\ 0 & s_2 \end{pmatrix} \in H$. Then \[
\begin{pmatrix} 1 & r_1 \\ 0 & s_1 \end{pmatrix} \begin{pmatrix} 1 & r_2 \\ 0 & s_2 \end{pmatrix} = \begin{pmatrix} 1 & r_2 + r_1 s_2 \\ 0 & s_1 s_2 \end{pmatrix}
\] Clearly $r_2 + r_1 s_2 \in \R$ and since $s_1, s_2 \in \R^*$, we know $s_1 s_2 \in \R^*$. Therefore the product is in $H$.

Consider $\begin{pmatrix} 1 & r \\ 0 & s \end{pmatrix} \in H$. The inverse is \[
\f1s \begin{pmatrix} s & -r \\ 0 & 1 \end{pmatrix} = \begin{pmatrix} 1 & -\f rs \\ 0 & \f1s \end{pmatrix}
\] Since $r \in \R$ and $s \in \R^*$, we know $-\f rs \in \R$ and $\f1s \in \R^*$ so the inverse is in $H$.

Therefore $H$ is a subgroup of $\mathrm{GL}_2(\R)$ and therefore a group.

\subsection{~} % 4.b

\begin{align*}
\begin{pmatrix} 1 & 2 \\ 0 & 3 \end{pmatrix} \begin{pmatrix} 1 & 3 \\ 0 & 2 \end{pmatrix} &= \begin{pmatrix} 1 & 7 \\ 0 & 6 \end{pmatrix} \\[1ex]
\begin{pmatrix} 1 & 3 \\ 0 & 2 \end{pmatrix} \begin{pmatrix} 1 & 2 \\ 0 & 3 \end{pmatrix} &= \begin{pmatrix} 1 & 8 \\ 0 & 6 \end{pmatrix}
\end{align*}
Notably these are not the same result, so by counterexample, $H$ is non-abelian.

\subsection{~} % 4.c

Elements of order 2 are self-inverse elements. That is, any $h \in H$ will have $h^2 = 1_H$. We can use the general $h$ and see that \begin{align*}
\begin{pmatrix} 1 & r \\ 0 & s \end{pmatrix}^2 &= \begin{pmatrix} 1 & r \\ 0 & s \end{pmatrix} \begin{pmatrix} 1 & r \\ 0 & s \end{pmatrix} \\[1ex]
&= \begin{pmatrix} 1 & r + sr \\ 0 & s^2 \end{pmatrix} \\[1ex]
&= \begin{pmatrix} 1 & 0 \\ 0 & 1 \end{pmatrix}
\end{align*}

So we require that $s^2 = 1$ and $r + sr = 0$. Therefore $s = \pm 1$ and $r(1+s) = 0$, so either $r=0$, or $s = -1$. That means the self-inverse elements are \[
\l\{ \begin{pmatrix} 1 & 0 \\ 0 & 1 \end{pmatrix} \r\}
\cup \l\{ \begin{pmatrix} 1 & r \\ 0 & -1 \end{pmatrix} \colon r \in \R \r\}
\]
This set is infinite, so $H$ has infinitely many elements of order 2.
% }}}

% {{{ Q5
\newquestion{5}

\renewcommand{\thesubsection}{Q\arabic{section}~(\roman{subsection})}

\begin{questionbody}
Recall that $\Z[x]$ is the ring of polynomials in variable $x$ wit coefficient in $\Z$. Which of the following subsets are subrings of $\Z[x]$? Which of the following subsets are ideals of $\Z[x]$? Give justification for your answers.
\begin{enumerate}[(i)]
\item $\Z$.
\item $\l\{ f \in \Z[x] \colon f(5) = 1 \r\}$.
\item $\l\{ f \in \Z[x] \colon f(5) = 0 \r\}$.
\item $\l\{ f^2 \colon f \in \Z[x] \r\}$.
\item $\l\{ (x^2 + 1) g(x) \colon g \in \Z[x] \r\}$.
\end{enumerate}
\end{questionbody}

\subsection{~} % 5.i
\subsection{~} % 5.ii
\subsection{~} % 5.iii
\subsection{~} % 5.iv
\subsection{~} % 5.v
% }}}

\end{document}
