% vim: set foldmethod=marker foldlevel=0:

\documentclass[a4paper]{article}
\usepackage[UKenglish]{babel}

\usepackage{preamble}

\renewcommand{\thesubsection}{Q\arabic{section}~(\roman{subsection})}

\fancyhead[L]{MA268 Assignment 3}
\title{MA268 Algebra 3, Assignment 3}
\colorlet{questionbodycolor}{violet!50}

% For long division in Q1, taken from https://tex.stackexchange.com/a/584377
\usepackage{scalerel}
\setcounter{MaxMatrixCols}{20}
\newcommand{\longdiv}{\smash{\mkern-0.43mu\vstretch{1.31}{\hstretch{.7}{)}}\mkern-5.2mu\vstretch{1.31}{\hstretch{.7}{)}}}} % chktex 9 chktex 10

\begin{document}

\maketitle

\setlength{\parindent}{0em}
\setlength{\parskip}{1em}

% {{{ Q1
\question{1}

\begin{questionbody}
Let \[
f = X^3 + X + 1, \qquad g = X^5 + X^2 + 3
\] in $\mathbb{F}_7[X]$. Determine the quotient and remainder you obtain on dividing $g$ by $f$.
\end{questionbody}

\[
\arraycolsep=1pt
\renewcommand\arraystretch{1.2}
\begin{array}{*1r @{\hskip\arraycolsep}c@{\hskip\arraycolsep} *{11}r}
            &          &     &   &   &   &      &   & X^2 &   &    & + & 6 \\
\cline{2-13}
X^3 + X + 1 & \longdiv & X^5 &   &   &   &      & + & X^2 &   &    & + & 3 \\
            &          & X^5 &   &   & + &  X^3 & + & X^2 &   &    &   &   \\
\cline{3-9}
            &          &     &   &   &   & 6X^3 &   &     &   &    & + & 3 \\
            &          &     &   &   &   & 6X^3 &   &     & + & 6X & + & 6 \\
\cline{7-13}
            &          &     &   &   &   &      &   &     &   &  X & + & 4
\end{array}
\]

% As per SageMath
So the quotient is $X^2 + 6$ and the remainder is $X + 4$.

% }}}

% {{{ Q2
\question{2}

\begin{questionbody}
Let $R$ be an integral domain. Show that $R[x]$ is an integral domain.
\end{questionbody}

For $R[x]$ to be an integral domain, it needs to have no zero divisors. For some coefficients $a_i, b_i \in R$, where at least one $a_i \ne 0$ and at least one $b_i \ne 0$, we have $\sum_{i=0}^\infty a_i x^i, \sum_{i=0}^\infty b_i x^i \in R[x]$. Their product is some other polynomial in $R[x]$ whose coefficients are all of the form $a_i b_j$. For this product to be 0, we would need all the coefficients to be 0.

But we know there exists at least one $a_k \ne 0$ and $b_\ell \ne 0$. Then $a_k b_\ell \ne 0$, so that term of the product is non-zero. That means the product must be non-zero, so $R[x]$ has no zero divisors and is thus an integral domain.

\hfill $\square$

% }}}

% {{{ Q3
\newquestion{3}

\begin{questionbody}
Let $R$ be an integral domain. Show that ${R[x]}^* = R^*$.
\end{questionbody}

Let $f \in R[x]$. Then $f \in {R[x]}^*$ if and only if there is some $g \in R[x]$ such that $fg = 1$. We shall suppose $f \ne 0$ and $g \ne 0$, and since $R$ is an integral domain, $fg \ne 0$.

The degree of a product is the sum of the degrees, so $\deg fg = \deg f + \deg g$. So if $\deg f > 0$ or $\deg g > 0$ then $\deg fg > 0$. But $\deg 1 = 0$, so we need $\deg f = \deg g = 0$.

Therefore all elements of ${R[x]}^*$ have degree 0, meaning they are just elements of $R$. Those elements must also all be units in $R$, so ${R[x]}^* \subset R^*$.

Clearly any unit in $R$ is a unit in $R[x]$, so $R^* \subset {R[x]}^*$. Therefore ${R[x]}^* = R^*$.

\hfill $\square$

Note that if $R$ were not an integral domain, we might have $\deg fg = \deg 0$, which would break things.

% }}}

% {{{ Q4
\newquestion{4}

\begin{questionbody}
Let $R$ be a ring. An element $a \in R$ is called \textit{nilpotent} if there is some positive integer $n$ such that $a^n = 0$.
\begin{enumerate}[(i)]
\item Show that if $a$ is nilpotent, then $1 + a$ is a unit.

\item Let $p$ be a prime and $r \ge 2$. Show that $\overline 1 + \overline p X$ is a unit $(\Z / p^r \Z)[X]$. Why doesn't this contradict~\textbf{Q3}?
\end{enumerate}
\end{questionbody}

\subsection{~} % 4.i

Clearly $\ds \sum_{k=0}^{n-1} {(-1)}^k a^k \in R$. Then
\begin{align*}
\l( \sum_{k=0}^{n-1} {(-1)}^k a^k \r) (1 + a) &= \sum_{k=0}^{n-1} {(-1)}^k a^k + \l( \sum_{k=0}^{n-1} {(-1)}^k a^k \r) a \\[0.5ex]
&= \sum_{k=0}^{n-1} {(-1)}^k a^k + \sum_{k=0}^{n-1} {(-1)}^k a^{k+1} \\[0.5ex]
&= 1 + a^n \\[0.5ex]
&= 1 + 0 \\[0.5ex]
&= 1 \\
%
\intertext{Likewise,}
(1 + a) \l( \sum_{k=0}^{n-1} {(-1)}^k a^k \r) &= \sum_{k=0}^{n-1} {(-1)}^k a^k + a \l( \sum_{k=0}^{n-1} {(-1)}^k a^k \r) \\[0.5ex]
&= 1 + a^n \\[0.5ex]
&= 1
\end{align*}
So $1 + a$ is a unit.

\subsection{~} % 4.ii

${(\overline p X)}^r = {\overline p\,}^r X^r = 0$, so $\overline p X$ is nilpotent. Therefore $\overline 1 + \overline p X$ is a unit by part~\textbf{(a)}.

This doesn't contradict~\textbf{Q3} because $\Z / p^r \Z$ is not an integral domain. If $s + t = r$ then ${\overline p\,}^s \, {\overline p\,}^t = {\overline p\,}^r = 0$, so ${\overline p\,}^s$ and ${\overline p\,}^t$ are zero divisors.

% }}}

% {{{ Q5
\newquestion{5}

\begin{questionbody}
Often the easiest way to show that a subset of a ring is an ideal is to find a homomorphism whose kernel is this set. Let $I$ be the subset of $\R[X]$ consists of all polynomials $a_0 + a_1 X + \dotsb + a_n X^n$ with $a_0 + a_1 + \dotsb + a_n = 0$.
\begin{enumerate}[(i)]
\item Show that $I$ is an ideal.

\item Show that $I = (X - 1) \R[X]$.

\item Show that $\R[X] / I \cong \R$.
\end{enumerate}
\end{questionbody}

\subsection{~} % 5.i

Let $\phi : \R[X] \to \R$ be defined by $\phi(f) = f(1)$. It is easy to see that $\phi(f + g) = f(1) + g(1) = \phi(f) + \phi(g)$, that $\phi(fg) = f(1) g(1) = \phi(f) \phi(g)$, and that $\phi(1) = 1$. Therefore $\phi$ is a ring homomorphism. Clearly $\ker \phi = I$ by the definition of $I$. Therefore $I$ is an ideal.

\subsection{~} % 5.ii

Suppose $f \in I$. Then $f(1) = 0$, so $X - 1$ is a factor of $f$. Therefore we can factor out $X - 1$ from any $f \in I$. Therefore $I = \ker \phi = (X - 1) \R[X]$.

\subsection{~} % 5.iii

Clearly $\phi$ is surjective, so $\Im \phi = \R$. Then the First Isomorphism Theorem tells us that $\R[X] / I \cong \R$, where the isomorphism $\hat\phi$ is defined by $\hat\phi(f + I) = \phi(f) = f(1)$.

% }}}

% {{{ Q6
\newquestion{6}

\begin{questionbody}
Let $I = (X^2 - X) \R[X] \subset \R[X]$ (i.e.\ $I$ is the principal ideal generated by ${X^2 - X}$). Let \[
\phi : \R \to \R[X] / I, \qquad \phi(a) = a X + I.
\]
\begin{enumerate}[(i)]
\item Show that $\phi(a + b) = \phi(a) + \phi(b)$ and $\phi(ab) = \phi(a) \phi(b)$ for all $a, b \in \R$.

\item Show that $\phi$ is not a homomorphism.
\end{enumerate}
\end{questionbody}

\subsection{~} % 6.i

By the rules of addition and multiplication in quotient rings,
\begin{align*}
\phi(a + b) &= (a + b)X + I \\[0.5ex]
&= (aX + bX) + I \\[0.5ex]
&= (aX + I) + (bX + I) \\[0.5ex]
&= \phi(a) + \phi(b) \\[1.5ex]
\phi(ab) &= abX + I \\[0.5ex]
&= (aX + I) (bX + I) \\[0.5ex]
&= \phi(a) \phi(b)
\end{align*}

\subsection{~} % 6.ii

To be a homomorphism, we would need $\phi(1) = 1 + I$. However, $\phi(1) = X + I$. For this to equal $1 + I$, we would need $X - 1 \in I$. This is impossible since the lowest-degree term of any polynomial in $I$ is $X$, so $X - 1 \notin I$. Therefore $\phi$ is not a homomorphism.

% }}}

\end{document} % chktex 17
