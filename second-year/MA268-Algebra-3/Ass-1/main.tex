% vim: set foldmethod=marker foldlevel=0:

\documentclass[a4paper]{article}
\usepackage[UKenglish]{babel}

\usepackage{preamble}

\fancyhead[L]{MA268 Assignment 1}
\title{MA268 Algebra 3, Assignment 1}
\colorlet{questionbodycolor}{violet!50}

\renewcommand{\thesubsection}{Q\arabic{section}~(\roman{subsection})}

\begin{document}

\maketitle

\setlength{\parindent}{0em}
\setlength{\parskip}{1em}

% This is questions 1 & 2 from homework 3, as requested

% {{{ Q1
\question{1}

\begin{questionbody}
Let \[
    G = \l\{ \begin{pmatrix}
        1 & x & z \\
        0 & 1 & y \\
        0 & 0 & 1
    \end{pmatrix} : x, y, z \in \R \r\}, \qquad
    H = \l\{ \begin{pmatrix}
        1 & 0 & z \\
        0 & 1 & 0 \\
        0 & 0 & 1
    \end{pmatrix} : z \in \R \r\}.
\] Note that $G$ is a subgroup of $\mathrm{GL}_3(\R)$.
%
\begin{enumerate}[(i)]
\item Let \[
    \phi : G \to \R^2, \qquad \phi \begin{pmatrix}
        1 & x & z \\
        0 & 1 & y \\
        0 & 0 & 1
    \end{pmatrix} = (x, y).
\] Show that $\phi$ is a homomorphism.

\item Show that $H$ is a normal subgroup of $G$.

\item Show that the only element of $G$ of finite order is $I_3$, the identity matrix. \textbf{Hint}: This is easier if you use $\phi$.
\end{enumerate}
\end{questionbody}

\subsection{~} % 1.i

Let \[ A = \begin{pmatrix}
    1 & a & c \\
    0 & 1 & b \\
    0 & 0 & 1
\end{pmatrix}, \qquad
X = \begin{pmatrix}
    1 & x & z \\
    0 & 1 & y \\
    0 & 0 & 1
\end{pmatrix}
\] be elements of $G$. Then $\phi(A) = (a, b)$, $\phi(X) = (x, y)$, and
\begin{align*}
\phi(AX) &= \phi \l( \begin{pmatrix}
    1 & a & c \\
    0 & 1 & b \\
    0 & 0 & 1
\end{pmatrix}
\begin{pmatrix}
    1 & x & z \\
    0 & 1 & y \\
    0 & 0 & 1
\end{pmatrix} \r) \\[1ex]
&= \phi \begin{pmatrix}
    1 & a + x & z + ay + c \\
    0 & 1 & b + y \\
    0 & 0 & 1
\end{pmatrix} \\[1ex]
&= (a + x, b + y)% \\
% &= (a, b) + (x, y) \\
% &= \phi(A) + \phi(X)
\end{align*}
And $\phi(A) + \phi(X) = (a + x, b + y)$, so $\phi(AB) = \phi(A)\phi(B)$ and therefore $\phi$ is a homomorphism.

\subsection{~} % 1.ii

For $H$ to be a normal subgroup, we need to have $gHg^{-1} = H$ for all $g \in G$, or equivalently, $ghg^{-1} = h$ for all $g \in G, h \in H$.

Let \[ g = \begin{pmatrix}
    1 & x & z \\
    0 & 1 & y \\
    0 & 0 & 1
\end{pmatrix} \in G, \qquad
h = \begin{pmatrix}
    1 & 0 & w \\
    0 & 1 & 0 \\
    0 & 0 & 1
\end{pmatrix} \in H
\]
Then
\begin{align*}
M_g &= \begin{pmatrix}
\begin{vmatrix} 1 & y \\ 0 & 1 \end{vmatrix} &
\begin{vmatrix} 0 & y \\ 0 & 1 \end{vmatrix} &
\begin{vmatrix} 0 & 1 \\ 0 & 0 \end{vmatrix} \\[2.5ex]
\begin{vmatrix} x & z \\ 0 & 1 \end{vmatrix} &
\begin{vmatrix} 1 & z \\ 0 & 1 \end{vmatrix} &
\begin{vmatrix} 1 & x \\ 0 & 0 \end{vmatrix} \\[2.5ex]
\begin{vmatrix} x & z \\ 1 & y \end{vmatrix} &
\begin{vmatrix} 1 & z \\ 0 & y \end{vmatrix} &
\begin{vmatrix} 1 & x \\ 0 & 1 \end{vmatrix}
\end{pmatrix} \\[1ex]
&= \begin{pmatrix}
1 & 0 & 0 \\
x & 1 & 0 \\
xy - z & y & 1
\end{pmatrix} \\[1ex]
%
C_g &= \begin{pmatrix}
1 & 0 & 0 \\
-x & 1 & 0 \\
xy - z & -y & 1
\end{pmatrix} \\[1ex]
%
C_g^T &= \begin{pmatrix}
1 & -x & xy - z \\
0 & 1 & -y \\
0 & 0 & 1
\end{pmatrix} \\[1ex]
%
\det g &=
1 \begin{vmatrix} 1 & y \\ 0 & 1 \end{vmatrix}
+ 0 \begin{vmatrix} x & z \\ 0 & 1 \end{vmatrix}
+ 0 \begin{vmatrix} x & z \\ 1 & y \end{vmatrix} \\[0.5ex]
&= 1 \\[1ex]
%
\therefore g^{-1} &= \f1{\det g} C_g^T \\[1ex]
&= \begin{pmatrix}
    1 & -x & xy - z \\
    0 & 1 & -y \\
    0 & 0 & 1
\end{pmatrix}
\end{align*}

And then we get
\begin{align*}
ghg^{-1} &= \begin{pmatrix}
    1 & x & z \\
    0 & 1 & y \\
    0 & 0 & 1
\end{pmatrix}
\begin{pmatrix}
    1 & 0 & w \\
    0 & 1 & 0 \\
    0 & 0 & 1
\end{pmatrix}
\begin{pmatrix}
    1 & -x & xy - z \\
    0 & 1 & -y \\
    0 & 0 & 1
\end{pmatrix} \\[1ex]
%
&= \begin{pmatrix}
    1 & x & w + z \\
    0 & 1 & y \\
    0 & 0 & 1
\end{pmatrix}
\begin{pmatrix}
    1 & -x & xy - z \\
    0 & 1 & -y \\
    0 & 0 & 1
\end{pmatrix} \\[1ex]
%
&= \begin{pmatrix}
    1 & x - x & xy - z - xy + w + z \\
    0 & 1 & y - y \\
    0 & 0 & 1
\end{pmatrix} \\[1ex]
%
&= \begin{pmatrix}
    1 & 0 & w \\
    0 & 1 & 0 \\
    0 & 0 & 1
\end{pmatrix} \\[0.5ex]
&= h
\end{align*}

\subsection{~} % 1.iii

Let $g = \begin{pmatrix} 1 & x & z \\ 0 & 1 & y \\ 0 & 0 & 1 \end{pmatrix} \in G$ and suppose $g$ has finite order $n > 0$, so $g^n = I_3$. This means that $\phi(g^n) = {\phi(g)}^n = \phi(I_3) = (0, 0)$. We can easily see that this requires $x$ and $y$ in $g$ to be 0, so $g$ has the form of an element of $H$.

Let $h = \begin{pmatrix} 1 & 0 & z \\ 0 & 1 & 0 \\ 0 & 0 & 1 \end{pmatrix} \in H$ and suppose $h$ has finite order $m > 0$. Trivially, $h^m = \begin{pmatrix} 1 & 0 & mz \\ 0 & 1 & 0 \\ 0 & 0 & 1 \end{pmatrix}$, so to get $h^m = I_3$, we need $mz = 0$. Since $m > 0$, this means the only elements of $H$ that have finite order are those with $z = 0$. That is, the only element of finite order is $I_3$.

Now we return to $g$ and observe further that $z$ must be 0 in $g$. Therefore the only element of $G$ that has finite order is $I_3$.

\hfill $\square$

% }}}

% {{{ Q2
\newquestion{2}

\begin{questionbody}
Let \[
V_4 = \l\{ \mathrm{id}, (1,2)(3,4), (1,3)(2,4), (1,4)(2,3) \r\}.
\] You may assume $V_4$ is a normal subgroup of $S_4$.
%
\begin{enumerate}[(i)]
\item Explain why $V_4$ must be a normal subgroup of $A_4$.

\item Let $\sigma$ be a 3-cycle. Show that \[ A_4 / V_4 = \langle \sigma V_4 \rangle . \]

\item Show that $S_4 / V_4$ is a non-cyclic group of order 6.
\end{enumerate}
\end{questionbody}

\subsection{~} % 1.i

$A_4$ is a subgroup of $S_4$ and $V_4$ is a normal subgroup of $S_4$ which also fits the restrictions of $A_4$ (every element is even). Therefore $V_4$ is a subgroup of $A_4$.

To show that $V_4$ is normal in $A_4$, we can show that $v a v^{-1} = a$ for all $v \in V_4$ and $a \in A_4$. This is clearly true for $v = \mathrm{id}$. We can imagine $v^{-1}$ as relabelling all the elements of whatever we're permuting. Then we apply $a$, and then $v$ does the relabelling in reverse, so doing $v a v^{-1}$ has the effect of just doing $a$. Therefore $V_4$ is normal in $A_4$.

\subsection{~} % 1.ii

Any 3-cycle $\sigma$ has order 3 since $\sigma^3 = \mathrm{id}$, therefore $\langle \sigma V_4 \rangle$ has order 3. That means we should be able to divide $A_4$ into 3 classes, each of which can be mapped to a power of $\sigma$.
% Since the order of $A_4$ is $\f{4!}2 = 12$, each class has 4 elements.

Note that every non-identity element of $V_4$ is two disjoint swaps. Every element of $A_4$ is either a product of two disjoint swaps (or the identity), or it is a product of two non-disjoint swaps with two disjoint swaps. The first class are the ones that map to $\sigma^0$, since they don't need to be changed, and the second group can be split into two halves which map to $\sigma$ and $\sigma^2$ respectively.

Therefore $A_4 / V_4 = \langle \sigma V_4 \rangle$.

\newpage
\subsection{~} % 1.iii

$S_4 / V_4$ is, in a way, 2-cyclic. So we have the cyclic subgroup $A_4 / V_4$ from part~\textbf{(ii)}, and another cyclic almost-subgroup\footnote{It can't be a proper subgroup because it doesn't contain the identity element.} formed of the odd permutations from $S_4$.

We know $\# (A_4 / V_4) = \# \langle \sigma V_4 \rangle = 3$, and by a similar argument to that in part~\textbf{(ii)}, the `order' of the almost-subgroup is also 3, since it will contain 3 unique elements. Therefore $\# (S_4 / V_4) = 6$ and it is not cyclic.

% }}}

\end{document}
