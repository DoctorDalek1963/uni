% vim: set foldmethod=marker foldlevel=0:

\documentclass[a4paper]{article}
\usepackage[UKenglish]{babel}

\usepackage{preamble}

\usepackage{array}

\renewcommand{\thesubsection}{Q\arabic{section}~(\roman{subsection})}

\fancyhead[L]{MA268 Assignment 2}
\title{MA268 Algebra 3, Assignment 2}
\colorlet{questionbodycolor}{violet!50}

\begin{document}

\maketitle

\setlength{\parindent}{0em}
\setlength{\parskip}{1em}

% {{{ Q1
\question{1}

\begin{questionbody}
Let $n \ge 3$. Recall that $D_{2n} = \grpprs{r, s}{r^n = s^2 = \mathrm{id}, srs = r^{-1}}$. We follow the convention of writing elements of $D_{2n}$ as $r^k$ or $sr^k$ where $k$ only matters modulo~$n$.
\begin{enumerate}[(i)]
\item Show that $r^k \cdot s = sr^{-k}$.

\item Complete the following multiplication rules for $D_{2n}$:
\[
\setlength{\extrarowheight}{3pt}% local setting
\begin{array}{c||c|c} % chktex 44
    & r^k & s r^k \\
\hline \hline % chktex 44
r^\ell && \\
\hline % chktex 44
s r^\ell &&
\end{array}
\]
\end{enumerate}
\end{questionbody}

\subsection{~} % 1.i

We know that $srs = r^{-1}$, so we can derive that $rs = s^{-1} r^{-1} = sr^{-1}$, since $s$ is self-inverse. Then we have
% \begin{align*}
% r^k s &= \underbrace{r \dotsm r}_{k \text{ times}} s \\
% &= \underbrace{r \dotsm r}_{k-1 \text{ times}} s r^{-1} \\
% &= \underbrace{r \dotsm r}_{k-2 \text{ times}} s r^{-1} r^{-1} \\
% % &\quad \vdots \\
% &= s \underbrace{r^{-1} \dotsm r^{-1}}_{k \text{ times}} \\
% &= sr^{-k}
% \end{align*}
\begin{align*}
r^k s &= r^{k-1} r s \\
&= r^{k-1} s r^{-1} \\
&= r^{k-2} r s r^{-1} \\
&= r^{k-2} s r^{-2} \\
&\,\ \vdots \\
&= s r^{-k}
\end{align*}

\subsection{~} % 1.ii

\[
\setlength{\extrarowheight}{3pt}% local setting
\begin{array}{c||c|c} % chktex 44
    & r^k & s r^k \\
\hline \hline % chktex 44
r^\ell & r^{\ell + k} & sr^{k - \ell} \\
\hline % chktex 44
s r^\ell & sr^{\ell + k} & r^{k - \ell}
\end{array}
\]

% }}}

% {{{ Q2
\newquestion{2}

\begin{questionbody}
Determine all homomorphisms $D_{2n} \to \mathbb{C}^*$.
\end{questionbody}

The obvious homomorphism is $\phi(r) = \e^{2 i \pi / n}$, $\phi(s) = -1$.
No. How do we do reflection? I don't think $\C$ can be reflected.

% }}}

% {{{ Q3
\newquestion{3}

\begin{questionbody}
Recall the rotation matrix $R_\theta = \begin{pmatrix} \cos \theta & -\sin \theta \\ \sin \theta & \cos \theta \end{pmatrix}$. This represents anti-clockwise rotation through angle $\theta$. It is obvious geometrically, and easy to check using trig identities that $R_\theta R_\phi = R_{\theta + \phi}$. Let $S = \begin{pmatrix} 1 & 0 \\ 0 & -1 \end{pmatrix}$.
\begin{enumerate}[(i)]
\item What does multiplying a vector by $S$ do geometrically?

\item Show that there is a unique homomorphism \[ \rho : D_{2n} \to \mathrm{GL}_2(\R), \qquad \rho(r) = R_{2\pi/n}, \qquad \rho(s) = S. \]
\end{enumerate}
\end{questionbody}

\subsection{~} % 3.i

Multiplying a vector by $S$ reflects it in the $x$-axis.

\subsection{~} % 3.ii

Clearly $\rho$ is a homomorphism since it preserves the relations that define $D_{2n}$: \[ r^n = s^2 = \mathrm{id}, srs = r^{-1} \]
\begin{align*}
\rho(r^n) &= \rho(\mathrm{id}) \\
&= I \\
&= {(R_{2\pi/n})}^n \\
&= {\rho(r)}^n \\[1ex]
%
\rho(s^2) &= \rho(\mathrm{id}) \\
&= I \\
&= S^2 \\
&= {\rho(s)}^2 \\[1ex]
%
\rho(srs) &= \rho(r^{-1}) \\
&= {(R_{2\pi/n})}^{-1} \\
&= R_{-2\pi/n} \\
&= S R_{2\pi/n} S \\
&= \rho(s) \rho(r) \rho(s)
\end{align*}

% TODO: Show unique

% }}}

% {{{ Q4
\newquestion{4}

\begin{questionbody}
Let \[
G = \grpprs{x, y}{x^4 = y^5 = 1, xy = y^2x}.
\] It can be shown (you don't have to) that $\# G = 20$ and that every element of $G$ can be written uniquely as $y^b x^a$ where $b \in \{0,1,2,3,4\}$ and $a \in \{0,1,2,3\}$. Complete the following table of multiplication rules for $G$. \textbf{Hint}: Start by proving that $xy^b = y^{2b}x$.

\[
\setlength{\extrarowheight}{3pt}% local setting
\begin{array}{c||*{3}{c|}c} % chktex 44
    & y^k & y^k x & y^k x^2 & y^k x^3 \\
\hline \hline % chktex 44
y^\ell &&&& \\
\hline % chktex 44
y^\ell x &&&& \\
\hline % chktex 44
y^\ell x^2 &&&& \\
\hline % chktex 44
y^\ell x^3 &&&&
\end{array}
\]
\end{questionbody}

We will start by proving that $xy^b = y^{2b}x$.

\begin{align*}
x y^b &= x y y^{b-1} \\
&= y^2 x y^{b-1} \\
&= y^2 x y y^{b-2} \\
&= y^4 x y^{b-2} \\
&\,\ \vdots \\
&= y^{2b} x
\end{align*}

\hfill $\square$

Thus the multiplication table is
\[
\setlength{\extrarowheight}{3pt}% local setting
\begin{array}{c||*{3}{c|}c} % chktex 44
    & y^k & y^k x & y^k x^2 & y^k x^3 \\
\hline \hline % chktex 44
y^\ell & y^{\ell + k} & y^{\ell + k}x & y^{\ell + k}x^2 & y^{\ell + k}x^3 \\
\hline % chktex 44
y^\ell x & y^{\ell + 2k}x & y^{\ell + 2k}x^2 & y^{\ell + 2k}x^3 & y^{\ell + 2k} \\
\hline % chktex 44
y^\ell x^2 & y^{\ell + 4k}x^2 & y^{\ell + 4k}x^3 & y^{\ell + 4k} & y^{\ell + 4k}x \\
\hline % chktex 44
y^\ell x^3 & y^{\ell + 8k}x^3 & y^{\ell + 8k} & y^{\ell + 8k}x & y^{\ell + 8k}x^2
\end{array}
\]

% }}}

% {{{ Q5
\newquestion{5}

\begin{questionbody}
\begin{enumerate}[(i)]
\item Let $n \ge 3$. Determine the elements of order 2 in $D_{2n}$.

\item Determine the elements of order 2 in $Q_8$.

\item Show that $Q_8 \not\cong D_{2n}$ for all $n \ge 3$.
\end{enumerate}
\end{questionbody}

\subsection{~} % 3.i

Clearly all reflections are order 2, and there are $n$ reflections. If $n$ is even, then $r^{n/2}$ is also of order 2. Therefore there are $n$ or $n+1$ elements of order 2 in $D_{2n}$.

\subsection{~} % 3.ii
\subsection{~} % 3.iii

% }}}

\end{document}
