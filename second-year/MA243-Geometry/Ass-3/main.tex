% vim: set foldmethod=marker foldlevel=0:

\documentclass[a4paper]{article}
\usepackage[UKenglish]{babel}

\usepackage[hidelinks]{hyperref}

\usepackage{preamble}

\usepackage{tikz}
\usetikzlibrary{calc, math}

\fancyhead[L]{MA243 Assignment 3}
\title{MA243 Geometry, Assignment 3}
\colorlet{questionbodycolor}{orange!50}

\begin{document}

\maketitle

\setlength{\parindent}{0em}
\setlength{\parskip}{1em}

% {{{ Q1
\question{1}

\begin{questionbody}
\begin{enumerate}[(a)]
\item Define, by analogy with Euclidean geometry, the notions of spherical circle and spherical disc with centre $P \in S^2$ and radius $\rho$.

\item Prove that a spherical circle of radius $\rho < \pi$ has (Euclidean) circumference $2\pi \sin \rho$.

\item Prove that a spherical disc of radius $\rho < \pi$ has area $2\pi (1 - \cos \rho)$.
% (\textbf{Hint}:~Integrate~\textbf{(b)}).

\item Let $0 < \rho < \f\pi2$ and let $C$ be the spherical circle with centre $P \in S^2$ and radius $\rho$. Show that $(C, d_{S^2})$ is not isometric to $(S_r^1, d_{S_r^1})$ for any $r$.

\item Show (e.g.\ using ideas from~\textbf{(d)}) that there is no isometry from any region in $S^2$ to any region in $\R^2$.%
% \footnotemark{}
\end{enumerate}
\end{questionbody}
% \footnotetext{An ad-hoc definition of region here is a non-empty subset $U$ which, for any $P \in U$, contains the disc with centre $P$ and sufficiently small $\rho$.}

\subsection{~} % 1.a

A spherical circle with centre $P \in S^2$ and radius $\rho$ is the set \[ \l\{ x \in S^2 : d_{S^2}(x, P) = \rho \r\}. \]
A (closed) spherical disc with centre $P \in S^2$ and radius $\rho$ is the set \[ \l\{ x \in S^2 : d_{S^2}(x, P) \le \rho \r\}. \]
And of course an open spherical disc is \[ \l\{ x \in S^2 : d_{S^2}(x, P) < \rho \r\}. \]

\newpage
\subsection{~} % 1.b

Consider a cross-section of $S^2$:

\begin{figure}[tbhp]
\centering
\tikzmath{
    \p = 30; % 90 - \rho
}
\begin{tikzpicture}[
    scale=2,
]
    \draw (0, 0) circle[radius=1];

    \draw (0, 0) -- (0, 1);
    % \draw (0, -1) -- (0, 1);
    \draw (0, 0) -- ($(\p:1)$);

    % Outer length label
    \draw (90:1.1) -- (90:1.3);
    \draw ($(\p:1.1)$) -- ($(\p:1.3)$);
    \draw ($(\p:1.2)$) arc (\p:90:1.2);
    \node () at ($(45 + \p/2:1.4)$) {$\rho$};

    % Inner angle label
    \draw ($(\p:0.3)$) arc (\p:90:0.3);
    \node () at ($(45 + \p/2:0.17)$) {$\rho$};

    % Projection onto vertical line
    \draw ($(\p:1)$) -- ($(0, -1)!(\p:1)!(0, 1)$);

    % Manually positioned :(
    \node () at (0.4, 0.6) {$\sin \rho$};

    \fill ($(\p:1)$) circle[radius=0.035];
\end{tikzpicture}
\end{figure}
The spherical circle is the black dot rotated about the vertical line. The circumference of this circle is $2\pi r$ where $r = \sin \rho$ as seen in the diagram.

\subsection{~} % 1.c

The area is of course just the integral of the circumferences of all smaller circles,
\begin{align*}
\intlim 0\rho {2\pi \sin r} r &= 2\pi \intlim 0\rho {\sin r} r \\[0.5ex]
&= 2\pi {\l[ -\cos r \r]}_0^\rho \\[0.5ex]
&= 2\pi \l( -\cos \rho - (-\cos 0) \r) \\[0.5ex]
&= 2\pi \l( 1 - \cos \rho \r)
\end{align*}
as required.

\subsection{~} % 1.d

% TODO: This is a bad explanation
The shortest path between any two points in $S_r^1$ is an arc of the circle. But the shortest path between any two points in $C$ is an arc going inside $C$. This is because the shortest path between two points in $S^2$ is a great circle, which will always pass through the interior of $C$ since $\rho < \f\pi2$. Because distances are measured in completely different ways in the two spaces, they cannot be isometric.

\subsection{~} % 1.e

If there were such an isometry, then it would preserve the areas of spherical discs. Using the same logic as in part~\textbf{(d)}, this is not possible.

% }}}

% {{{ Q2
\newquestion{2}

\begin{questionbody}
Suppose $P$ and $Q$ are two distinct points in the hyperbolic plane $\cal H^2$ and let $U := P + Q$ be their vector sum. Show that the mid-point of the hyperbolic line between $P$ and $Q$ is the point \[
R = \f{U}{\sqrt{-U \cdot_L U}}
\] (where for a positive real number we take the positive square root).

In other words, you need to prove:
\begin{enumerate}[(a)]
\item $-U \cdot_L U$ is a positive real number.

\item $R \in \cal H^2$.

\item $d_{\cal H^2}(R, P) = d_{\cal H^2}(R, Q)$.
\end{enumerate}
\end{questionbody}

Let $P = (x_0, x_1, x_2)$ and $Q = (y_0, y_1, y_2)$. Therefore $U = ({x_0 + y_0}, {x_1 + y_1}, {x_2 + y_2})$. Since $P, Q \in \cal H^2$, we know that $x_0 > 0$, $y_0 > 0$, and
\begin{align*}
P \cdot_L P = -x_0^2 + x_1^2 + x_2^2 &= -1, \\[0.5ex]
Q \cdot_L Q = -y_0^2 + y_1^2 + y_2^2 &= -1.
\end{align*}

\subsection{~} % 2.a

% \begin{align*}
% -U \cdot_L U &= -(-1){(x_0 + y_0)}^2 + (-1){(x_1 + y_1)}^2 + (-1){(x_2 + y_2)}^2 \\[0.5ex]
% &= {(x_0 + y_0)}^2 - {(x_1 + y_1)}^2 - {(x_2 + y_2)}^2 \\[0.5ex]
% &= x_0^2 + 2 x_0 y_0 + y_0^2 - x_1^2 - 2 x_1 y_1 - y_1^2 - x_2^2 - 2 x_2 y_2 - y_2^2 \\[0.5ex]
% &= -(-x_0^2 + x_1^2 + x_2^2) - (-y_0^2 + y_1^2 + y_2^2) + 2 x_0 y_0 - 2 x_1 y_1 - 2 x_2 y_2 \\[0.5ex]
% &= 2 + 2 x_0 y_0 - 2 x_1 y_1 - 2 x_2 y_2 \\[0.5ex]
% &= 2(1 + x_0 y_0 - x_1 y_1 - x_2 y_2) \\[0.5ex]
% \end{align*}

\begin{align*}
-U \cdot_L U &= -(P + Q) \cdot_L (P + Q) \\[0.5ex]
&= (-P - Q) \cdot_L (P + Q) \\[0.5ex]
&= -P \cdot_L P - P \cdot_L Q - Q \cdot_L P - Q \cdot_L Q \\[0.5ex]
&= 1 - P \cdot_L Q - Q \cdot_L P + 1 \\[0.5ex]
&= 2 + \cosh\l( \cosh^{-1}(-P \cdot_L Q) \r) + \cosh\l( \cosh^{-1}(-Q \cdot_L P) \r) \\[0.5ex]
&= 2 + \cosh( d_{\cal H^2}(P, Q) ) + \cosh( d_{\cal H^2}(Q, P) )
\end{align*}
Since distances are positive real numbers and $\cosh \ge 1$, $-U \cdot U \ge 4$.

\subsection{~} % 2.b

To have $R \in \cal H^2$, we need $\|R\|_L = i$. That is, $\|R\|_L^2 = -1$. We have
\begin{align*}
\|R\|_L^2 &= R \cdot_L R \\[0.5ex]
&= \f1{-U \cdot_L U} U \cdot_L U \\[0.5ex]
&= \f{U \cdot_L U}{-U \cdot_L U} \\[0.5ex]
&= -1
\end{align*}
Therefore $R \in \cal H^2$.

\subsection{~} % 2.c

% \begin{align*}
% d_{\cal H^2}(R, P) &= \cosh^{-1} \l( - R \cdot_L P \r) \\[0.5ex]
% &= \cosh^{-1} \l( - \f{U}{\sqrt{-U \cdot_L U}} \cdot_L P \r) \\[0.5ex]
% &= \cosh^{-1} \l( \f{- U \cdot_L P}{\sqrt{-U \cdot_L U}} \r) \\[0.5ex]
% &= \cosh^{-1} \l( \f{x_0 (x_0 + y_0) - x_1 (x_1 + y_1) - x_2 (x_2 + y_2)}{\sqrt{-U \cdot_L U}} \r) \\[0.5ex]
% \end{align*}

$d_{\cal H^2}(R, P) = d_{\cal H^2}(R, Q)$ means $\cosh^{-1}(-R \cdot_L P) = \cosh^{-1}(-R \cdot_L Q)$. Since $\cosh^{-1}$ is a strictly increasing function, this is equivalent to saying that ${-R \cdot_L P} = {-R \cdot_L Q}$.

\begin{align*}
- R \cdot_L P &= - \f{U}{\sqrt{-U \cdot_L U}} \cdot_L P \\[0.5ex]
&= \f{-U \cdot_L P}{\sqrt{-U \cdot_L U}} \\[0.5ex]
&= \f{x_0 (x_0 + y_0) - x_1 (x_1 + y_1) - x_2 (x_2 + y_2)}{\sqrt{-U \cdot_L U}} \\[0.5ex]
&= \f{x_0^2 + x_0 y_0 - x_1^2 - x_1 y_1 - x_2^2 - x_2 y_2}{\sqrt{-U \cdot_L U}} \\[0.5ex]
&= \f{x_0^2 - x_1^2 - x_2^2}{\sqrt{-U \cdot_L U}} + \f{x_0 y_0 - x_1 y_1 - x_2 y_2}{\sqrt{-U \cdot_L U}} \\[0.5ex]
&= \f{-P \cdot_L P}{\sqrt{-U \cdot_L U}} + \f{-P \cdot_L Q}{\sqrt{-U \cdot_L U}} \\[0.5ex]
&= \f{1 - P \cdot_L Q}{\sqrt{-U \cdot_L U}} \\
\displaybreak \\
- R \cdot_L Q &= - \f{U}{\sqrt{-U \cdot_L U}} \cdot_L Q \\[0.5ex]
&= \f{-U \cdot_L Q}{\sqrt{-U \cdot_L U}} \\[0.5ex]
&= \f{y_0 (x_0 + y_0) - y_1 (x_1 + y_1) - y_2 (x_2 + y_2)}{\sqrt{-U \cdot_L U}} \\[0.5ex]
&= \f{y_0^2 + x_0 y_0 - y_1^2 - x_1 y_1 - y_2^2 - x_2 y_2}{\sqrt{-U \cdot_L U}} \\[0.5ex]
&= \f{y_0^2 - y_1^2 - y_2^2}{\sqrt{-U \cdot_L U}} + \f{x_0 y_0 - x_1 y_1 - x_2 y_2}{\sqrt{-U \cdot_L U}} \\[0.5ex]
&= \f{-Q \cdot_L Q}{\sqrt{-U \cdot_L U}} + \f{-P \cdot_L Q}{\sqrt{-U \cdot_L U}} \\[0.5ex]
&= \f{1 - P \cdot_L Q}{\sqrt{-U \cdot_L U}}
\end{align*}

Therefore $-R \cdot_L P = -R \cdot_L Q$ and therefore $d_{\cal H^2}(R, P) = d_{\cal H^2}(R, Q)$.

Therefore $R$ is the midpoint of $P$ and $Q$.

\hfill $\square$

% }}}

% {{{ Q3
\newquestion{3}

\begin{questionbody}
A hyperbolic line $L \subset \cal H^2$ is the non-empty intersection $\cal H^2 \cap V$ with a plane $V$ through the origin in $\R^3$. Prove that any hyperbolic line on $\cal H^2$ that passes through the point $(1, 0, 0)$ projects to a straight line in $x_1 x_2$-plane. (Here, the projection is given by $(x_0, x_1, x_2) \mapsto (x_1, x_2)$).
\end{questionbody}

Any such line contains the point $(1, 0, 0)$ and some other point $(y_0, y_1, y_2)$. The plane $V$ defining this line must therefore contain the vectors \[
\begin{pmatrix} 1 \\ 0 \\ 0 \end{pmatrix}, \quad
\begin{pmatrix} y_0 \\ y_1 \\ y_2 \end{pmatrix}.
\] Or we can instead use the orthogonal vectors \[
\begin{pmatrix} 1 \\ 0 \\ 0 \end{pmatrix}, \quad
\begin{pmatrix} 0 \\ y_1 \\ y_2 \end{pmatrix}.
\]

Therefore any point on the hyperbolic line must be of the form \[
\lambda \begin{pmatrix} 1 \\ 0 \\ 0 \end{pmatrix} +
\mu \begin{pmatrix} 0 \\ y_1 \\ y_2 \end{pmatrix}
\] for some $\lambda, \mu \in \R$, and so the projection of any point is $(\mu y_1, \mu y_2)$. The set of projected points is therefore a straight line in $\R^2$.

\hfill $\square$

% }}}

\end{document}
