% vim: set foldmethod=marker foldlevel=0:

\documentclass[a4paper]{article}
\usepackage[UKenglish]{babel}

\usepackage[hidelinks]{hyperref}

\usepackage{preamble}

% \usepackage{graphicx}
% \graphicspath{ {./imgs/} }

\fancyhead[L]{MA243 Assignment 3}
\title{MA243 Geometry, Assignment 3}
\colorlet{questionbodycolor}{orange!50}

\begin{document}

\maketitle

\setlength{\parindent}{0em}
\setlength{\parskip}{1em}

% {{{ Q1
\question{1}

\begin{questionbody}
\begin{enumerate}[(a)]
\item Define, by analogy with Euclidean geometry, the notions of spherical circle and spherical disc with centre $P \in S^2$ and radius $\rho$.

\item Prove that a spherical circle of radius $\rho < \pi$ has (Euclidean) circumference $2\pi \sin \rho$.

\item Prove that a spherical disc of radius $\rho < \pi$ has are $2\pi (1 - \cos \rho)$. (\textbf{Hint}:~Integrate~\textbf{(b)}).

\item Let $0 < \rho < \pi/2$. Show that the spherical circle with centre $P \in S^2$ and radius $\rho$ is not isometric to $S_r^1$ for any $r$.

\item Show (e.g.\ using~\textbf{(d)}) that there is no isometry from any region in $S^2$ to any region in $\R^2$.\footnotemark{}
\end{enumerate}
\end{questionbody}
\footnotetext{An ad-hoc definition of region here is a non-empty subset $U$ which, for any $P \in U$, contains the disc with centre $P$ and sufficiently small $\rho$.}

\subsection{~} % 1.a

Answer

\subsection{~} % 1.b

Answer

\subsection{~} % 1.c

Answer

\subsection{~} % 1.d

Answer

\subsection{~} % 1.e

Answer

% }}}

% {{{ Q2
\newquestion{2}

\begin{questionbody}
Suppose $P$ and $Q$ are two distinct points in the hyperbolic plane $\cal H^2$ and let $U := P + Q$ be their vector sum. Show that the mid-point of the hyperbolic line between $P$ and $Q$ is the point \[
R = \f{U}{\sqrt{-U \cdot_L U}}
\] (where for a positive real number we take the positive square root).

In other words, you need to prove:
\begin{enumerate}[(a)]
\item $-U \cdot_L U$ is a positive real number.

\item $R \in \cal H^2$.

\item $d_{\cal H^2}(R, P) = d_{\cal H^2}(R, Q)$.
\end{enumerate}
\end{questionbody}

\subsection{~} % 2.a

Answer

\subsection{~} % 2.b

Answer

\subsection{~} % 2.c

Answer

% }}}

% {{{ Q3
\newquestion{3}

\begin{questionbody}
A hyperbolic line $L \subset \cal H^2$ is the non-empty intersection $\cal H^2 \cap V$ with a plane $V$ through the origin in $\R^3$. Prove that any hyperbolic line on $\cal H^2$ that passes through the point $(1, 0, 0)$ projects to a straight line in $x_1 x_2$-plane. (Here, the projection is given by $(x_0, x_1, x_2) \mapsto (x_1, x_2)$).
\end{questionbody}

Answer

% }}}

\end{document}
