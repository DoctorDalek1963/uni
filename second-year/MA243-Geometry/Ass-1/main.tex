% vim: set foldmethod=marker foldlevel=0:

\documentclass[a4paper]{article}
\usepackage[UKenglish]{babel}

\usepackage{preamble}

\usepackage{tikz}

\fancyhead[L]{MA243 Assignment 1}
\title{MA243 Geometry, Assignment 1}
\colorlet{questionbodycolor}{orange!50}

\begin{document}

\maketitle

\setlength{\parindent}{0em}
\setlength{\parskip}{1em}

% {{{ Q1
\question{1}

\begin{questionbody}
Write down all the symmetries of the rectangle $X$ in $\R^2$ with vertices at \[
(1, 0), (1, 4), (3, 0), (3, 4)
\] That is, \[
X = \l\{ (x, y) \in \R^2 \colon 1 \le x \le 3, 0 \le y \le 4 \r\}
\]

Write all your symmetries as affine transformations of the form \[
T(\mathbf v) = A \mathbf v + \mathbf b
\] where $A$ is a $2 \times 2$ matrix and $\mathbf b$ is a vector in $\R^2$.
\end{questionbody}

\begin{center}
    \begin{tikzpicture}
        \draw (1, 0) -- (1, 4) -- (3, 4) -- (3, 0) -- cycle;

        \node at (1, 0) [label={225:$(1, 0)$}] {};
        \node at (3, 0) [label={315:$(3, 0)$}] {};
        \node at (1, 4) [label={135:$(1, 4)$}] {};
        \node at (3, 4) [label={45:$(3, 4)$}] {};

        \draw[dashed] (2, -0.2) -- (2, 4.2);
        \draw[dashed] (0.8, 2) -- (3.2, 2);
    \end{tikzpicture}
\end{center}

The rectangle $X$ has 4 symmetries, the identity, a rotation by $\pi$ followed by a translation, a reflection in $y=2$, and a reflection in $x=2$. We shall write these as affine transformations with matrices.

The identity is of course \[ T(\mathbf v) = \begin{pmatrix} 1 & 0 \\ 0 & 1 \end{pmatrix} \mathbf v + \begin{pmatrix} 0 \\ 0 \end{pmatrix} \]

The rotation and translation is \[ T(\mathbf v) = \begin{pmatrix} -1 & 0 \\ 0 & -1 \end{pmatrix} \mathbf v + \begin{pmatrix} 4 \\ 4 \end{pmatrix} \]

The reflection in $y=2$ is \[ T(\mathbf v) = \begin{pmatrix} 1 & 0 \\ 0 & -1 \end{pmatrix} \mathbf v + \begin{pmatrix} 0 \\ 4 \end{pmatrix} \]

And the reflection in $x=2$ is \[ T(\mathbf v) = \begin{pmatrix} -1 & 0 \\ 0 & 1 \end{pmatrix} \mathbf v + \begin{pmatrix} 4 \\ 0 \end{pmatrix} \]

% }}}

% {{{ Q2
\newquestion{2}

\begin{questionbody}
Suppose that a function $f \colon X \times X \to \R$ is non-degenerate, symmetric, and satisfies the triangle inequality. Then \begin{enumerate}[(a)]
\item Prove that the image of $f$ is in $[0, \infty)$ (hence $f$ is a metric). % chktex 9
\item When is $f$ injective? Explain.
\item Suppose $X = \R$. Does $f$ have to be surjective onto $[0, \infty)$? Either prove this or give a counterexample. % chktex 9
\end{enumerate}
\end{questionbody}

\subsection{~} % 2.a

Since the codomain of $f$ is $\R$, we know the image of $f$ will be a subset of $\R$, so we only need to prove $f \ge 0$.

Suppose there exist $a, b, c \in X$ with $a \ne b \ne c$ and $f(a, b) < 0$. Since $f$ satisfies the triangle inequality, we know three things: \begin{align*}
f(a, b) \le f(a, c) + f(b, c) \\
f(b, c) \le f(a, b) + f(a, c) \\
f(a, c) \le f(a, b) + f(b, c)
\end{align*}

TODO

\subsection{~} % 2.b

TODO

\subsection{~} % 2.c

TODO

% }}}

% {{{ Q3
\newquestion{3}

\begin{questionbody}
Let $\delta : \R^2 \times \R^2 \to [0, \infty)$ be a function defined by \[ % chktex 9
\delta\big( (x_1, y_1), (x_2, y_2) \big) = \sqrt{{(x_1 - x_2)}^2 + 2{(y_1 - y_2)}^2}
\] Is $(\R^2, \delta)$ a metric space? Prove it is, or explain why it isn't.
\end{questionbody}

For $(\R^2, \delta)$ to be a metric space, $\delta$ must be a metric. This means is must be non-degenerate, symmetric, and satisfy the triangle inequality.

The only way to make $\delta = 0$ is to make everything under the square root equal to 0, which means making $x_1 - x_2 = 0$ and $y_1 - y_2 = 0$. That means $x_1 = x_2$ and $y_1 = y_2$, so $\delta(\ul u, \ul v) = 0$ only when $\ul u = \ul v$ and therefore $\delta$ is non-degenerate.

Since the terms $x_1 - x_2$ and $y_1 - y_2$ are both squared, their order is unimportant, so $\delta$ is symmetric.

For the triangle inequality, let $\ul x, \ul y, \ul z \in \R^n$. We can use translation invariance to move $\ul z$ to the origin, then let $\ul x' = \ul x - \ul z$, $\ul y' = \ul y - \ul z$, $\ul z' = \ul z - \ul z = \ul 0$.

Now we want to show that $\delta(\ul x, \ul y) \le \delta(\ul x, \ul z) + \delta(\ul y, \ul z)$. By translation invariance again, this is equivalent to $\delta(\ul x', \ul y') \le \delta(\ul x', \ul 0) + \delta(\ul y', \ul 0)$, which is equivalent to $\| \ul x' - \ul y' \| \le \| \ul x' \| + \| \ul y' \|$. Since both sides are non-negative, it suffices to square both sides and show the resulting equation holds.
\begin{align*}
{\l( \| \ul x' \| + \| \ul y' \| \r)}^2 &=
  {\| \ul x' \|}^2 + 2 {\| \ul x' \|} \, {\| \ul y' \|} + {\| \ul x' \|}^2 \\
&\ge {\| \ul x' \|}^2 + 2 {| \ul x' \cdot \ul y' |} + {\| \ul x' \|}^2 \\
&\ge {\| \ul x' \|}^2 - 2 (\ul x' \cdot \ul y') + {\| \ul x' \|}^2 \\
&= \ul x' \cdot \ul x' - 2 (\ul x' \cdot \ul y') + \ul y' \cdot \ul y' \\
&= (\ul x' - \ul y') \cdot (\ul x' - \ul y') \\
&= {\| \ul x' - \ul y' \|}^2
\end{align*}

Since $\delta$ is non-degenerate, symmetric, and preserves the triangle inequality, it is a metric and therefore $(\R^2, \delta)$ is a metric space.

\hfill $\square$

% }}}

% {{{ Q4
\newquestion{4}

\begin{questionbody}
Let $L$ be the line in $\R^2$ given by \[
L = \big\{ (x, y) \in \R^2 \colon y - 2x = 1 \big\}
\]

Define a metric $d$ on $L$ given by the restriction of the Euclidean metric on $\R^2$ to $L$. That is, for points $(x_1, y_2)$, $(x_2, y_2)$, we set \[
d\big( (x_1, y_1), (x_2, y_2) \big) = \sqrt{{(x_1 - x_2)}^2 + {(y_1 - y_2)}^2}
\]

Find an isometry $f \colon L \to \R$, where $\R$ has the usual Euclidean metric, $d_1(x, y) = |x - y|$.
\end{questionbody}

\begin{center}
    \begin{tikzpicture}
        \draw[->] (-1.5, 0) -- (1.5, 0);
        \draw[->] (0, -1.5) -- (0, 3.5);

        \node at (-1, -1) [label={135:$(-1, -1)$}] {};
        \fill (-1, -1) circle[radius=0.05];
        \node at (0, 1) [label={135:$(0, 1)$}] {};
        \fill (0, 1) circle[radius=0.05];
        \node at (1, 3) [label={315:$(1, 3)$}] {};
        \fill (1, 3) circle[radius=0.05];

        \draw (-1.25, -1.5) -- (1.25, 3.5);
    \end{tikzpicture}
\end{center}

% On $L$, $d\big( (0, 1), (1, 3) \big) = \sqrt5$, so the isometry $f$ needs to preserve this distance. We can't just project the line $L$ down onto the $x$-axis, we need to also scale it by a factor of $\sqrt5$. Let \[ f(x, y) = x\sqrt5 \]

Consider two points $(x_1, 2 x_1 + 1)$ and $(x_2, 2 x_2 + 1)$ on $L$. The distance between them is \begin{align*}
d\big( (x_1, 2 x_1 + 1), (x_2, 2 x_2 + 1) \big) &= \sqrt{{(x_1 - x_2)}^2 + {(2 x_1 + 1 - 2 x_2 - 1)}^2} \\
&= \sqrt{{(x_1 - x_2)}^2 + {(2 (x_1 - x_2))}^2} \\
&= \sqrt{{(x_1 - x_2)}^2 + 4 {(x_1 - x_2)}^2} \\
&= \sqrt{5 {(x_1 - x_2)}^2} \\
&= \sqrt5\; |x_1 - x_2|
\end{align*}

This is the normal Euclidean metric scaled by $\sqrt5$, so we just project $L$ down onto the $x$-axis and apply this scaling factor. So let $f(x, y) = x \sqrt5$.

% }}}

\end{document}
