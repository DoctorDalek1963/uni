% vim: set foldmethod=marker foldlevel=0:

\documentclass[a4paper]{article}
\usepackage[UKenglish]{babel}

\usepackage[hidelinks]{hyperref}

\usepackage{preamble}

\usepackage{tikz}

\fancyhead[L]{MA243 Assignment 4}
\title{MA243 Geometry, Assignment 4}
\colorlet{questionbodycolor}{orange!50}

\begin{document}

\maketitle

\setlength{\parindent}{0em}
\setlength{\parskip}{1em}

% {{{ Q1
% \question{1}

% \begin{questionbody}
% The goal of this exercise is to show that the isometry $\sigma : \cal H^2 \to \cal H^2_P$ from class takes hyperbolic lines to either, straight lines through the origin, or arcs of circles orthogonal to the unit circle. Establish this using the following steps:
% \begin{enumerate}[(a)]
% \item Let $\sigma_1 : \cal H^2 \to \{x_0 = 1\}$ be the projection from 0 to the plane $\{x_0 = 1\}$. Show that $\sigma_1$ sends hyperbolic lines to straight (Euclidean) lines.

% \item Let $\sigma_2 : \{x_0 = 1\} \to S^2$ be the vertical projection onto the upper hemisphere. Show that $\sigma_2$ sends straight lines to (arcs of) Euclidean circles perpendicular to the equator.

% \item Let $\sigma_3 : S^2 \to \{x_0 = 0\}$ be the stereographic projection from the south pole $S = (-1, 0, 0)$. Show that $\sigma_3$ sends great circles through $S$ to lines through the origin.

% \item Let $C = S^2 \cap \{x_1 = \rho\}$ be a circle as in~\textbf{(b)}, for some $0 < \rho < 1$, and assume that $\sigma_3(C) \subset \R^2$ is a circle $S^1(M, r)$ with radius $r$ and centre $M$, orthogonal to the unit circle. Find $r$ and $M$.

% \textbf{Hint}: Use that two circles $S^1(M, r)$ and $S^1(M', r')$ are orthogonal if and only if $r^2 + {(r')}^2 = {\l( d_{\R^2}(M, M') \r)}^2$.

% \item Show that $\sigma_3(C)$ is indeed the circle with radius $r$ and centre $M$.

% \item Show that $\sigma = \sigma_3 \circ \sigma_2 \circ \sigma_1$ and conclude.
% \end{enumerate}
% \end{questionbody}

% \subsection{~} % 1.a

% TODO: Finish
% A hyperbolic line is the intersection of $\cal H^2$ with a time-like plane $V$ through the origin.

% Note that since the hyperboloid $\cal H^2$ is contained within the cone $x_0^2 = x_1^2 + x_2^2$, every point in $\cal H^2$ will get projected to a point in the open disk \[ \{ (1, a, b) : a, b \in \R, a^2 + b^2 < 1 \}. \]

% \subsection{~} % 1.b

% Answer

% \subsection{~} % 1.c

% Answer

% \subsection{~} % 1.d

% Answer

% \subsection{~} % 1.e

% Answer

% \subsection{~} % 1.f

% Answer

% }}}

% {{{ Q2
\question{2}

\begin{questionbody}
Which of the following sets of points in $\mathbb{P}^2(\R)$ are collinear (i.e.\ lie on a projective line)? In the collinear case, give an equation in the homogeneous coordinates $(X_0 : X_1 : X_2)$ for the projective line.
\begin{enumerate}[1.]
\item $(2 : 3 : 1), (1 : 3 : 2), (2 : 4 : 2)$
\item $(1 : 2 : 3), (3 : 2 : 1), (2 : 4 : 2)$
\end{enumerate}
\end{questionbody}

% The first set of points are collinear since they form a plane through the origin ($x - y + z = 0$ by GeoGebra).

For the first set of points to form a plane through the origin, we want to find $a, b, c \in \R$ such that $ax + by + cz = 0$, so
\begin{align*}
2a + 3b + c &= 0 \\
a + 3b + 2c &= 0 \\
2a + 4b + 2c &= 0
\end{align*}
Solving this system of linear equations gives $a + b = 0$ and $b + c = 0$, so $a = c$ and $b = -a$. Therefore the plane $x - y + z = 0$ contains these three points, so they are collinear in $\mathbb P^2(\R)$.

The points $(2 : 3 : 1)$ and $(1 : 3 : 2)$ each correspond to a 1-dimensional subspace of $\R^3$. Their span is a 2-dimensional subspace in $\R^3$ and hence a line in $\mathbb P^2(\R)$. This line has equation \[
\angb{\big\{ (2 : 3 : 1), (1 : 3 : 2) \big\}}.
\]

% The second set of points are not collinear since they form a plane that does not include the origin ($x + z =  4$ by GeoGebra).

For the second set of points to form a plane through the origin, we again want to find $a, b, c \in \R$ such that $ax + by + cz = 0$, so
\begin{align*}
a + 2b + 3c &= 0 \\
3a + 2b +  c &= 0 \\
2a + 4b + 2c &= 0
\end{align*}

That means we need \begin{equation}
\begin{pmatrix}
1 & 2 & 3 \\
3 & 2 & 1 \\
2 & 4 & 2
\end{pmatrix}
\begin{pmatrix}
a \\ b \\ c
\end{pmatrix}
= \ul 0 \label{eqn:Q2-2-matrix}
\end{equation}
where at least one of $a, b, c$ are non-zero. That means we need the determinant of this matrix to equal 0 but
\begin{align*}
\begin{vmatrix}
1 & 2 & 3 \\
3 & 2 & 1 \\
2 & 4 & 2
\end{vmatrix} &=
\begin{vmatrix}
2 & 1 \\
4 & 2
\end{vmatrix}
- 2 \begin{vmatrix}
3 & 1 \\
2 & 2
\end{vmatrix}
+ 3 \begin{vmatrix}
3 & 2 \\
2 & 4
\end{vmatrix} \\[0.5ex]
&= 0 - 2 \cdot 4 + 3 \cdot 8 \\
% &= 24 - 8 \\
&= 16 \\
&\ne 0
\end{align*}
Therefore the only $a, b, c$ that satisfy~\eqref{eqn:Q2-2-matrix} are $a = b = c = 0$. Therefore the three points $(1 : 2 : 3), (3 : 2 : 1), (2 : 4 : 2)$ are not collinear in $\mathbb P^2(\R)$.

In contrast, the determinant of the matrix for the first three points would be
\begin{align*}
\begin{vmatrix}
2 & 3 & 1 \\
1 & 3 & 2 \\
2 & 4 & 2
\end{vmatrix} &=
2 \begin{vmatrix}
3 & 2 \\
4 & 2
\end{vmatrix}
-3 \begin{vmatrix}
1 & 2 \\
2 & 2
\end{vmatrix}
+ \begin{vmatrix}
1 & 3 \\
2 & 4
\end{vmatrix} \\[0.5ex]
&= 2 (-2) - 3 (-2) + (-2) \\
&= -4 + 6 -2 \\
&= 0
\end{align*}

% }}}

% {{{ Q3
\newquestion{3}

\begin{questionbody}
Let $\ell_1, \ell_2$ be lines in $\R^2$ defined by $x_0 = -1$ and $x_1 = 3$, respectively. Let $F : \ell_1 \to \ell_2$ be the perspectivity associated with some point $O \in \R^2$ of which we know $O \ne 0, O \notin \ell_1 \cup \ell_2$ (i.e.\ $F$ is given by projection from $O$). We may then write $F(-1, x_1) = (f(x_1), 3)$ for some undetermined function $f : \R \to \R$.
\begin{enumerate}[(a)]
\item Assume we know that $f(1) = 3$ and $f(2) = 0$. Determine $O$.

\item Determine $f(3)$.

\item Consider the (injective) map
\begin{align*}
\ell_1 &\to \mathbb{P}^1, \\
(-1, x_1) &\mapsto (x_1 : 1).
\end{align*}
Find a point in $\mathbb{P}^1 \setminus \ell_1$. Similarly, in the following we will also use the (injective) map
\begin{align*}
\ell_2 &\to \mathbb{P}^1, \\
(x_0, 3) &\mapsto (x_0 : 1).
\end{align*}

\item Let $A$ be an invertible $2 \times 2$ matrix. Then multiplication by $A$ sends lines in $\R^2$ to lines in $\R^2$, thus a map $T_A : \mathbb{P}^1 \to \mathbb{P}^1$. Find $A$ such that $T_A$ restricts to $F : \ell_1 \to \ell_2$.

\item Use the matrix $A$ to determine an expression for $f$ as a ratio of two polynomials.
\end{enumerate}
\end{questionbody}

\begin{figure}[tbhp]
\centering
\begin{tikzpicture}[scale=1.4]
    \draw[->] (-3.5, 0) -- (3.5, 0);
    \foreach \i in {-3, -2, ..., 3} \draw (\i, -0.1) -- (\i, 0.1);
    \draw[->] (0, -0.5) -- (0, 3.5);
    \foreach \i in {1, 2, 3} \draw (-0.1, \i) -- (0.1, \i);

    \draw (-1, -0.5) -- (-1, 3.5);
    \node () at (-1, 3.8) {$\ell_1$};
    \draw (-3.5, 3) -- (3.5, 3);
    \node () at (3.8, 3) {$\ell_2$};

    \node () at (-2.85, -0.2) {$O$};

    \begin{scope}
        \clip (-3.5, -0.5) rectangle (3.5, 3.5);

        \draw (-5, -1) -- (5, 4);
        \draw (-4, -1) -- (1, 4);
        \draw (-4, -1.5) -- (0, 4.5);
    \end{scope}
\end{tikzpicture}
\caption{A diagram of the situation}
\end{figure}

\subsection{~} % 3.a

Since $f(1) = 3$ and $f(2) = 0$, we have $F(-1, 1) = (3, 3)$ and $F(-1, 2) = (0, 3)$. Therefore we can draw lines through these pairs of points, and $O$ will be their intersection. The lines are
\begin{align*}
x_1 &= \f12 x_0 + \f32 \\
x_1 &= x_0 + 3
\end{align*}
or written more nicely,
\begin{align*}
-x_0 + 2 x_1 &= 3 \\
-x_0 + x_1 &= 3
\end{align*}

The intersection of these lines is $x_0 = -3$ and $x_1 = 0$, so $O = (-3, 0)$.

\subsection{~} % 3.b

We know $F(-1, 3) = (f(3), 3)$, so we draw a line through $O$ and $(-1, 3)$ and see where it intersects $\ell_2$. The equation of this line is \[
x_1 = \f32 x_0 + \f92.
\] We plug in $x_1 = 3$ and find $x_0 = -1$, so $f(3) = -1$.

\subsection{~} % 3.c

The point $(0 : 1)$ is in $\mathbb P^1 \setminus \ell_1$ because it is parallel to $\ell_1$ in $\R^2$. Any other line in $\R^2$ would intersect $\ell_1$ at one point.

\subsection{~} % 3.d

The standard frame of reference for $\mathbb P^1(\R)$ is $(1 : 0), (0 : 1), (1 : 1)$. We want to find matrices $B$ and $C$ such that $B$ sends the standard frame of reference to points on $\ell_1$ and $C$ sends the standard frame of reference to points on $\ell_2$. Then $A = CB^{-1}$.

Let's say we want $B$ to map $(1 : 0)$ to $(-1 : 2)$ and $(0 : 1)$ to $(-1 : 1)$. Then using Proposition~6.40 in the lecture notes, we get \[
B = \begin{pmatrix}
-\lambda & -\mu \\
2\lambda & \mu
\end{pmatrix}
\] for some $\lambda, \mu \in \R$. We also need $B$ to map $(1 : 1)$ to $(-1 : 3)$, so we get
\begin{align*}
-\lambda - \mu &= -1 \\
2\lambda + \mu &= 3
\end{align*}
and we conclude that $\lambda = 2$ and $\mu = -1$. Therefore \[
B = \begin{pmatrix}
-2 & 1 \\
4 & -1
\end{pmatrix}.
\]

We do the same thing for $C$, but being careful to map points in the correct pairings. $B(1 : 0) = (-1 : 2)$ so we want $C(1 : 0) = (0 : 3)$ and $B(0 : 1) = {(-1 : 1)}$ so we want $C(0 : 1) = (3 : 3)$, so we get \[
C = \begin{pmatrix}
0 & 3\mu \\
3\lambda & 3\mu
\end{pmatrix}
\] for some $\lambda, \mu \in \R$. We also need $C(1 : 1) = (-1 : 3)$, so we get
\begin{align*}
3\mu &= -1 \\
3\lambda + 3\mu &= 3 \\
\end{align*}
and we conclude that $\lambda = \df43$ and $\mu = -\df13$. Therefore \[
C = \begin{pmatrix}
0 & -1 \\
4 & -1
\end{pmatrix}.
\]

Now we just need to find $A$.
\begin{align*}
\det B &= 2 - 4 \\
&= -2 \\
B^{-1} &= \f1{-2} \begin{pmatrix}
-1 & -1 \\
-4 & -2
\end{pmatrix} \\[0.5ex]
&= \begin{pmatrix}
\f12 & \f12 \\[0.25ex]
2 & 1
\end{pmatrix} \\[0.5ex]
A &= CB^{-1} \\
&= \begin{pmatrix}
0 & -1 \\
4 & -1
\end{pmatrix}
\begin{pmatrix}
\f12 & \f12 \\[0.25ex]
2 & 1
\end{pmatrix} \\[0.5ex]
&= \begin{pmatrix}
-2 & -1 \\
0 & 1
\end{pmatrix}.
\end{align*}

\subsection{~} % 3.e

We know that $F(-1, x) = (f(x), 3)$, we can use $A$ to find $f$.
\begin{align*}
\begin{pmatrix}
-2 & -1 \\
0 & 1
\end{pmatrix}
\begin{pmatrix}
-1 \\ x
\end{pmatrix}
&= \lambda \begin{pmatrix}
2 - x \\ x
\end{pmatrix}
= \begin{pmatrix}
f(x) \\ 3
\end{pmatrix} \\[0.5ex]
\lambda (2 - x) &= f(x) \\
\lambda x &= 3
\end{align*}
We find $\lambda = \df x3$ and so \[
f(x) = \f{6 - 3x}x.
\]
Indeed, this satisfies the three values we already know: $f(1) = 3$, $f(2) = 0$, and $f(3) = -1$.

% }}}

\end{document}
