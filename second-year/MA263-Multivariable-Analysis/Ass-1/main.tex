% vim: set foldmethod=marker foldlevel=0:

\documentclass[a4paper]{article}
\usepackage[UKenglish]{babel}

\usepackage{preamble}

\fancyhead[L]{MA263 Assignment 1}
\title{MA263 Multivariable Analysis, Assignment 1}
\colorlet{questionbodycolor}{cyan!50}

\begin{document}

\maketitle

\setlength{\parindent}{0em}
\setlength{\parskip}{1em}

% {{{ Q1
\question{1}

\begin{questionbody}
Show directly from the definition that matrix multiplication \begin{gather*}
m : \R^{m,n} \times \R^{n,p} \to \R^{m, p} \\
m(A, B) := AB
\end{gather*} is Fr\'echet differentiable.

(\textit{Suggestion}: Rather than compute all the partial derivatives, see if you can directly find the linear part of the `best linear approximation' to $m$.)
\end{questionbody}

We want to find $L$ such that \[
\llim{h \to (0, 0)} \f{\| m(x + h) - m(x) - L(h) \|}{\| h \|} = 0,
\] where $x, h \in \R^{m,n} \times \R^{n,p}$.

% }}}

% {{{ Q2
\newquestion{2}

\begin{questionbody}
Let $K \subset \R^n$ be compact. Let $\|\cdot\|$ be a norm on $\R^m$. For $f : K \to \R^m$, define \[
\|f\|_\infty := \sup_{x \in K} \|f(x)\|.
\]

\begin{enumerate}[(a)]
\item Show that this is finite and defines a norm on the vector space of continuous functions $f : K \to \R^m$.

\item Suppose that a sequence of continuous functions $f_i : K \to \R^m$ is such that \[
\forall \varepsilon > 0, \exists I \in \N \text{ such that } i, j \ge I \implies {\|f_i - f_j\|}_\infty < \varepsilon.
\]
Show that there exists a continuous function $f : K \to \R^m$ such that \[
\forall \varepsilon > 0, \exists I \in \N \text{ such that } i \ge I \implies {\|f_i - f\|}_\infty < \varepsilon.
\]

(\textit{Suggestion}: This is a multivariable extension of the uniform convergence material from Analysis 3---see if you can adapt one of the proofs from your notes.)
\end{enumerate}
\end{questionbody}

\subsection{~} % 2.a

% TODO: Explain this more
Since $K$ is compact, it is sequentially compact. Since $f$ is continuous, $\|f\|_\infty$ must be finite.

To show that $\|\cdot\|_\infty$ is a norm, we need to show three things. Firstly, since $\|\cdot\|$ is a norm on $\R^m$, it is non-negative. So $\sup_{x \in K} \|f(x)\| \ge 0$, and the only way to get equality is when $f$ is the constant 0 function.

Secondly, we require that for any $\lambda \in \R$, $\| \lambda f \|_\infty = |\lambda| \, \|f\|_\infty$. This is evident from the definition, again since $\|\cdot\|$ is a norm on $\R^m$: \[
\| \lambda f \|_\infty = \sup_{x \in K} \|\lambda f(x)\| = \sup_{x \in K} |\lambda| \, \|f(x)\| = |\lambda| \, \|f\|_\infty.
\]

Finally, we require the triangle inequality. Let $f$ and $g$ be functions $K \to \R^m$. Since $\|\cdot\|$ is a norm on $\R^m$,
\begin{align*}
\|f + g\|_\infty &= \sup_{x \in K} \|f(x) + g(x)\| \\
&\le \sup_{x \in K} \big( \|f(x)\| + \|g(x)\| \big) \\
&\le \sup_{x \in K} \|f(x)\| + \sup_{x \in K} \|g(x)\| \\
&= \|f\|_\infty + \|g\|_\infty.
\end{align*}
Therefore $\|\cdot\|_\infty$ is a norm.

\subsection{~} % 2.b

Let $x \in K$ and define $x_i = f_i(x)$. Then $(x_i)$ is a Cauchy sequence and since $\R^m$ is complete, $(x_i)$ converges to some point in $\R^m$. We can do this for any $x$, so there must exist a function $f$ such that $f_i(x)$ converges to $f(x)$ at least pointwise. We want to show it converges uniformly.

By the hypothesis, \[
\forall \varepsilon > 0, \exists I \in \N \text{ such that } i, j \ge I \implies {\|f_i(x) - f_j(x)\|}_\infty < \varepsilon.
\]

That means that every component of the function is bounded by $\varepsilon$. So for all $k \in \{1, \dotsc, n\}$, \[
|f_{k,i}(x) - f_{k,j}(x)| < \varepsilon,
\] where $f_{k,i}(x)$ denotes the $k$th component of $f_i(x)$.

Consider some $k \in \{1, \dotsc, n\}$. The previous inequality tells us that \[
f_{k,j}(x) - \varepsilon < f_{k,i}(x) < f_{k,j}(x) + \varepsilon.
\] Since this holds for all $j \ge I$, we can take limits as $j \to \infty$. We find \[
f_k(x) - \varepsilon < f_{k,i}(x) < f_k(x) + \varepsilon,
\] from which it follows that \[
|f_k(x) - f_{k,i}(x)| < \varepsilon.
\] which proves the result.

We can apply the previous argument to all $k$ and convert back to the $\infty$-norm to see that \[
\|f(x) - f_i(x)\|_\infty < \varepsilon,
\] which proves the result.

\hfill $\square$

% }}}

% {{{ Q3
\newquestion{3}

\begin{questionbody}
Let $A \subset \R^n$, $f : A \to \R^m$. Suppose $\|\cdot\|$ and $\|\cdot\|'$ are norms on $\R^n$ and $\R^m$ respectively.

We say that $f$ is Lipschitz if there is a constant $L > 0$ so that \[
\|f(x) - f(y)\|' \le L \|x - y\|.
\]

\begin{enumerate}[(a)]
\item Show that a Lipschitz function is continuous.

\item Let $U = \{x \in \R^n : \|x\|_\infty < 1\}$ and $f : U \to \R^m$ be differentiable. Suppose also that for some $L > 0$, $\|Df(x)\|_{\infty \to \infty} \le L$ for every $x \in U$. Show that $f$ is Lipschitz with the $\infty$-norms on $\R^n$ and $\R^m$.
\end{enumerate}
\end{questionbody}

\subsection{~} % 3.a

Suppose $f$ is Lipschitz. To show that $f$ is continuous, we want to show that for any $\varepsilon > 0$, we can find a $\delta > 0$ such that $\|x - y\| < \delta \implies \|f(x) - f(y)\|' < \varepsilon$.

We know that $\|f(x) - f(y)\|' \le L \|x - y\|$ and so if $\|x - y\| < \delta$, then $\|f(x) - f(y)\|' < L \delta$. Now we can just set $L \delta = \varepsilon$ and see that our $\delta$ must be $\df \varepsilon L$. Therefore $f$ is continuous.

\subsection{~} % 3.b

We know that for some $L > 0$, \[
\|Df(x)\|_{\infty \to \infty} = \sup_{x \in U \setminus \{0\}} \f{\|Df(x)\|_\infty}{\|x\|_\infty} \le L
\]

Therefore \[
\|f(x) - f(y)\|_\infty \le L \|x - y\|_\infty.
\]

\hfill $\square$

% }}}

% {{{ Q4
\newquestion{4}

\begin{questionbody}
Use the following strategy to show that all norms on $\R^n$ are equivalent:

\begin{enumerate}[(a)]
\item Explain why it suffices to prove that an arbitrary norm $\|\cdot\|$ on $\R^n$ is equivalent to $\|\cdot\|_\infty$.

\item Show that $\exists C > 0$ so that $\|x\| \le C \|x\|_\infty$.

(\textit{Hint}: Write $x = x_1 e_1 + \dotsb + x_n e_n$.)

\item Show that $\|x\|$ is a continuous function on $\R^n$ with respect to the $\infty$-norm.

(\textit{Suggestion}: Show that it is Lipschitz using the previous part.)

\item Show that $S = \{x \in \R^n : \|x\|_\infty = 1\}$ is compact. You may use any of the equivalent formulations of compactness.

\item Show that $\exists C' > 0$ such that $\|x\| \ge C'$ for $x \in S$.

\item Conclude that $\|\cdot\|$ and $\|\cdot\|_\infty$ are equivalent.
\end{enumerate}
\end{questionbody}

\subsection{~} % 4.a

Equivalence of norms is transitive. Suppose $\|\cdot\|$ and $\|\cdot\|'$ are both equivalent to $\|\cdot\|_\infty$. Then for all $x \in \R^n$, there exists $c_1, c_2, c_1', c_2' > 0$ such that \[
c_1 \|x\| \le \|x\|_\infty \le c_2 \|x\|
\] and  \[
c_1' \|x\|' \le \|x\|_\infty \le c_2' \|x\|'.
\]

Therefore \[
c_1 \|x\| \le \|x\|_\infty \le c_2' \|x\|'
\] and so \[
c_1 \|x\| \le c_2' \|x\|'.
\]
Likewise, \[
c_1' \|x\|' \le c_2 \|x\|.
\]

Therefore \[
\f{c_1}{c_2'} \|x\| \le \|x\|' \le \f{c_2}{c_1'} \|x\|,
\] and so $\|\cdot\|$ and $\|\cdot\|'$ are equivalent.

\subsection{~} % 4.b

$\|x\|_\infty = \max\{x_1, \dotsc, x_n\}$, so without loss of generality, assume $\|x\|_\infty = x_1$. Then we just choose $C \ge \df{\|x\|}{x_1}$.

\subsection{~} % 4.c

Let $x, y \in \R^n$. We want to show that $\|\cdot\|$ is Lipschitz with respect to the $\infty$-norm, so we want to show that \[
\big\| \|x\| - \|y\| \big\|_\infty \le L \|x - y\|_\infty
\] for some $L > 0$.

We know there exists $C_x, C_y > 0$ such that \[
\|x\| \le C_x \|x\|_\infty, \qquad \|y\| \le C_y \|y\|_\infty.
\]
% TODO: Finish

Since $\|\cdot\|$ is Lipschitz, it is continuous, as per \textbf{Q3(a)}.

\subsection{~} % 4.d

Consider a sequence $(x_i)$ in $S$ which converges to a point $x$.
% Then $\forall \varepsilon > 0$, $\exists N \in \N$ such that $\forall n > N$, $\|x_n - x\|_\infty < \varepsilon$.
Since all $\|x_i\|_\infty = 1$, $\|x\|_\infty = 1$. The limit of the sequence is in $S$, so $S$ is closed.
Clearly $S$ is bounded since for all $x \in S$, $\|x\|_\infty \le 1$. Since $S$ is closed and bounded, it is compact.

\subsection{~} % 4.e

We know $\|x\| = 0$ if and only if $x = 0$ but we require $x \in S$ and we know $0 \ne S$. Therefore we can set $C = \inf_{x \in S} \|x\|$ and we know that $C > 0$.

\subsection{~} % 4.f

Since $x \in S$ in part \textbf{Q4(e)}, we can multiply by $\|x\|_\infty = 1$ and swap sides to get \[
C' \|x\|_\infty \le \|x\|.
\] We combine this with the result of \textbf{Q4(b)} to get \[
C' \|x\|_\infty \le \|x\| \le C \|x\|_\infty.
\] Therefore $\|\cdot\|$ is equivalent to $\|\cdot\|_\infty$, so all norms on $\R^n$ are equivalent.

\hfill $\square$

% }}}

\end{document}
