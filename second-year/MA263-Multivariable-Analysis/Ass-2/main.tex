% vim: set foldmethod=marker foldlevel=0:

\documentclass[a4paper]{article}
\usepackage[UKenglish]{babel}

% \usepackage[hidelinks]{hyperref}

\usepackage{preamble}

% \usepackage{graphicx}
% \graphicspath{ {./imgs/} }

% \renewcommand{\thesubsection}{Q\arabic{section}~(\roman{subsection})}

\fancyhead[L]{MA263 Assignment 2}
\title{MA263 Multivariable Analysis, Assignment 2}
\colorlet{questionbodycolor}{cyan!50}

\begin{document}

\maketitle

\setlength{\parindent}{0em}
\setlength{\parskip}{1em}

% {{{ Q1
\question{1}

\begin{questionbody}
Compute the second order Taylor polynomial of $f(x, y) = \arctan(x + y)$ at $(0, 0)$.
\end{questionbody}

We want
\begin{align*}
T_{\ul 0}^2 f(\ul x) &= f(\ul 0) + \nabla f(\ul 0) \cdot (\ul x - \ul 0) + {(\ul x - \ul 0)}^T \cdot \nabla^2 f(\ul 0) \cdot (\ul x - \ul 0) \\
&= f(\ul 0) + \nabla f(\ul 0) \cdot \ul x + {\ul x}^T \cdot \nabla^2 f(\ul 0) \cdot \ul x,
\end{align*}
where $\ul x = (x, y)$ and $\ul 0 = (0, 0)$.

We first compute the gradient and Hessian.
\begin{align*}
\nabla f(x, y) &= \begin{pmatrix}
\partial_x f \\
\partial_y f
\end{pmatrix} \\
&= \begin{pmatrix}
\f1{{(x + y)}^2 + 1} \\
\f1{{(x + y)}^2 + 1}
\end{pmatrix} \\[1.5ex]
%
\nabla^2 f(x, y) &= \begin{pmatrix}
\partial_x^2 f & \partial_{xy} f \\
\partial_{yx} f & \partial_y^2 f
\end{pmatrix} \\
&= \begin{pmatrix}
- \f{2 (x + y)}{{( {(x + y)}^2 + 1 )}^2} & - \f{2 (x + y)}{{( {(x + y)}^2 + 1 )}^2} \\
- \f{2 (x + y)}{{( {(x + y)}^2 + 1 )}^2} & - \f{2 (x + y)}{{( {(x + y)}^2 + 1 )}^2}
\end{pmatrix}
\end{align*}

Then we evaluate these at $\ul 0$.
\begin{align*}
\nabla f(0, 0) &= \begin{pmatrix}
\f1{0^2 + 1} \\
\f1{0^2 + 1}
\end{pmatrix} \\
&= \begin{pmatrix}
1 \\
1
\end{pmatrix} \\[1.5ex]
%
\nabla^2 f(x, y) &= \begin{pmatrix}
- \f{2 (0)}{{( 0^2 + 1 )}^2} & - \f{2 (0)}{{( 0^2 + 1 )}^2} \\
- \f{2 (0)}{{( 0^2 + 1 )}^2} & - \f{2 (0)}{{( 0^2 + 1 )}^2}
\end{pmatrix} \\
&= \begin{pmatrix}
0 & 0 \\
0 & 0
\end{pmatrix}
\end{align*}

And so,
\begin{align*}
T_{\ul 0}^2 f(\ul x) &= f(\ul 0) + \nabla f(\ul 0) \cdot \ul x + {\ul x}^T \cdot \nabla^2 f(\ul 0) \cdot \ul x \\
&= \arctan(0) + \begin{pmatrix} 1 \\ 1 \end{pmatrix} \cdot \begin{pmatrix} x \\ y \end{pmatrix} + \begin{pmatrix} x & y \end{pmatrix} \cdot \begin{pmatrix} 0 & 0 \\ 0 & 0 \end{pmatrix} \cdot \begin{pmatrix} x \\ y \end{pmatrix} \\
&= 0 + x + y + 0 \\
&= x + y.
\end{align*}

% }}}

% {{{ Q2
\newquestion{2}

\begin{questionbody}
Prove the following corollary.

Let $U \subset \R^2$ and $f \in C^2(U; \R)$ have a critical point at $a \in U$. If the determinant of the Hessian is positive then $f$ is a strict local extremum, and if it is negative then $f$ is a saddle point.

In the former case, it is a maximum if and only if the diagonal entries of $\nabla^2 f(a)$ are negative and a minimum if and only if they are positive.
\end{questionbody}

% Do some linear algebra and appeal to the second derivative test.
% Recall the trace, which is the sum of the diagonal entries, and also the sum of the eigenvalues.

Answer

% }}}

% {{{ Q3
\newquestion{3}

\begin{questionbody}
Using the optimisation methods of calculus, find the global maximum and minimum of \[
f(x, y) = 2x^2 + 6xy - 26x + 5y^2 - 42y + 93
\] over ${[0, 4]}^2$.
\end{questionbody}

Answer

% }}}

% {{{ Q4
\newquestion{4}

\begin{questionbody}
Consider $f L \R^2 \to \R^2$ defined by \[
f(x, y) = (x^2 - y^2, y - x).
\] Find all points near which $f$ is locally invertible. Explain why your list is complete.
\end{questionbody}

Answer

% }}}

\end{document}
