% vim: set foldmethod=marker foldlevel=0:

\documentclass[a4paper]{article}
\usepackage[UKenglish]{babel}

\usepackage{preamble}

\renewcommand{\thesubsection}{Q\arabic{section}~(\roman{subsection})}

\fancyhead[L]{MA260 Assignment 2}
\title{MA260 Norms Metrics and Topologies, Assignment 2}
\colorlet{questionbodycolor}{orange!50!violet}

\begin{document}

\maketitle

\setlength{\parindent}{0em}
\setlength{\parskip}{1em}

% {{{ Q1
\question{1}

\begin{questionbody}
Let $(X, d)$ be a metric space and let $x \ne y$ be two elements of $X$. Show that $\exists \varepsilon_1, \varepsilon_2 > 0$ such that $\bb B(y, \varepsilon_1) \cap \bb B(y, \varepsilon_2) = \emptyset$.
\end{questionbody}

Answer

% }}}

% {{{ Q2
\newquestion{2}

\begin{questionbody}
Let $(X, d)$ be a metric space and let $Y \subset X$. Show that $U$ is open in $Y$ if and only if $U = Y \cap V$, where $V$ is open in $X$.
\end{questionbody}

Answer

% }}}

% {{{ Q3
\newquestion{3}

\begin{questionbody}
In this exercise, we will consider $\R^2$ with two different metrics: the standard Euclidean metric $d$ and the \textit{sunflower metric} $d_\mathrm{sf}$ defined by \[
d_\mathrm{sf}(x, y) = \begin{cases}
\|x - y\| & \text{if $x$ and $y$ lie on the same line through the origin}, \\
\|x\| + \|y\| & \text{otherwise}.
\end{cases}
\]
%
\begin{enumerate}[(i)]
\item Let ${(x_n)}_{n=1}^\infty$ be a sequence in $\R^2$. Show that if $(x_n)$ converges to $x \in \R^2$ with respect to the sunflower metric then $(x_n)$ converges to $x$ with respect to the standard metric.

\item By giving an example, show that it is possible for a sequence $(x_n)$ to converge to $x \in \R^2$ with respect to the standard metric but not to converge to $x$ with respect to the sunflower metric.

\item Show that any sequence in $\R^2$ with the property described in part~\textbf{(ii)} does not converge to \underline{any} limit with respect to the sunflower metric.
\end{enumerate}
\end{questionbody}

\subsection{~} % 3.i

Answer

\subsection{~} % 3.ii

Answer

\subsection{~} % 3.iii

Answer

% }}}

% {{{ Q4
\newquestion{4}

\begin{questionbody}
Let $\cal T$ be a topology on $\R$. Suppose that for every pair of real numbers $a$ and $b$ with $a < b$, we have $[a, b] \in \cal T$. Show that $\cal T$ must be the discrete topology.
\end{questionbody}

Answer

% }}}

\end{document}
