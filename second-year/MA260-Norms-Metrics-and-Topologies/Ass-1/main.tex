% vim: set foldmethod=marker foldlevel=0:

\documentclass[a4paper]{article}
\usepackage[UKenglish]{babel}

\usepackage{preamble}

\renewcommand{\thesubsection}{Q\arabic{section}~(\roman{subsection})}

\fancyhead[L]{MA260 Assignment 1}
\title{MA260 Norms Metrics and Topologies, Assignment 1}
\colorlet{questionbodycolor}{orange!50!violet}

\begin{document}

\maketitle

\setlength{\parindent}{0em}
\setlength{\parskip}{1em}

% {{{ Q1
\question{1}

\begin{questionbody}
Give a reason why each of the following is \textbf{not} a norm on $\R^3$:
\begin{enumerate}[(i)]
\item $\|(x, y, z)\| = x + y + z$

\item $\|(x, y, z)\| = {\l( |x|^2 + |y|^2 + |z|^2 \r)}^{1/3}$

\item $\|(x, y, z)\| = {\l( |x|^4 - |y|^4 + 2 |z|^4 \r)}^{1/4}$
\end{enumerate}
\end{questionbody}

\subsection{~} % 1.i

Let $v = (1, -2, 1)$. Then $v \ne \ul 0$ but $\|v\| = 1 - 2 + 1 = 0$, so this norm does not satisfy non-degeneracy.

\subsection{~} % 1.ii

Let $v = (x, y, z) \in \R^3$ and $\lambda \in \R$. Then
\begin{align*}
\| \lambda v \| &= {\l( |\lambda x|^2 + |\lambda y|^2 + |\lambda z|^2 \r)}^{1/3} \\
&= {\l( |\lambda|^2 \l( |x|^2 + |y|^2 + |z|^2 \r) \r)}^{1/3} \\
&= |\lambda|^{2/3} \|v\|.
\end{align*}
It this were a norm, we would expect to get $\|\lambda v\| = |\lambda| \|v\|$, but we don't, so this is not a norm.

\subsection{~} % 1.iii

If this were a norm, it should satisfy the triangle inequality, so we would have $\|v + w\| \le \|v\| + \|w\|$. Take $v = (1, 2, 3)$ and $w = (3, 2, 1)$. Then
\begin{align*}
\|v\| &= {\l( 1 - 16 + 162 \r)}^{1/4} \\
&= 147^{1/4} \\
&\approx 3.482 \\
\|w\| &= {\l( 81 - 16 + 2 \r)}^{1/4} \\
&= 67^{1/4} \\
&\approx 2.861 \\
\|v\| + \|w\| &\approx 6.343 \\
\|v + w\| &= \|(4, 4, 4)\| \\
&= {\l( 256 - 256 + 512 \r)}^{1/4} \\
&= 512^{1/4} \\
&\approx 4.757
\end{align*}
This counterexample shows that this `norm' doesn't satisfy the triangle inequality, so is not a norm.

% }}}

% {{{ Q2
\newquestion{2}

\begin{questionbody}
Consider the space $\ell^1$ with norm $\| \cdot \|_1$ and define $g : \ell^1 \to \R$ by \[
g \big( {(x_j)}_{j=1}^\infty \big) = \sum_{j=1}^\infty {(-1)}^j x_j.
\] Show that $g : \ell^1 \to \R$ is well-defined and continuous.
\end{questionbody}

The space $\ell^1$ contains only sequences which are `1-summable', meaning $\sum_{j=1}^\infty x_j$ converges to a finite value. Therefore $\sum_{j=1}^\infty {(-1)}^j x_j$ also converges to a finite value. % TODO: Why?
That means $g \big( (x_j) \big)$ is in $\ell^1$ and is therefore well-defined.

For $g$ to be continuous, we want to show that $\forall \varepsilon > 0$, $\exists \delta > 0$ such that \[
g \big( \bb B_\delta((x_j)) \big) \subset \bb B_\varepsilon \big( g((x_j)) \big).
\]

Every sequence in $\bb B_\delta((x_j))$ is the sequence $(x_j)$ with a perturbation of $\delta$. So if $(x_j) = (x_1, x_2, \dotsc)$, then an element of $\bb B_\delta((x_j))$, call it $(x_j) + \delta$, would look something like $(x_1, \dotsc, x_i + \delta, \dotsc)$ or $(x_1, \dotsc, x_i + \delta/3, \dotsc, x_k + 2\delta/3, \dotsc)$.

Since $\ell^1$ uses the 1-norm, this perturbation of $\delta$ will be carried through the norm, so \[
\|(x_j) + \delta\| = \|(x_j)\| + |\delta|.
\]
Then this $|\delta|$ will just get included in the sum in $g$, so \[
g \big( (x_j) + \delta \big) = g \big( (x_j) \big) + |\delta|.
\] Therefore \[
g \big( \bb B_\delta((x_j)) \big) = \bb B_\delta \big( g((x_j)) \big)
\]
Then we just choose $\delta \le \varepsilon$ and see that $g$ is continuous.

% }}}

\end{document}
