% vim: set foldmethod=marker foldlevel=0:

\documentclass[a4paper]{article}
\usepackage[UKenglish]{babel}

\usepackage{preamble}

\fancyhead[L]{MA260 Assignment 3}
\title{MA260 Norms Metrics and Topologies, Assignment 3}
\colorlet{questionbodycolor}{orange!50!violet}

\begin{document}

\maketitle

\setlength{\parindent}{0em}
\setlength{\parskip}{1em}

% {{{ Q1
\question{1}

\begin{questionbody}
Suppose that $f : (X, \cal T_X) \to (Y, \cal T_Y)$ is continuous. If $x$ is a limit point of the subset $A \subset X$, is it necessarily true that $f(x)$ is a limit point of $f(A)$?
\end{questionbody}

Consider for a counterexample $f(x) = \sin x$ on the open interval $(0, 2) \subset \R$. Then clearly $f : \R \to \R$ is continuous with the usual topologies, and $f((0, 2)) = (0, 1]$. $2$ is a limit point in $(0, 2)$ but $f(2) \approx 0.909$, which is not a limit point of $(0, 1]$. For example, $\bb B_{0.01}(f(2))$ as an open neighbourhood does not intersect $\R \setminus (0, 1]$. % chktex 9

% }}}

% {{{ Q2
\question{2}

\begin{questionbody}
Let $U \subset (X, \cal T)$. If $U$ is open, is it true that $U = {\l( \ol U \r)}^\circ$? Justify your answer.
\end{questionbody}

Consider for a counterexample $U = (0, 1) \cup (1, 2)$. Then $1 \notin U$, but ${\l( \ol U \r)}^\circ = (0, 2)$ and $1 \in (0, 2)$.

% }}}

% {{{ Q3
\newquestion{3}

\begin{questionbody}
Find the boundary and interior of each of the following subsets of $\R^2$ equipped with the standard topology:
\begin{enumerate}[1.]
\item $A = \{x \times y : y = 0\}$,

\item $B = \{x \times y : x > y \text{ and } y \ne 0\}$,

\item $C = A \cup B$.
\end{enumerate}
\end{questionbody}

% === If $\times$ means real multiplication

% $A = \{0\}$ so simply $\ol A = \{0\}$ since $[0, 0]$ is a closed set. Then since $\{0\}$ is closed, the only open subset of $A$ is $\emptyset$, so $A^\circ = \emptyset$. Therefore $\partial A = \ol A \setminus A^\circ = \{0\}$
%
% $B = \R$ since any $z \in \R$ is a product of some $x, y \in \R$ with $x > y$. So $\ol B = \R$ and $B^\circ = \R$. Therefore $\partial B = \ol B \setminus B^\circ = \emptyset$.
% % $B = \R \setminus \{0\}$
% % $B^\circ = \R \setminus \{0\}$
% % $\partial B = \ol B \setminus B^\circ = \{0\}$
%
% $C = \R$ so by the same logic we used for $B$, $C^\circ = \R$ and $\partial C = \emptyset$.

% === If $\times$ means tuple

$A = \{(x, 0) : x \in \R\}$. Since this a straight line in $\R^2$, $\partial A = A$ and $A^\circ = \emptyset$.

% TODO: Justify and nice-ify
$B = \{(x, y) : x, y \in \R, x > y, y \ne 0\} = \{(x, y) : x, y \in \R, x > y\} \setminus A$.

$\partial B = \{(x, x) : x \in \R\} \cup \{(x, 0) : x \in \R, x > 0\}$.

$B^\circ = B$.

$C = \{(x, y) : x, y \in \R, x > y\} \cup A$.

$\partial C = \{(x, x) : x \in \R\} \cup \{(x, 0) : x \in \R, x < 0\}$.

$C^\circ = \{(x, y) : x, y \in \R, x > y\}$.

% }}}

% {{{ Q4
\newquestion{4}

\begin{questionbody}
Show that $X$ is Hausdorff if and only if the \textit{diagonal} $\Delta = \{x \times x : x \in X\}$ is closed in $X \times X$.
\end{questionbody}

Answer
% }}}

% {{{ Q5
\newquestion{5}

\begin{questionbody}
Let $A \subset X$ and let $f : A \to Y$ be continuous. Suppose $Y$ is Hausdorff. Show that there is at most one continuous function $g : \ol A \to Y$ such that $g(x) = f(x)$ for all $x \in A$.
\end{questionbody}

% See problem sheet 5, exercise 5

Answer

% }}}

\end{document} % chktex 17
