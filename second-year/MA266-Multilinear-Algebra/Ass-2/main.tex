% vim: set foldmethod=marker foldlevel=0:
\documentclass[a4paper]{article}
\usepackage[UKenglish]{babel}

\usepackage{preamble}

\renewcommand{\thesubsection}{Q\arabic{section}~(\roman{subsection})}

\fancyhead[L]{MA266 Assignment 2}
\title{MA266 Multilinear Algebra, Assignment 2}
\colorlet{questionbodycolor}{magenta!50}

\begin{document}

\maketitle

\setlength{\parindent}{0em}
\setlength{\parskip}{1em}

% {{{ Q11
\question{11}

\begin{questionbody}
Let $V$ be a finite dimensional $\bb F$-vector space and let $B$ be a basis for $V$. Let \[
M = {(V \otimes V)}_- := \bb F \angb{\{ v \otimes w - w \otimes v : v, w \in V \}}.
\] Show that the set \[
\{ b \otimes c + M : \{b, c\} \subset B \text{ of size at most 2} \} = \{ b \otimes c + M : b, c \in B \}
\] is a basis for the symmetric square $S^2(V) = (V \otimes V) / M$.
\end{questionbody}

Set $X := \{b_i \otimes b_j + M : 1 \le i \le j \le n\}$. We need to prove that $X$ is linearly independent and that $X$ spans $S^2(V)$.

First, we prove linear independence. Assume $\exists \lambda_{i, j} \in \bb F$ such that \[
\sum_{1 \le i < j \le n} \lambda_{i, j} (b_i \otimes b_j + M) = 0_{S^2(V)}.
\] By the distributive laws for tensor products, we have \[
\sum_{1 \le i < j \le n} \lambda_{i, j} (b_i \otimes b_j) + M = 0_{S^2(V)}.
\] Equivalently, \[
\sum_{1 \le i < j \le n} \lambda_{i, j} (b_i \otimes b_j) \in M.
\]

% TODO

Secondly, we prove that $X$ spans $S^2(V)$. So let $w \in S^2(V) = (V \otimes V) / M$. Then since $\{b_i \otimes b_j : 1 \le i, j \le n\}$ is a basis for $V \otimes V$ and $w \in V \otimes V$, $\exists \lambda_{i, j} \in \bb F$ such that \[
w = \sum_{1 \le i, j \le n} \lambda_{i, j} (b_i \otimes b_j).
\]
% Thus, $$w = \sum_{i < j} \lambda_{i, j} (b_i \otimes b_j) + \sum_{1 \le i \le n} \lambda_{i, i} (b_i \otimes b_i) + \sum_{i < j} \lambda_{i, j} (b_j \otimes b_i).$$
% Clearly the middle term on the RHS is in $D$. Consider the rightmost term. By [[coset equality for pure tensor cosets]], we have $b_j \otimes b_i + D = -b_i \otimes b_j + D$. Thus, by definition of addition and scalar multiplication in a [[quotient vector space]], we have $$w + D = \sum_{1 \le i < j \le n} (\lambda_{i, j} \mu_{i, j}) (b_i \otimes b_j) + D.$$
Therefore $X$ spans $V$.

\hfill $\square$


% }}}

% {{{ Q12
\newquestion{12}

\begin{questionbody}
Let $V$ be an $\bb F$-vector space. Define \[
{(V \otimes V)}_+ := \bb F \angb{\{ v \otimes w + w \otimes v : v, w \in V \}}.
\]
\begin{enumerate}[(i)]
\item Prove that ${(V \otimes V)}_+ \subset \op{Diag}(V \otimes V)$.

\item Suppose that $1_{\bb F} \ne -1_{\bb F}$. Prove that ${(V \otimes V)}_+ = \op{Diag}(V \otimes V)$.

\item Let $V = \bb F_2^2$ be the space of 2-dimensional column vectors over the Galois field $\bb F_2 = \{0, 1\}$. Show that ${(V \otimes V)}_+ \ne \op{Diag}(V \otimes V)$.
\end{enumerate}
\end{questionbody}

\subsection{~} % 12.i

Let $D := \op{Diag}(V \otimes V)$.

We want to show that every basis element $v \otimes w + w \otimes v$ is in $D$. Every other element of ${(V \otimes V)}_+$ will follow as linear combinations.

Clearly $(v + w) \otimes (v + w)$ is an element of $D$ and we can expand it with the distributive laws to see that \[
v \otimes v + v \otimes w + w \otimes v + w \otimes w
\] is an element of $D$. Since the first and last terms of this expression are clearly elements of $D$, the part in the middle, $v \otimes w + w \otimes v$ must also be an element of $D$ by closure.

\hfill $\square$

\subsection{~} % 12.ii

Now we want to show that for each basis element $v \otimes w + w \otimes v$, there exists a $u \in V$ such that $v \otimes w + w \otimes v = u \otimes u$.

If $1_{\bb F} \ne -1_{\bb F}$ then $1_{\bb F} + -1_{\bb F} = 0_{\bb F}$. So then \[
% 0_D = v \otimes (1_{\bb F} + -1_{\bb F}) = v \otimes
% 0_D = 1 (v \otimes v) + -1 (v \otimes v) =
2(v \otimes v) = v \otimes v + v \otimes v
\]

\subsection{~} % 12.iii

Answer

% }}}

\end{document}
