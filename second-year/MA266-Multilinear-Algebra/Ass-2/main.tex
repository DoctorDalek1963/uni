% vim: set foldmethod=marker foldlevel=0:
\documentclass[a4paper]{article}
\usepackage[UKenglish]{babel}

\usepackage{preamble}

\renewcommand{\thesubsection}{Q\arabic{section}~(\roman{subsection})}

\fancyhead[L]{MA266 Assignment 2}
\title{MA266 Multilinear Algebra, Assignment 2}
\colorlet{questionbodycolor}{magenta!50}

\begin{document}

\maketitle

\setlength{\parindent}{0em}
\setlength{\parskip}{1em}

% {{{ Q11
\question{11}

\begin{questionbody}
Let $V$ be a finite dimensional $\bb F$-vector space and let $B$ be a basis for $V$. Let \[
M = {(V \otimes V)}_- := \bb F \angb{\{ v \otimes w - w \otimes v : v, w \in V \}}.
\] Show that the set \[
\{ b \otimes c + M : \{b, c\} \subset B \text{ of size at most 2} \} = \{ b \otimes c + M : b, c \in B \}
\] is a basis for the symmetric square $S^2(V) = (V \otimes V) / M$.
\end{questionbody}

Set $X := \{b_i \otimes b_j + M : 1 \le i \le j \le n\}$. We need to prove that $X$ is linearly independent and that $X$ spans $S^2(V)$.

First, we prove linear independence. Assume $\exists \lambda_{i, j} \in \bb F$ such that \[
\sum_{1 \le i < j \le n} \lambda_{i, j} (b_i \otimes b_j + M) = 0_{S^2(V)}.
\] By the distributive laws for tensor products, we have \[
\sum_{1 \le i < j \le n} \lambda_{i, j} (b_i \otimes b_j) + M = 0_{S^2(V)}.
\] Equivalently, \[
\sum_{1 \le i < j \le n} \lambda_{i, j} (b_i \otimes b_j) \in M.
\]

Thus, by the definition of $M$, $\exists \mu_{i, j}, \alpha_i \in \bb F$ such that \[
\sum_{i \le i < j \le n} \lambda_{i, j} (b_i \otimes b_j) = \sum_{i=1}^n \alpha_i (b_i \otimes b_i) + \sum_{i \ne j} \mu_{i, j} (b_i \otimes b_j - b_j \otimes b_i).
\]

Hence, \[
\sum_{1 \le i < j \le n} \lambda_{i, j} \mu_{i, j} (b_i \otimes b_j) + \sum_{1 \le i < j \le n} \mu_{i, j} (b_j \otimes b_i) - \sum_{1 = 1}^n \alpha_i (b_i \otimes b_i) = 0_{V \otimes V}.
\]

By linear independence of the basis $\{b_i \otimes b_j : 1 \le i, j \le n\}$ in $V \otimes V$, it follows that $\alpha_i = 0_{\bb F}$ for all $i$ and $\mu_{i, j} = 0_{\bb F}$ for all $i > j$, and so $\lambda_{i, j} = 0_{\bb F}$ for all $1 \le i < j \le n$. Thus, $X$ is linearly independent.

% TODO: This is definitely wrong but I barely understand the question, and I definitely don't understand the proof given in the lecture notes for a similar problem

Secondly, we prove that $X$ spans $S^2(V)$. So let $w + M \in S^2(V) = (V \otimes V) / M$. Then since $\{b_i \otimes b_j : 1 \le i, j \le n\}$ is a basis for $V \otimes V$ and $w \in V \otimes V$, $\exists \lambda_{i, j} \in \bb F$ such that \[
w = \sum_{1 \le i, j \le n} \lambda_{i, j} (b_i \otimes b_j).
\] Thus, \[
w = \sum_{i < j} \big( \lambda_{i, j} (b_i \otimes b_j) + \lambda_{i, j} (b_j \otimes b_i) \big) + \sum_{1 \le i \le n} \lambda_{i, i} (b_i \otimes b_i).
\]
Clearly the first term on the RHS is in $M$ and the second term is in $\op{Diag}(V \otimes V)$. From lectures, we know $b_j \otimes b_i + D = -b_i \otimes b_j + D$. Thus, by definition of addition and scalar multiplication in a quotient vector space, we have \[
w + D = \sum_{1 \le i < j \le n} (\lambda_{i, j} \mu_{i, j}) (b_i \otimes b_j) + D.
\] Therefore $X$ spans $V$.

\hfill $\square$

% }}}

% {{{ Q12
\newquestion{12}

\begin{questionbody}
Let $V$ be an $\bb F$-vector space. Define \[
{(V \otimes V)}_+ := \bb F \angb{\{ v \otimes w + w \otimes v : v, w \in V \}}.
\]
\begin{enumerate}[(i)]
\item Prove that ${(V \otimes V)}_+ \subset \op{Diag}(V \otimes V)$.

\item Suppose that $1_{\bb F} \ne -1_{\bb F}$. Prove that ${(V \otimes V)}_+ = \op{Diag}(V \otimes V)$.

\item Let $V = \bb F_2^2$ be the space of 2-dimensional column vectors over the Galois field $\bb F_2 = \{0, 1\}$. Show that ${(V \otimes V)}_+ \ne \op{Diag}(V \otimes V)$.
\end{enumerate}
\end{questionbody}

\subsection{~} % 12.i

Let $D := \op{Diag}(V \otimes V)$.

We want to show that every element $v \otimes w + w \otimes v$ is in $D$. Every other element of ${(V \otimes V)}_+$ will follow as linear combinations.

Clearly $(v + w) \otimes (v + w)$ is an element of $D$ and we can expand it with the distributive laws to see that \[
v \otimes v + v \otimes w + w \otimes v + w \otimes w
\] is an element of $D$. Since the first and last terms of this expression are clearly elements of $D$, the part in the middle, $v \otimes w + w \otimes v$ must also be an element of $D$ by closure.

\hfill $\square$

\subsection{~} % 12.ii

Now we want to show that $D \subset {(V \otimes V)}_+$, so we want to show that every element $v \otimes v$ is in ${(V \otimes V)}_+$. Every other element of $D$ will follow as linear combinations.

We can choose $v = w$ and see that $v \otimes v + v \otimes v$ is an element of ${(V \otimes V)}_+$. Since we're taking the span, we can just multiply this by $\f12$ and see that $v \otimes v$ is an element of ${(V \otimes V)}_+$ as required.

Note that if $1_{\bb F} = -1_{\bb F}$ then $v \otimes v + v \otimes v = 0_{V \otimes V}$ so this trick wouldn't work.

Since $(V \otimes V) \subset D$ and $D \subset {(V \otimes V)}_+$, we have ${(V \otimes V)}_+ = D$ as required.

\hfill $\square$

\subsection{~} % 12.iii

In this $V$, $1_{\bb F_2^2} = \begin{pmatrix} 1 \\ 1 \end{pmatrix}$ and $-1_{\bb F_2^2} = \begin{pmatrix} 1 \\ 1 \end{pmatrix}$. Thus $1_{\bb F_2^2} = -1_{\bb F_2^2}$ and so we can't do our trick from \textbf{Q12~(ii)}. The logic in \textbf{Q12~(i)} still works, so we still have ${(V \otimes V)}_+ \subset D$.

The only way to get any element of the form $\lambda v \otimes v$ from ${(V \otimes V)}_+$ is to add an element to itself. Since $1_{\bb F_2^2} = -1_{\bb F_2^2}$, adding any element to itself will give zero, and so we cannot make those required elements from ${(V \otimes V)}_+$. Therefore $D \not\subset {(V \otimes V)}_+$ and so ${(V \otimes V)}_+ \ne D$.

\hfill $\square$

% }}}

\end{document}
