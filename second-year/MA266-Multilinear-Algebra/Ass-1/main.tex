% vim: set foldmethod=marker foldlevel=0:

\documentclass[a4paper]{article}
\usepackage[UKenglish]{babel}

% NOTE: hyperref has to come before preamble
% \usepackage[hidelinks]{hyperref}

\usepackage{preamble}

% \usepackage{graphicx}
% \graphicspath{ {./imgs/} }

\renewcommand{\thesubsection}{Q\arabic{section}~(\roman{subsection})}

\fancyhead[L]{MA266 Assignment 1}
\title{MA266 Multilinear Algebra, Assignment 1}
\colorlet{questionbodycolor}{magenta!50}

\begin{document}

\maketitle

\setlength{\parindent}{0em}
\setlength{\parskip}{1em}

% {{{ Q10
\question{10}

\begin{questionbody}
Recall that a matrix $M \in M_n(\bb F)$ is invertible if there exists $N \in M_n(\bb F)$ such that $MN = NM = I_n$, where $I_n$ is the $n \times n$ identity matrix.
\begin{enumerate}[(i)]
\item Show that matrix multiplication is associative. That is, for $L, M, N \in M_n(\bb F)$, show that $(LM)N = L(MN)$. Deduce that if we set \[
\mathrm{GL}_n(\bb F) := {\{ M : M \in M_n(\bb F) \text{ is invertible} \}},
\] then $(\mathrm{GL}_n(\bb F), \circ)$ is a group where ${M \circ N := MN}$.

\item Suppose that there exists $L, R \in M_n(\bb F)$ with $LM = MR = I_n$. Prove that $L = R$.

\item Deduce from~\textbf{(ii)} that if $M \in M_n(\bb F)$, then $M$ is invertible if and only if $M^k$ is invertible for all $k \in \N$.
\end{enumerate}
\end{questionbody}

\subsection{~} % 1.i

% TODO: Show associativity

The identity element of $\mathrm{GL}_n(\bb F)$ is $I_n$, whose inverse is $I_n$. Every element in $\mathrm{GL}_n(\bb F)$ has an inverse by definition. The product of two invertible matrices $M$ and $N$ is $MN$ and it has inverse $N^{-1} M^{-1}$, so $\mathrm{GL}_n(\bb F)$ is closed. We've already proven that matrix multiplication is associative, so $(\mathrm{GL}_n(\bb F), \circ)$ is a group.

\subsection{~} % 1.ii

\begin{align*}
LM &= I_n \\
LMR &= I_n R \\
&= R \\
MR &= I_n \\
LMR &= L I_n \\
&= L \\
\therefore\ LMR &= R = L
\end{align*}

\subsection{~} % 1.iii

Suppose $M$ is invertible. Then for all $k \in \N$, $M^k$ is invertible and has inverse $M^{-k}$ by induction on $k$.
% \begin{align*}
% M^k M^{-k} &= \underbrace{M \dotsm M}_{k \text{ times}} \, \underbrace{M^{-1} \dotsm M^{-1}}_{k \text{ times}} \\
% &= \underbrace{M \dotsm M}_{k-1 \text{ times}} \, (M M^{-1}) \, \underbrace{M^{-1} \dotsm M^{-1}}_{k-1 \text{ times}} \\
% &= \underbrace{M \dotsm M}_{k-1 \text{ times}} \, \underbrace{M^{-1} \dotsm M^{-1}}_{k-1 \text{ times}} \\
% &\,\ \vdots \\
% &= I_n.
% \end{align*}

For the converse, suppose $M^k$ is invertible for all $k \in \N$. Then we just choose $k=1$ and get that $M M^{-1} = I_n$, so $M$ is invertible.

% }}}

% {{{ Q11
\newquestion{11}

\begin{questionbody}
Let $V$ be an $\bb F$-vector space and let $X$ be a subset of $V$.
\begin{enumerate}[(i)]
\item Prove that $\bb F \langle X \rangle$ is a subspace of $V$.

\item Prove that $\bb F \langle X \rangle = \bigcap_{X \subseteq W \le V} W$. That is, $\bb F \langle X \rangle$ in the intersection of all subspaces of $V$ containing $X$.
\end{enumerate}
\end{questionbody}

\subsection{~} % 2.i

Answer

\subsection{~} % 2.ii

Answer

% }}}

% {{{ Q12
\newquestion{12}

\begin{questionbody}
% Recall that if $f \in \bb F[t]$ is a polynomial with $f(t) = \alpha_k t^k + \alpha_{k-1} t^{k-1} + \dotsb + \alpha_1 t + \alpha_0$, and $M$ is an $n \times n$ matrix, then $f(M)$ is defined to be the $n \times n$ matrix $f(M) = \alpha_k M^k + \alpha_{k-1} M^{k-1} + \dotsb + \alpha_1 M + \alpha_0 I_n$.

Let $r, s$ be positive integers and let $X \in M_r(\bb F), Z \in M_s(\bb F)$. Also, let $Y$ be an $r \times s$ matrix over $\bb F$, and consider the matrix \[
M := \begin{pmatrix}
X & Y \\
0_{s, r} & Z
\end{pmatrix}.
\] Let $f \in \bb F[x]$. Show that there exists an $r \times s$ matrix $Y_1$ such that \[
f(M) = \begin{pmatrix}
f(X) & Y_1 \\
0_{s, r} & f(Z)
\end{pmatrix}.
\]
\textit{Hint}: What does $M^d$ look like for a positive integer $d$?
\end{questionbody}

Answer

% }}}

\end{document}
