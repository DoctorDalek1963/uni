% vim: set foldmethod=marker foldlevel=0:

\documentclass[a4paper]{article}
\usepackage[UKenglish]{babel}

\usepackage[hidelinks]{hyperref}

\usepackage{preamble}

\renewcommand{\thesubsection}{Q\arabic{section}~(\roman{subsection})}

\fancyhead[L]{MA265 Assignment 4}
\title{MA265 Methods of Mathematical Modelling 3, Assignment 4}
\colorlet{questionbodycolor}{green!50!teal!50}

\begin{document}

\maketitle

\setlength{\parindent}{0em}
\setlength{\parskip}{1em}

% {{{ Q1
\question{1}

\begin{questionbody}
\textit{Convergence of Fourier series}: Let $f$ be a $2\pi$ periodic function on $\R$. Furthermore, let it be H\"older continuous, that is, there exists a constant $L > 0$ and an exponent $\alpha \in (0, 1]$ such that $\forall x \in [0, 2\pi]$, \[ % chktex 9
|f(x) - f(y)| \le L {|x - y|}^\alpha.
\]
\begin{enumerate}[(i)]
\item Show that the Fourier series of $f$ converges pointwise to itself.

\textbf{Hint}: Use Dini's~criterion: let $f$ be a $2\pi$ periodic function and assume there exists a $\delta > 0$ such that \[
\intlim 0 \delta {\f{|f(x+h) + f(x-h) - 2f(x)|}{h}} h < \infty.
\] Then the Fourier series of $f$ converges pointwise to itself.

\item Give an example of a H\"older continuous function on $\R$.
\end{enumerate}
\end{questionbody}

\subsection{~} % 1.i

Using Dini's~criterion, we just have to show that there exists a $\delta > 0$ such that \[
\intlim 0 \delta {\f{|f(x+h) + f(x-h) - 2f(x)|}{h}} h < \infty.
\] Then the Fourier series of $f$ converges pointwise to itself.

% TODO
This $\delta$ surely exists, but I don't know how to find it.

\subsection{~} % 1.ii

The identity function $f(x) = x$ is H\"older continuous. Take $\alpha = 1$ and any $L > 0$.

% }}}

% {{{ Q2
\newquestion{2}

\begin{questionbody}
\textit{Fourier series solutions}: Write the solution to the initial boundary value problem
\begin{equation}
\begin{cases}
\partial_t u(x, t) = \partial_{xx} u(x, t) & (x, t) \in (0, \pi) \times (0, \infty), \\[0.5ex]
u(x, 0) = x(\pi - x) & x \in (0, \pi), \\[0.5ex]
u(0, t) = 0 & t \in [0, \infty), \\[0.5ex] % chktex 9
u(\pi, t) = 0 & t \in [0, \infty) % chktex 9
\end{cases}
\label{eqn:Q2:ivbp}
\end{equation}
as a sine series.
\end{questionbody}

The solution will be of the form \[
% u(x, t) = \sum_{k = -\infty}^\infty c_k \sin(kx) \e^{-k^2 t}.
% u(x, t) = \sum_{k = 1}^\infty c_k \sin(kx).
u(x, t) = \sum_{j \in \N} \e^{-j^2 t} D_j \sin(j x)
\] where \[
x(\pi - x) = \sum_{j \in \N} D_j \sin(j x).
\]

% Then the derivatives are
% \begin{align*}
% \partial_t u(x, t) &= \sum_{k = 1}^\infty (\partial_t c_k) \sin(kx) \\[1.5ex]
% %
% \partial_x u(x, t) &= \sum_{k = 1}^\infty \l( k c_k \cos(kx) + (\partial_x c_k) \sin(kx) \r) \\[0.5ex]
% %
% \partial_{xx} u(x, t) &= \\[0.5ex]
% \end{align*}

We need to calculate the Fourier series of $\phi(x) = x (\pi - x)$.
\begin{align*}
\hat\phi(x) &= \f1{2\pi} \intlim {-\pi}\pi {x (\pi - x) \e^{-ikx}} x \\[0.5ex]
&= \f1{2\pi} \intlim {-\pi}\pi {\l( \pi x \e^{-ikx} - x^2 \e^{-ikx} \r)} x \\[0.5ex]
% &= \f1{2\pi} {\l[ \f{\pi (1 + ikx) \e^{-ikx}}{k^2} - \f{\pi (i k^2 x^2 + 2kx - 2i) \e^{-ikx}}{k^3} \r]}_{-\pi}^\pi \\[0.5ex]
\end{align*}
% TODO
I give up.

% }}}

% {{{ Q3
\newquestion{3}

\begin{questionbody}
\textit{Duhamel's principle}: Use Duhamel's principle to solve
\begin{equation}
\begin{cases}
\partial_t u(x, t) - \partial_{xx} u(x, t) = t \e^{-t} \sin(5x) & (x, t) \in (0, \pi) \times (0, \infty), \\[0.5ex]
u(x, 0) = 0 & x \in (0, \pi), \\[0.5ex]
u(0, t) = 0 & t \in [0, \infty), \\[0.5ex] % chktex 9
u(\pi, t) = 0 & t \in [0, \infty). % chktex 9
\end{cases}
\label{eqn:Q3:original-ibvp}
\end{equation}
You may use the fact that the general solution of $\partial_t v = \partial_{xx} v$ with homogeneous Dirichlet boundary conditions and homogeneous initial condition imposed at a time $\tau > 0$ is given by
\begin{equation}
v(x, t; \tau) = \sum_{j \in \N} D_j \e^{-j^2 (t - \tau)} \sin(jx).
\label{eqn:Q3:given-identity}
\end{equation}
\end{questionbody}

Using Duhamel's principle, we get a new problem in $w$: \[
\begin{cases}
\partial_t w(x, t; \tau) = \partial_{xx} w(x, t; \tau) & (x, t) \in (0, \pi) \times (\tau, \infty), \\[0.5ex]
w(x, \tau; \tau) = \tau \e^{-\tau} \sin(5x) & x \in (0, \pi), \\[0.5ex]
w(0, t; \tau) = 0 & t \in [\tau, \infty), \\[0.5ex] % chktex 9
w(\pi, t; \tau) = 0 & t \in [\tau, \infty). % chktex 9
\end{cases}
\]
The desired solution will be \[ u(x, t) = \intlim 0t {w(x, t; \tau)} \tau. \]

We know \[
w(x, \tau; \tau) = \tau \e^{-\tau} \sin(5x) = \sum_{j \in \N} D_j \e^{-j^2 \tau} \sin(jx)
\] which implies $j=5$ and is this case we get
\begin{align*}
\tau \e^{-\tau} &= D_5 \e^{-25 \tau} \\[0.5ex]
\implies D_5 &= \tau \e^{24 \tau} \\[0.5ex]
\end{align*}
%
We plug this into~\eqref{eqn:Q3:given-identity} and get
\begin{align*}
v(x, t; \tau) &= \tau \e^{24 \tau} \e^{-25(t - \tau)} \sin(5x) \\[0.5ex]
&= \tau \e^{49 \tau - 25 t} \sin(5x)
\end{align*}
%
Then we integrate to find
\begin{align*}
u(x, t) &= \intlim 01 {\tau \e^{49 \tau - 25t} \sin(5x)} \tau \\[0.5ex]
&= \e^{-25t} \sin(5x) \intlim 01 {\tau \e^{49 \tau}} \tau \\[0.5ex]
&= \e^{-25t} \sin(5x) {\l[ \f1{49^2} \e^{49 \tau} (49 \tau - 1) \r]}_0^t \\[0.5ex]
&= \f1{49^2} \e^{-25t} \sin(5x) \l( \e^{49t} (49t - 1) + 1 \r) \\[0.5ex]
&= \f{\sin(5x)}{49^2} \l( \e^{24t} (49t - 1) + \e^{-25t} \r)
\end{align*}
This clearly satisfies the initial conditions of~\eqref{eqn:Q3:original-ibvp}: when $t=0$, $x=0$, or $x=\pi$.

To see that it satisfies the PDE itself, we just need to differentiate.
\begin{align*}
\partial_t u(x, t) &= \f{\sin(5x)}{49^2} \l( 24 \e^{24t} (49t - 1) + 49 \e^{24t} - 25 \e^{-25t} \r) \\[0.5ex]
&= \f{\sin(5x)}{49^2} \l( \e^{24t} (24 \cdot 49t - 24 + 49) - 25 \e^{-25t} \r) \\[0.5ex]
&= \f{\sin(5x)}{49^2} \l( \e^{24t} (1176 t + 25) - 25 \e^{-25t} \r) \\[0.5ex]
%
\partial_x u(x, t) &= \f{5 \cos(5x)}{49^2} \l( \e^{24t} (49t - 1) + \e^{-25t} \r) \\[0.5ex]
%
\partial_{xx} u(x, t) &= \f{-25 \sin(5x)}{49^2} \l( \e^{24t} (49t - 1) + \e^{-25t} \r)
\end{align*}
Then
\begin{align*}
\partial_t u(x, t) - \partial_{xx} u(x, t) &= \f{\sin(5x)}{49^2} \l( \e^{24t} (1176 t + 25) - 25 \e^{-25t} \r)
    \\ &\qquad - \f{-25 \sin(5x)}{49^2} \l( \e^{24t} (49t - 1) + \e^{-25t} \r) \\[0.5ex]
&= \f{\sin(5x)}{49^2} \l( \e^{24t} (1176 t + 25) - 25 \e^{-25t} + 25 \e^{24t} (49t - 1) + 25 \e^{-25t} \r) \\[0.5ex]
&= \f{\sin(5x)}{49^2} \l( \e^{24t} (1176 t + 25) + 25 \e^{24t} (49t - 1) \r) \\[0.5ex]
&= \f{\sin(5x)}{49^2} \l( \e^{24t} \l( 1176 t + 25 + 1225 t - 25 \r) \r) \\[0.5ex]
&= \f{\sin(5x)}{49^2} \l( 2401 t \e^{24t} \r) \\[0.5ex]
&= t \e^{24t} \sin(5x) \\[0.5ex]
% TODO: Fix this
&= t \e^{-t} \sin(5x)
\end{align*}

\hfill $\square$

% }}}

% {{{ Q4
% \newquestion{4}

% \begin{questionbody}
% \textit{Inhomogeneous Neumann BCs (not marked)}: Consider the initial boundary value problem with Neumann boundary conditions \[
% \begin{cases}
% \partial_t u(x, t) = k \partial_{xx} u(x, t) & (x, t) \in (0, L) \times (0, \infty), \\[0.5ex]
% u(x, 0) = \Phi(x) & x \in (0, L), \\[0.5ex]
% \partial_x u(0, t) = h_0(t) & t \in [0, \infty), \\[0.5ex] % chktex 9
% \partial_x u(L, t) = h_L(t) & t \in [0, \infty) % chktex 9
% \end{cases}
% \] where $\Phi$, $h_0$, and $h_L$ are given smooth functions satisfying the compatibility conditions $h_0(0) = \Phi'(0)$ and $h_L(0) = \Phi'(L)$. Find the solution $u(x, t)$ by decomposing it into the sum of three terms: \[
% u = u^{(B)} + u^{(I)} + w,
% \] where $u^{(B)}$ is chosen to account for the Neumann boundary conditions, $u^{(I)}$ is chosen to incorporate the initial conditions, and $w$ is chosen such that the heat equation is satisfied. You do not need to solve for $w$ explicitly, simply state the PDE problem of which it is the solution.

% \textbf{Hint}: Construct an appropriate $u^{(B)}$ then follow the same logic as in the lectures. The problem for $w$ should be an inhomogeneous heat equation with homogeneous initial and boundary conditions.
% \end{questionbody}

% Answer

% }}}

\end{document} % chktex 17
