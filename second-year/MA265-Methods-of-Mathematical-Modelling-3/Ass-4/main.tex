% vim: set foldmethod=marker foldlevel=0:

\documentclass[a4paper]{article}
\usepackage[UKenglish]{babel}

% NOTE: hyperref has to come before preamble
% \usepackage[hidelinks]{hyperref}

\usepackage{preamble}

% \usepackage{graphicx}
% \graphicspath{ {./imgs/} }

\renewcommand{\thesubsection}{Q\arabic{section}~(\roman{subsection})}

\fancyhead[L]{MA265 Assignment 4}
\title{MA265 Methods of Mathematical Modelling 3, Assignment 4}
\colorlet{questionbodycolor}{green!50!teal!50}

\begin{document}

\maketitle

\setlength{\parindent}{0em}
\setlength{\parskip}{1em}

% {{{ Q1
\question{1}

\begin{questionbody}
\textit{Convergence of Fourier series}: Let $f$ be a $2\pi$ periodic function on $\R$. Furthermore, let it be H\"older continuous, that is, there exists a constant $L > 0$ and an exponent $\alpha \in (0, 1]$ such that $\forall x \in [0, 2\pi]$, \[ % chktex 9
|f(x) - f(y)| \le L {|x - y|}^\alpha.
\]
\begin{enumerate}[(i)]
\item Show that the Fourier series of $f$ converges pointwise to itself.

\textbf{Hint}: Use the so-called Dini-criterium: let $f$ be a $2\pi$ periodic function and assume there exists a $\delta > 0$ such that \[
\intlim 0 \delta {\f{|f(x+h) + f(x-h) - 2f(x)|}{h}} h < \infty.
\] Then the Fourier series of $f$ converges pointwise to itself.

\item Give an example of a H\"older continuous function on $\R$.
\end{enumerate}
\end{questionbody}

\subsection{~} % 1.i

Answer

\subsection{~} % 1.ii

Answer

% }}}

% {{{ Q2
\newquestion{2}

\begin{questionbody}
\textit{Fourier series solutions}: Write the solution to the initial boundary value problem \[
\begin{cases}
\partial_t u(x, t) = \partial_{xx} u(x, t) & (x, t) \in (0, \pi) \times (0, \infty), \\[0.5ex]
u(x, 0) = x(\pi - x) & x \in (0, \pi), \\[0.5ex]
u(0, t) = 0 & t \in [0, \infty), \\[0.5ex] % chktex 9
u(\pi, t) = 0 & t \in [0, \infty) % chktex 9
\end{cases}
\] as a sine series.
\end{questionbody}

Answer

% }}}

% {{{ Q3
\newquestion{3}

\begin{questionbody}
\textit{Duhamel's principle}: Use Duhamel's principle to solve \[
\begin{cases}
\partial_t u(x, t) - \partial_{xx} u(x, t) = t \e^{-t} \sin(5x) & (x, t) \in (0, \pi) \times (0, \infty), \\[0.5ex]
u(x, 0) = 0 & x \in (0, \pi), \\[0.5ex]
u(0, t) = 0 & t \in [0, \infty), \\[0.5ex] % chktex 9
u(\pi, t) = 0 & t \in [0, \infty). % chktex 9
\end{cases}
\] You may use the fact that the general solution of $\partial_t v = \partial_{xx} v$ with homogeneous Dirichlet boundary conditions and initial condition imposed at a time $\tau > 0$ is given by \[
v(x, t; \tau) = \sum_{j \in \N} D_j \e^{-j^2 (t - \tau)} \sin(jx).
\]
\end{questionbody}

Answer

% }}}

% {{{ Q4
% \newquestion{4}

% \begin{questionbody}
% \textit{Inhomogeneous Neumann BCs (not marked)}: Consider the initial boundary value problem with Neumann boundary conditions \[
% \begin{cases}
% \partial_t u(x, t) = k \partial_{xx} u(x, t) & (x, t) \in (0, L) \times (0, \infty), \\[0.5ex]
% u(x, 0) = \Phi(x) & x \in (0, L), \\[0.5ex]
% \partial_x u(0, t) = h_0(t) & t \in [0, \infty), \\[0.5ex] % chktex 9
% \partial_x u(L, t) = h_L(t) & t \in [0, \infty) % chktex 9
% \end{cases}
% \] where $\Phi$, $h_0$, and $h_L$ are given smooth functions satisfying the compatibility conditions $h_0(0) = \Phi'(0)$ and $h_L(0) = \Phi'(L)$. Find the solution $u(x, t)$ by decomposing it into the sum of three terms: \[
% u = u^{(B)} + u^{(I)} + w,
% \] where $u^{(B)}$ is chosen to account for the Neumann boundary conditions, $u^{(I)}$ is chosen to incorporate the initial conditions, and $w$ is chosen such that the heat equation is satisfied. You do not need to solve for $w$ explicitly, simply state the PDE problem of which it is the solution.

% \textbf{Hint}: Construct an appropriate $u^{(B)}$ then follow the same logic as in the lectures. The problem for $w$ should be an inhomogeneous heat equation with homogeneous initial and boundary conditions.
% \end{questionbody}

% Answer

% }}}

\end{document} % chktex 17
