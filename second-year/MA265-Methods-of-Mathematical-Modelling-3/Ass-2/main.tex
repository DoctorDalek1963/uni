% vim: set foldmethod=marker foldlevel=0:

\documentclass[a4paper]{article}
\usepackage[UKenglish]{babel}

\usepackage[hidelinks]{hyperref}

\usepackage{preamble}

\renewcommand{\thesubsection}{Q\arabic{section}~(\roman{subsection})}

\usepackage{graphicx}
\graphicspath{ {./imgs/} }

\fancyhead[L]{MA265 Assignment 2}
\title{MA265 Methods of Mathematical Modelling 3, Assignment 2}
\colorlet{questionbodycolor}{green!50!teal!50}

\begin{document}

\maketitle

\setlength{\parindent}{0em}
\setlength{\parskip}{1em}

% {{{ Q1
\question{1}

\begin{questionbody}
\textit{Transport equation}: Solve the equation $\partial_t u + v(x, t) \partial_x u = 0$ for $(x, t) \in \R \times (0, \infty)$ with velocity field $v(x, t) = x {(t + 1)}^{-2}$ and initial data $u(x, 0) = \cos x$.
\end{questionbody}

We will use the method of characteristics. We first have to solve the associated ODE\@.
\begin{align*}
\xi'(t) &= v(\xi(t), t) = \xi(t) {(t + 1)}^{-2} \\
\xi(0) &= x_0
\end{align*}
We can solve this with separation of variables.
\begin{align*}
\dd \xi t &= \xi(t) {(t + 1)}^{-2} \\
\inte{\f1{\xi}} \xi &= \inte{{(t + 1)}^{-2}} t \\
\ln \xi &= \f{-1}{t + 1} + C \\
\xi(t) &= A \e^{-1/(t + 1)}
\end{align*}
We use the initial value to find that
\begin{align*}
x_0 &= A \e^{-1} \\
x_0 \e &= A \\
\xi(t) &= x_0 \e^{1 - \f1{t + 1}} \\
\xi(t) &= x_0 \e^{\f{t + 1 - 1}{t + 1}} \\
\xi(t) &= x_0 \e^{t / (t + 1)}
\end{align*}
We know that $x = \xi(t)$, so we can solve for $x_0$ and find that \[
x_0 = x \e^{-t / (t + 1)}.
\]
Now we apply the fact that $u(x, t) = \cos x_0$ to find that \[
u(x, t) = \cos \l( x \e^{-t / (t + 1)} \r).
\]

% TODO: I don't think this actually solves the PDE

% }}}

% {{{ Q2
\newquestion{2}

\begin{questionbody}
\textit{Wave equation}: Solve the equation $u_{tt}(x, t) = 9 u_{xx}(x, t)$ for $(x, t) \in \R \times (0, \infty)$ with initial conditions $u(x, 0) = \cos x$ and $u_t(x, 0) = 2 \cos x$. Verify that if $x_n = \df{2n+1}2 \pi$ for $n \in \Z$, then $u(x_n, t) = 0$ for all $n \in \Z$ and $t \ge 0$.
\end{questionbody}

We can define
\begin{align*}
\phi(x) &= \cos x \\
V(x) &= 2 \cos x \\
c &= 3
\end{align*}
and use d'Alembert's Formula \[
u(x, t) = \f12 \big( \phi(x + ct) + \phi(x - ct) \big) + \f1{2c} \intlim {x - ct} {x + ct} {V(r)} r
\] to find the solution.

\begin{align*}
u(x, t) &= \f12 \big( \cos(x + 3t) + \cos(x - 3t) \big) + \f16 \intlim {x - 3t} {x + 3t} {2 \cos r} r \\
&= \f12 \big( \cos(x + 3t) + \cos(x - 3t) \big) + \f16 {\big[ 2 \sin r \big]}_{x - 3t}^{x + 3t} \\
&= \f12 \cos(x + 3t) + \f12 \cos(x - 3t) + \f13 \sin(x + 3t) - \f13 \sin(x - 3t)
\end{align*}

So if $x_n = \df{2n + 1}2 \pi$ then
\begin{align*}
u(x_n, t) &= \f12 \cos(x_n + 3t) + \f12 \cos(x_n - 3t) + \f13 \sin(x_n + 3t) - \f13 \sin(x_n - 3t) \\[0.5ex]
&= \f12 \cos\l(\f{2n + 1}2 \pi + 3t\r) + \f12 \cos\l(\f{2n + 1}2 \pi - 3t\r) \\[0.5ex]
    &\qquad + \f13 \sin\l(\f{2n + 1}2 \pi + 3t\r) - \f13 \sin\l(\f{2n + 1}2 \pi - 3t\r) \\[0.5ex]
\end{align*}
% TODO: This has to be wrong

% }}}

% {{{ Q3
\newquestion{3}

\begin{questionbody}
\textit{Burger's equation}: Consider Burger's equation $\partial_t u + u \partial_x u = 0$ for $(x, t) \in \R \times (0, \infty)$ with the initial conditions $u(x, 0) = \Phi(x)$ for all $x \in \R$. Use the method of characteristics to show that the solutions for the following initial data $\Phi$
\begin{enumerate}[(i)]
\item $\Phi(x) = \begin{cases}
    0 & x \le 0 \\
    \e^{-1 / x} & x > 0
\end{cases}$
\item $\Phi(x) = -x$
\end{enumerate}
are given by
\begin{enumerate}[(i)]
\item $u(x, t) = 0$ for $x \le 0$ and $u(x, t) = \e^{-1 / \xi}$ where $\xi > 0$ is such that $x - \xi = t \e^{-1 / \xi}$ for $x > 0$.
\item $u(x, t) = \df{-x}{1 - t}$ for $x \in \R$ and times $t < 1$.
\end{enumerate}
Sketch the characteristics in the $xt$-plane in each case.
\end{questionbody}

\subsection{~} % 3.i

For the first case of $\Phi(x) = \e^{-1/x}$ when $x > 0$ and $\Phi(x) = 0$ otherwise, we consider the ODE
\begin{align*}
\xi'(t) &= u(\xi(t), t) = \Phi(x) \\
\xi(0) &= x_0
\end{align*}

We shall consider the cases of $x > 0$ and $x \le 0$ separately. First the simpler case of $x \le 0$.
\begin{align*}
\zeta'(t) &= u(\zeta(t), t) \\[0.5ex]
\dd \zeta t &= 0 \\[0.5ex]
\inte{}\zeta &= 0 \\[0.5ex]
\zeta(t) &= C \\[0.5ex]
\zeta(0) &= C = x_0 \\[0.5ex]
\zeta(t) &= x_0 \\[0.5ex]
x &= x_0
\end{align*}
So for $x \le 0$, $u(x, t) = 0$ as required.

Now for the case of $x > 0$.
\begin{align*}
\xi'(t) &= u(\xi(t), t) \\[0.5ex]
\dd \xi t &= \e^{-1 / \xi} \\[0.5ex]
% \inte{\e^{1 / \xi}}\xi &= \inte{}t \\[0.5ex]
% &= t + C \\[0.5ex]
\inte{}\xi &= \inte{\e^{-1 / \xi}}t \\[0.5ex]
\xi(t) &= t \e^{-1 / \xi} + C \\[0.5ex]
\xi(0) &= C = x_0 \\[0.5ex]
\xi(t) &= t \e^{-1 / \xi} + x_0 \\[0.5ex]
\xi - x_0 &= t \e^{-1 / \xi}
\end{align*}
I'm not sure where to go from here. % TODO: Finish

\begin{figure}[tbhp]
\centering
\includegraphics[width=0.95\textwidth]{Q3-i-characteristics}
\caption{The characteristics for \textbf{Q3~(i)}}\label{fig:Q3-i-characteristics}
\end{figure}

\subsection{~} % 3.ii

For the case of $\Phi(x) = -x$, we once again consider the ODE
\begin{align*}
\xi'(t) &= u(\xi(t), t) = \Phi(x_0) \\
\xi(0) &= x_0
\end{align*}

We solve this like so,
\begin{align*}
\xi'(t) &= u(\xi(t), t) \\[0.5ex]
\dd \xi t &= \f{-\xi(t)}{1 - t} \\[0.5ex]
-\f1{\xi} \d \xi &= \f1{1 - t} \d t \\[0.5ex]
\inte{-\f1{\xi}}\xi &= \inte{\f1{1 - t}}t \\[0.5ex]
-\ln \xi &= -\ln (1 - t) + C \\[0.5ex]
\xi(t) &= A \cdot (1 - t) \\[0.5ex]
\xi(0) &= A = x_0 \\[0.5ex]
\xi(t) &= x_0 (1 - t).
\end{align*}
Or more simply, without involving $u$,
\begin{align*}
\xi'(t) &= \Phi(x_0) \\[0.5ex]
&= -x_0 \\[0.5ex]
\xi(t) &= -x_0 t + C \\[0.5ex]
\xi(0) &= C = x_0 \\[0.5ex]
\xi(t) &= -x_0 t + x_0 \\[0.5ex]
x &= x_0 (1 - t) \\[0.5ex]
x_0 &= \f x{1 - t}.
\end{align*}
And then $u(x, t) = \Phi(x_0) = \df{-x}{1 - t}$ as required.

\begin{figure}[tbhp]
\centering
\includegraphics[width=0.95\textwidth]{Q3-ii-characteristics}
\caption{The characteristics for \textbf{Q3~(ii)}}\label{fig:Q3-ii-characteristics}
\end{figure}

% }}}

% {{{ Q4
\newquestion{4}

\begin{questionbody}
\textit{Initial boundary value problems}: Consider the initial boundary value problem (IBVP):
\begin{equation}
\begin{cases}
\partial_{tt} u(x, t) = c^2 \partial_{xx} u(x, t) & (x, t) \in (0, \infty) \times (0, \times) \\
u(x, 0) = \Phi(x) & x \in (0, \infty) \\
\partial_t u(x, 0) = V(x) & x \in (0, \infty) \\
u(0, t) = 0 & t \in [0, \infty) % chktex 9
\end{cases} \label{eqn:Q4-ivbp}
\end{equation}
for given smooth functions $\Phi(x)$ and $V(x)$. Extend the functions $\Phi(x)$ and $V(x)$ on $(0, \infty)$ to odd functions $\tilde \Phi(x)$ and $\tilde V(x)$ on $\R$ by \[
\tilde \Phi(x) = \begin{cases}
\Phi(x)   & x > 0 \\
0         & x = 0 \\
-\Phi(-x) & x > 0
\end{cases}
\quad \text{and} \quad
\tilde V(x) = \begin{cases}
V(x)   & x > 0 \\
0      & x = 0 \\
-V(-x) & x > 0
\end{cases}
\]
Assume that $\tilde \Phi(x)$ and $\tilde V(x)$ are smooth functions. Show that d'Alembert's formula with data $\tilde \Phi(x)$ and $\tilde V(x)$ produces a solution of~\eqref{eqn:Q4-ivbp}.
\end{questionbody}

D'Alembert's formula with $\tilde \Phi$ and $\tilde V$ is \[
u(x, t) = \f12 \l(\tilde \Phi(x + ct) + \tilde \Phi(x - ct)\r) + \f1{2c} \intlim{x - ct}{x + ct}{\tilde V(r)}r
\]

% We will need some derivatives of this formula,
% \begin{align*}
% \partial_t u(x, t) &=
% \end{align*}

The wave equation is linear, so we can break this up into three functions and show that each one satisfies the equation. \[
u(x, t) = \f12 \Big(
    \underbrace{\tilde \Phi(x + ct)}_{\textbf{I}} +
    \underbrace{\tilde \Phi(x - ct)}_{\textbf{II}}
\Big) + \f1{2c}
\underbrace{\intlim{x - ct}{x + ct}{\tilde V(r)}r}_{\textbf{III}}
\]

For part \textbf{I}, let $v(x, t) = \tilde \Phi(x + ct)$. Then
\begin{align*}
\partial_t v &= c\, \tilde \Phi'(x + ct)
    & \partial_x v &= \tilde \Phi'(x + ct) \\[0.5ex]
\partial_{tt} v &= c^2 \tilde \Phi''(x + ct)
    & \partial_{xx} v &= \tilde \Phi''(x + ct)
\end{align*}
And of course by observation, $\partial_{tt} v = c^2 \partial_{xx} v$, so $v$ satisfies the wave equation.

Part \textbf{II} is shown in the same way. Let $v(x, t) = \tilde \Phi(x - ct)$. Then
\begin{align*}
\partial_t v &= -c\, \tilde \Phi'(x + ct)
    & \partial_x v &= \tilde \Phi'(x + ct) \\[0.5ex]
\partial_{tt} v &= c^2 \tilde \Phi''(x + ct)
    & \partial_{xx} v &= \tilde \Phi''(x + ct)
\end{align*}
And so $v$ satisfies the wave equation in this case also.

For part \textbf{III}, let $\ds v(x, t) = \intlim{x-ct}{x+ct}{\tilde V(r)}r$. Then
\begin{align*}
\partial_t v(x, t) &= \f12 \l(c\, \tilde \Phi'(x + ct) - c\, \tilde \Phi'(x - ct)\r) + \partial_t \f1{2c} \intlim{x - ct}{x + ct}{\tilde V(r)}r \\[0.5ex]
&= \f12 \l(c\, \tilde \Phi'(x + ct) - c\, \tilde \Phi'(x - ct)\r) + \f1{2c} \l( c \tilde V(x + ct) + c \tilde V(x - ct) \r) \\[0.5ex]
&= \f12 \l(c\, \tilde \Phi'(x + ct) - c\, \tilde \Phi'(x - ct) + \tilde V(x + ct) + \tilde V(x - ct) \r) \\[0.5ex]
%
\partial_{tt} v(x, t) &= \f12 \l(c^2\, \tilde \Phi''(x + ct) - c^2\, \tilde \Phi''(x - ct) + c \tilde V'(x + ct) + c \tilde V'(x - ct) \r) \\[0.5ex]
%
\partial_x v(x, t) &= \f12 \l(\tilde \Phi'(x + ct) - \tilde \Phi'(x - ct)\r) + \partial_x \f1{2c} \intlim{x - ct}{x + ct}{\tilde V(r)}r \\[0.5ex]
&= \f12 \l(\tilde \Phi'(x + ct) - \tilde \Phi'(x - ct)\r) + \f1{2c} \l( \tilde V(x + ct) - \tilde V(x - ct) \r) \\[0.5ex]
&= \f12 \l(\tilde \Phi'(x + ct) - \tilde \Phi'(x - ct) + \f1c \tilde V(x + ct) - \f1c \tilde V(x - ct) \r) \\[0.5ex]
%
\partial_{xx} v(x, t) &= \f12 \l(\tilde \Phi''(x + ct) - \tilde \Phi''(x - ct) + \f1c \tilde V'(x + ct) - \f1c \tilde V'(x - ct) \r)
\end{align*}
We can see that $\partial_{tt} v = c^2 \partial_{xx} v$, and so $v$ satisfies the wave equation. Since \textbf{I}, \textbf{II}, and \textbf{III} all satisfy the wave equation, and the wave operator is linear, d'Alembert's formula with $\tilde \Phi$ and $\tilde V$ satisfies the wave equation.

Now all that remains is to check the initial conditions. We first want $u(x, 0) = \tilde \Phi(x)$. D'Alembert's formula becomes
\begin{align*}
u(x, 0) &= \f12 \l(\tilde \Phi(x) + \tilde \Phi(x)\r) + \f1{2c} \intlim{x}{x}{\tilde V(r)}r \\[0.5ex]
&= \f12 \l(2\tilde \Phi(x)\r) \\[0.5ex]
&= \tilde \Phi(x)
\end{align*}

We also want $\partial_t u(x, 0) = \tilde V(x)$. D'Alembert's formula becomes
\begin{align*}
\partial_t u(x, t) &= \f12 \l(c\, \tilde \Phi'(x + ct) - c\, \tilde \Phi'(x - ct)\r) + \partial_t \f1{2c} \intlim{x - ct}{x + ct}{\tilde V(r)}r \\[0.5ex]
&= \f12 \l(c\, \tilde \Phi'(x + ct) - c\, \tilde \Phi'(x - ct)\r) + \f1{2c} \l( c \tilde V(x + ct) + c \tilde V(x - ct) \r) \\[0.5ex]
% \partial_t u(x, 0) &= \f12 \l(c\, \tilde \Phi'(x) - c\, \tilde \Phi'(x)\r) + \f1{2c} \intlim{x}{x}{\tilde V'(r)}r \\[0.5ex]
&= \f12 \l(c\, \tilde \Phi'(x + ct) - c\, \tilde \Phi'(x - ct) + \tilde V(x + ct) + \tilde V(x - ct) \r) \\[0.5ex]
\partial_t u(x, 0) &= \f12 \l(c\, \tilde \Phi'(x) - c\, \tilde \Phi'(x) + \tilde V(x) + \tilde V(x) \r) \\[0.5ex]
&= \tilde V(x)
\end{align*}

And finally, we want $u(0, t) = 0$.
\begin{align*}
u(0, t) &= \f12 \l(\tilde \Phi(ct) + \tilde \Phi(-ct)\r) + \f1{2c} \intlim{-ct}{ct}{\tilde V(r)}r \\[0.5ex]
&= \f12 \l(\tilde \Phi(ct) - \tilde \Phi(ct)\r) + \f1{2c} \intlim{-ct}{ct}{\tilde V(r)}r \\[0.5ex]
    %+ \f1{2c} \l( \intlim{0}{ct}{\tilde V(r)}r + \intlim{-ct}{0}{\tilde V(r)}r \r) \\[0.5ex]
&= \f12 \cdot 0 + \f1{2c} \cdot 0 \\
&= 0
% &= \f1{2c} \l( \intlim{0}{ct}{\tilde V(r)}r - \intlim{0}{-ct}{\tilde V(r)}r \r) \\[0.5ex]
% &= \f1{2c} \l( \intlim{0}{ct}{\tilde V(r)}r - \intlim{0}{ct}{\tilde V(r)}r \r) \tag{*}\label{eqn:Q4-odd-integral-symmetry} \\[0.5ex]
% &= 0
\end{align*}
% To justify~\eqref{eqn:Q4-odd-integral-symmetry}, we observe that $\tilde V$ is an odd function
To justify the integral becoming 0, we observe that $\tilde V$ is an odd function by definition, so integrating it from $-ct$ to $ct$ will always be 0.

And thus, we have a solution to~\eqref{eqn:Q4-ivbp}.

\hfill $\square$

% }}}

\end{document} % chktex 17
