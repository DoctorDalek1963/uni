% vim: set foldmethod=marker foldlevel=0:

\documentclass[a4paper]{article}
\usepackage[UKenglish]{babel}

\usepackage[hidelinks]{hyperref}

\usepackage{preamble}

% \usepackage{graphicx}
% \graphicspath{ {./imgs/} }

\fancyhead[L]{MA2K4 Assignment 1}
\title{MA2K4 Numerical Methods and Computing, Assignment 1}
\colorlet{questionbodycolor}{green!50}

\begin{document}

\maketitle

\setlength{\parindent}{0em}
\setlength{\parskip}{1em}

% {{{ Q1
\question{1}

\begin{questionbody}
For a vector $x \in \R^m$ and a matrix $A \in \R^{n \times n}$, and any norm $\|\cdot\|$ on $\R^n$, show that \[
\opnorm A = \sup_{x \ne 0} \f{\|Ax\|}{\|x\|}
\] is a norm on $\R^{n \times n}$.
\end{questionbody}

We need to show three things:
\begin{enumerate}[1)] % chktex 9 chktex 10
\item $\opnorm A \ge 0$ for all $A \in \R^{n \times n}$, and $\opnorm A = 0$ if and only if $A = 0$;

\item $\opnorm{\lambda A} = |\lambda| \, \opnorm A$ for all $A \in \R^{n \times n}$ and $\lambda \in \R$;

\item $\opnorm{A + B} \le \opnorm A + \opnorm B$ for all $A, B \in \R^{n \times n}$.
\end{enumerate}

Since $\|\cdot\|$ on $\R^m$ has the first property and we take the supremum over $x \ne 0$, we see that $\|\cdot\|_\mathrm{op}$ has the first property.

As before, $\|\cdot\|$ on $\R^m$ has the second property, so
\begin{align*}
\opnorm{\lambda A} &= \sup_{x \ne 0} \f{\|\lambda A x\|}{\|x\|} \\
&= \sup_{x \ne 0} \f{|\lambda| \, \|Ax\|}{\|x\|} \\
&= \sup_{x \ne 0} |\lambda| \f{\|Ax\|}{\|x\|} \\
&= |\lambda| \sup_{x \ne 0} \f{\|Ax\|}{\|x\|} \\
&= |\lambda| \opnorm A
\end{align*}
as required.

Since $\|\cdot\|$ on $\R^m$ has the triangle inequality, and since $\|\cdot\|_\mathrm{op}$ has property 2,
\begin{align*}
\| (A + B) x \| &= \| Ax + Bx \| \\
&\le \| Ax \| + \| Bx \| \\
&\le \big( \opnorm A + \opnorm B \big) \| x \|,
\end{align*}
and therefore
\begin{align*}
\opnorm{A + B} &= \sup_{x \ne 0} \f{\| (A + B) x \|}{\| x \|} \\
&\le \opnorm A + \opnorm B.
\end{align*}

% }}}

% {{{ Q2
\newquestion{2}

\begin{questionbody}
Compute the condition number $K(d)$ in the $|\cdot|$-norm on $\R$ for the following problems:
\begin{enumerate}[(a)]
\item $x - a^d = 0$ with $a > 0$;

\item the area of a triangle with all three sides having lengths $d$;
\end{enumerate}
where $d$ is the datum, and $x$ is the `unknown'. Comment on values of $d$ for which the problem is ill-posed. Compute the condition number via the derivative of the resolvent.
\end{questionbody}

\subsection{~} % 2.a

% The problem is ill-posed when $d$ is an even integer, since there will be multiple solutions. Consider $d=2$.
Since $a$ is a fixed constant, the problem is always well-posed.

We want $x = a^d$ and so $x = G(d) = a^d$. Then $G'(d) = a^d \log a$. Then we have
\begin{align*}
K(d) &\approx |G'(d)| \f{|d|}{|G(d)|} \\[0.5ex]
&= |a^d \log a| \f{|d|}{|a^d|} \\[0.5ex]
&= |d \log a|.
\end{align*}

\subsection{~} % 2.b

The problem is well-posed when $d > 0$.

Any triangle with all three sides of equal length must be equilateral, so its area is $x = G(d) = \df{d^2 \sqrt 3}4$. Then $G'(d) = \df{d \sqrt 3}2$. Then we have
\begin{align*}
K(d) &\approx |G'(d)| \f{|d|}{|G(d)|} \\[0.5ex]
&= \l| \f{d \sqrt 3}2 \r| \f{|d|}{\l| \f{d^2 \sqrt 3}4 \r|} \\[0.5ex]
&= \f{|d| \sqrt 3}2 \f{4 |d|}{d^2 \sqrt 3} \\[0.5ex]
&= \f{4 d^2 \sqrt 3}{2 d^2 \sqrt 3} \\[0.5ex]
&= 2.
\end{align*}

% }}}

% {{{ Q3
\newquestion{3}

\begin{questionbody}
For the system of equations \begin{equation}\label{eqn:Q3-system}
\begin{aligned}
x + dy &= 0 \\
dx + dy &= 1,
\end{aligned}
\end{equation}
consider the following questions:

\begin{enumerate}[(a)]
\item Considering the vector $(x, y)$ as the unknown and $d$ as given datum, for what values of $d$ is the problem well-posed?

\item Now, consider only $x$ the unknown. For what values of $d$ does a unique solution exist for this problem?

\item Again, considering $x$ the unknown, what is the condition number $K(d)$ in the $|\cdot|$-norm on $\R$?

\item For what values of $d$ is the problem well-conditioned? In this case, we demand the condition number to be smaller than 10.

Compute the condition number via the derivative of the resolvent.
\end{enumerate}
\end{questionbody}

\subsection{~} % 3.a

The problem is well-posed when the matrix $\begin{pmatrix}
1 & d \\
d & d
\end{pmatrix}$ has non-zero determinant, so whenever $d - d^2 \ne 0$. We can factorised this into $d(1 - d) \ne 0$, so we only need to require that $d \ne 0$ and $d \ne 1$. Any other value of $d$ gives a well-posed problem.

\subsection{~} % 3.b

If $x$ is the only unknown then we need $x = -dy$ to satisfy the first equation. Then we need $-d^2 y + dy = 1$ for the second equation, so we get
\begin{align*}
y d^2 - yd + 1 &= 0 \\
d &= \f{y \pm \sqrt{y^2 - 4y}}{2y} \\
&= \f{y \pm \sqrt{y (y - 4)}}{2y}
\end{align*}
as the only values of $d$ that admit a unique solution.

\subsection{~} % 3.c

% TODO: This has got to be the wrong idea. Maybe it means $y$ is a known constant and not part of the given data?

The resolvent map in this case is $x = G(d, y) = -dy$. We shall consider $G$ to be a function $\R^2 \to \R$, and denote the input space with $\begin{pmatrix} d \\ y \end{pmatrix}$ coordinates and endow it with the 1-norm. So \[
\opnorm G = \sup_{\|v\|_1 = 1} \|G(v)\|_1.
\] This clearly happens at $\begin{pmatrix} 1 \\ -1 \end{pmatrix}$ or $\begin{pmatrix} -1 \\ 1 \end{pmatrix}$, and so $\opnorm G = 1$.

To find the Fr\'echet derivative of $G$, we note that $\partial_d G = -y$ and $\partial_y G = -d$. Since $G$ is trivially Fr\'echet differentiable, we see that it is the Jacobian, \[
DG(d, y) = \partial G(d, y) = \begin{pmatrix} -y & -d \end{pmatrix}.
\]

Then we get \[
\opnorm{DG} = \sup_{\|v\|_1 = 1} \|DG(d, y) (v)\|_1,
\] which is $|1| + |1| = 2$.

And simply \[
\l\| \begin{pmatrix} d \\ y \end{pmatrix} \r\|_1 = |d| + |y|.
\]

Then we can compute the condition number as
\begin{align*}
K(d, y) &= \opnorm{DG} \f{\l\| \begin{pmatrix} d \\ y \end{pmatrix} \r\|_1}{\opnorm G} \\
&= 2 \f{|d| + |y|}{1} \\
&= 2(|d| + |y|).
\end{align*}

\subsection{~} % 3.d

Answer

% }}}

% {{{ Q4
\newquestion{4}

\begin{questionbody}
Consider the problem of finding the solution to a quadratic equation \[
x^2 + 2px - q =0,
\] which has the solutions \[
x_\pm = -p \pm \sqrt{p^2 + q}.
\] Consider each solution as a separate problem on $\R$. Study the conditioning of these two problems:

\begin{enumerate}[(a)]
\item Consider first $p$ as the datum to compute $K(p)$ in the $|\cdot|$-norm on $\R$. Simplify as much as possible to show that it does not depend on whether one considers $x_+$ or $x_-$.

\item Now, consider instead $q$ as the datum to compute $K(q)$ in the $|\cdot|$-norm on $\R$. Compute the condition number via the derivative of the resolvent.
\end{enumerate}
\end{questionbody}

\subsection{~} % 4.a

Answer

\subsection{~} % 4.b

Answer

% }}}

% {{{ Q5
\newquestion{5}

\begin{questionbody}
Write down the linear interpolant $p_1(x)$ for the function $f(x) = x^3 - x$, using the nodes $x_0 = 0$ and $x_1 = a$.

Denote by \[
e(x) = f(x) - p_1(x)
\] the error that we produced in the linear interpolation. Show that this error fulfils \[
e(x) = \f12 f''(\xi) x (x-a)
\] for the unique $\xi = \df13 (x+a)$.
\end{questionbody}

Answer

% }}}

% {{{ Q6
\newquestion{6}

\begin{questionbody}
Repeat this calculation for the function $f(x) = {(x-a)}^4$ for the same nodes, and show that in this case there are two possible values for $\xi$. Give their values.
\end{questionbody}

Answer

% }}}

\end{document} % chktex 17
