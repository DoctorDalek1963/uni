\documentclass[a4paper,11pt]{article}

\title{\S 8. Using \hologo{BibTeX} for \LaTeX\, referencing}
\author{Andrew Brendon-Penn}
\date{}
\usepackage{hologo} % only used to make the BibTeX logo in the title

% packages needed for the bibliography to work nicely:
\usepackage[numbers]{natbib}
\bibliographystyle{plainnat}
\usepackage{cite}

\usepackage{url}% allows use of the \url{...} command to typeset a url properly

\begin{document}

\maketitle

\section{My First Citations}

This is some text I want to cite from a page in a book \citep[p.~7]{LPSP}.

Sometimes I may want to reference several pages \citep[pp.~31--42]{MTS} or even a whole chapter \citep[\S~3]{LDC} or chapters \citep[\S\S~6--8]{MTS}.

It isn't often you should do this, but you can just reference a whole book \citep{ATD} by not including the optional argument.

\section{Retrieving Information From The Bibliography}

\citeauthor{LDC} % gives the author's surname
states in her \citeyear{LDC}  % gives the year
article \citep[p.~7]{LDC} that monic polynomials all have leading coefficients equal to 1. Whereas \citefullauthor{LPSP} % adapting the code for multiple authors
make no mention of such things in their book \citep{LPSP}.

On the other hand, \Citet[p.~78]{MTS} % gives the author and the proper citation in one go.
says something completely different.

\section{Citing Multiple Sources At Once}

If you want to cite more than one source at the same time, then you should do this within the same citation bracket \citep{ATD, MTS}.

If you want the optional arguments, then you can do this using more cumbersome code \citetext{\citealp[p.~10]{ATD};~\citealp[pp.~23--27]{MTS}}.

% \nocite{*} % used if you want to include everything in the bibliography even if they were never used.
\bibliography{main}

\end{document}
