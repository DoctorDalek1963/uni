\documentclass[a4paper,11pt]{article}

\title{\S7. Theorem Environments in \LaTeX}
\author{Andrew Brendon-Penn }

\usepackage{graphicx}
\usepackage{framed}
\usepackage{amsmath}
\usepackage{amssymb}
\usepackage{amsfonts}
\usepackage{cite}
\usepackage{mathrsfs}
\usepackage{array}

\usepackage{amsthm} %  package used to make the theorem environments work.

%
% code below sets up new theorem environments
%

\theoremstyle{plain} % this sets the style for all new environments created using \newtheorem to have the "theorem" style, which as a bold title, italic text and vertical space above and below it.
\newtheorem{thm}{Theorem}[section]
\newtheorem{lem}[thm]{Lemma}
\newtheorem{prop}[thm]{Proposition}
\newtheorem*{cor}{Corollary}
\newtheorem*{claim}{Claim}
\newtheorem*{lagrange}{Lagrange's Theorem} % the star makes an unnumbered theorem

\theoremstyle{definition} % this sets the style for all new environments created using \newtheorem to have the "definition" style, which as a bold title, upright text and vertical space above and below it.
\newtheorem{defn}{Definition}[section]
\newtheorem{eg}{Example}[section]

\theoremstyle{remark} % this sets the style for all new environments created using \newtheorem to have the "remark" style, which as an italic non-bold title, upright text and no extra vertical space above and below it.
\newtheorem*{rem}{Remark}
\newtheorem{case}{Case}

% note that there is already an in-built "proof" environment which we do not need to create

\begin{document}
\maketitle
\section{Group Theory}

I might start my section with a bit of introductory text highlighting the motivation, or signposting towards the end goal of this section. For example, I might want to point out the very important theorem we are going to prove.

\begin{lagrange}
	Here I state the theorem which I've \emph{chosen} to name rather than number.
\end{lagrange}

With this under my belt, I will probably want to make some definitions.

\begin{defn}
	Here I might define a \emph{subgroup}.
\end{defn}

I might like to mention that the definition is often axiom-heavy when it comes to double-checking, so I might provide a useful result.

\begin{prop}[The One Step Test]\label{prop:1steptest}
	Here I might give a minimal set of axioms which can be used to guarantee that a subset is in fact a subgroup.
\end{prop}

\begin{proof}
	I might prove the lemma here. Suppose that it splits into cases.
	\begin{case}
		First case proven
	\end{case}
	\begin{case}
		Second case proven
	\end{case}
	\begin{case}
		third case proven
	\end{case}
	And since all cases have been proven we are done.
\end{proof}

To demonstrate that Proposition~\ref{prop:1steptest} % note that the tilde creates a non-breaking space
really is useful, I might provide the following examples.
\begin{eg}
	Here I might show that $\mathbb{Z}[i]$ is a subgroup of $\mathbb{C}$.
\end{eg}

Some more intermediate text might lead me to the following definition.
\begin{defn}
	Here I would define a \emph{coset} $gH$.
\end{defn}
\begin{rem}
	Here I might point out some subtlety that the reader might have missed.
\end{rem}

I might add text which isn't part of the definition or lemma. This could be quite lengthy in fact because I have attendance to waffle on and on, so it's a good job this will get edited down at a later date. Do we have milk in or should I buy some on the way home? Such questions are almost certainly not answered by the following result.:

\begin{lem}\label{lem:LeftCosetsDisjoint}
	I might add a lemma about disjoint cosets.
\end{lem}

\begin{proof}
	Let's pretend I have proven my lemma.
	If the proof ends with displayed maths you might find that the tombstone is in the wrong place, you can change this using ``qedhere'' as I have here:
	\begin{equation}
		g^2H= g(gH)= gH.  \qedhere
	\end{equation}
\end{proof}

\begin{lem}\label{lem:SizeOfLeftCoset}
	We all know a good lemma about cosets being of equal size.
\end{lem}

\begin{proof}
	This is left as an exercise for the reader. (Never do this in your work!)
\end{proof}

\begin{thm}[Lagrange's Theorem] % the optional argument gives a secondary name to the theorem in parentheses
\label{thm:Lagrange}
	A very important theorem in group theory linking the cardinalities of the subgroup, index and group.
\end{thm}

\begin{proof}
	Proof involving cosets, and probably referring to Lemma~\ref{lem:LeftCosetsDisjoint} and Lemma~\ref{lem:SizeOfLeftCoset}.
	If my proof is very complicated, then I might want to introduce a claim.
	\begin{claim}
		Here is a claim that will help me with my proof.
	\end{claim}
	\begin{proof}[Proof of Claim] % the optional argument changes the title of the proof
		Here is where I might prove my claim.
	\end{proof}
	And now that the claim is established, I can finish my big proof.
\end{proof}

\section{More Group Theory}

\begin{defn}
	Now I might define a \emph{normal subgroup}.
\end{defn}

I might add text which isn't part of the definition or lemma.
\begin{defn}
	Now I might define the operation on cosets of a normal subgroup and call the result a \emph{quotient group}.
\end{defn}

\begin{lem}\label{lem:quotientgroup}
	I might add a lemma about the quotient group being a group.
\end{lem}

\begin{proof}
	Let's pretend I have proven my lemma. I might refer to Theorem~\ref{thm:Lagrange}.
\end{proof}

Some intermediate text might lead me to create the following.

\begin{defn}
	Now I might define the terms  \emph{homomorphism} and \emph{isomorphism}.
\end{defn}

\begin{thm}[The First Isomorphism Theorem for Groups] \label{thm:1stIsomorphismThm}
	Another very important theorem in group theory.
\end{thm}

\begin{proof}
	Proof involving cosets, and probably referring to Lemma~\ref{lem:quotientgroup} and Lemma~\ref{lem:SizeOfLeftCoset}.
	It probably doesn't but what if the proof  split into cases again.
	\setcounter{case}{0} % re-sets the numbering for the cases environment. Using \setcounter{case}{n} means the next "case" will be numbered n+1
	\begin{case}
		First case.
	\end{case}
	\begin{case}
		Second case.
	\end{case}
	\begin{case}
		third case.
	\end{case}
\end{proof}

\end{document}
