\documentclass[a4paper,11pt]{article}

\title{Exercise: \hologo{BibTeX} for \LaTeX\, referencing}
\author{Stu Dent}
\date{June 11, 2018}
\usepackage{hologo} % only used to make the BibTeX logo in the title

% packages needed for the bibliography to work nicely:
\usepackage[numbers]{natbib}
\bibliographystyle{plainnat}
\usepackage{cite}

\usepackage{url} % allows use of the \url{...} command to typeset a url properly

\begin{document}

\maketitle

\section{Mathematical Writing Style}

\citeauthor{knuth-1989-mathematical-writing} \citep[p.~1]{knuth-1989-mathematical-writing} remind us that it is poor style to start a sentence with a mathematical symbol whereas \citeauthor{higham-1998-handbook} \citep[pp.~17--18]{higham-1998-handbook} gives some useful ideas about how we might guide a reader through our proofs. Other handy guides to writing give examples of phrasing one might use to introduce or state a theorem \citep[\S~2]{trzeciak-1995-writing}.

\bibliography{main}

\end{document}
