\documentclass[a4paper,11pt]{article}

% this first part of the code is called the "pre-amble"
% it won't appear in the final document,
% the pre-amble is all the code before the line "\begin{document}"
% the pre-amble is what you set up the document including:
%     		information about the document (such as author, title, creation date)
% 			the look of the finished document
% 			what "packages" of LaTeX code can be used

\title{\S2 Understanding the basics of \LaTeX}
\author{Andrew Brendon-Penn}

\usepackage{amsmath}
\usepackage{amssymb}
\usepackage{amsfonts}

\begin{document}   % now comes the content - what will appear in the document itself

\maketitle  % this makes a title using the information you provided in the pre-amble


\section{My first section}

This is some text!

% this is just a comment, it won't appear in the document because of the % sign

This is me talking about percentages like 20\%.

\subsection{My first subsection}
Here's more text.

Notice      how        \LaTeX \        ignores         excess         spaces.
Do
you
notice
how
\LaTeX \
ignores
single
carriage
returns?
If I want to start a new paragraph, then I need to leave an empty line.

Or, I can leave lots of empty lines by pressing the enter key several times as I shall soon.




The effect is the same. Note that the first paragraph in a (sub)section is not indented, but all others are.



New paragraphs have an indented first line. But if I want to start a new line  without starting   a new paragraph, then I can use  the double  backslash to end the previous line.\\ This starts a new line without starting a new paragraph.




\section*{My first section without a number}

The star stopped \LaTeX \ from numbering the section.

\section{My first bit of maths in \LaTeX}

Here's some in-line maths
$\sum_{k=1}^{\infty} \frac{\beta_k x^k}{k!}$
which all fits on the same line.

And here's the same thing, but as displayed maths
\[
\sum_{k=1}^{\infty} \frac{\beta_k x^k}{k!}.
\]
The displayed maths is easier to read and is more prominent on the page.


\end{document}

\LaTeX will ignore this because the document has already ended.
