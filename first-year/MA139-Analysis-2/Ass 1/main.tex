% vim: set foldmethod=marker foldlevel=0:

\documentclass[a4paper]{article}
\usepackage[UKenglish]{babel}

\usepackage{preamble}

\usepackage{graphicx}
\graphicspath{ {./imgs/} }

\fancyhead[L]{MA139 Assignment 1}
\title{MA139 Analysis 2, Assignment 1}
\colorlet{questionbodycolor}{cyan!50}

\begin{document}

\maketitle

\setlength{\parindent}{0em}
\setlength{\parskip}{1em}

% {{{ Q1
\question{1}

\begin{questionbody}
Show that the series\[
\sum_{n=1}^\infty \f{x^n}n
\] converges for every $x$ in the interval satisfying $-1 \le x < 1$ and that it diverges for all other real values of $x$.
\end{questionbody}

Let $\ds S = \sum_{n=1}^\infty \f{x^n}n$. Using the ratio test, we find \begin{align*}
r &= \lim_{n \to \infty} \l| \l. \f{x^{n+1}}{n+1} \r/ \f{x^n}n \r|\\[1ex]
&= \lim_{n \to \infty} \l| \f{x^{n+1} n}{x^n (n+1)} \r|\\[1ex]
&= \lim_{n \to \infty} \l| x \f{n}{(n+1)} \r|\\[1ex]
&= |x| \lim_{n \to \infty} \l| \f{n}{(n+1)} \r|\\[1ex]
&= |x|
\end{align*}
$S$ converges when $r < 1$, so $S$ converges when $|x| < 1$.

Note that the ratio test requires the terms to be nonzero, but the terms of $S$ are only $0$ when $x=0$, and $S$ clearly also converges in this case.

Now we check the edges of the radius of convergence. When $x=1$, $S$ is the harmonic series, which we know diverges. % TODO: Prove this?

When $x=-1$, $S = \sum\limits_{n=1}^\infty {(-1)}^n \f1n$ and we can use the alternating series test. $\f1n$ is decreasing, non-negative, and converges to $0$, so $S$ must converge by the alternating series test.

Therefore $S$ converges when $-1 \le x < 1$.
% }}}

% {{{ Q2
\newquestion{2}

\begin{questionbody}
Let $\sum_{n=1}^\infty a_n x^n$ be the power series in which \[
a_n = \begin{cases}
1 & \text{if $n$ is a prime number} \\
0 & \text{if $n$ is not a prime number}
\end{cases} \] Prove that the series has radius of convergence 1.
\end{questionbody}

Let $S = \sum_{n=1}^\infty a_n x^n$. When $0 < x < 1$, \[
\sum_{n=1}^\infty a_n x^n < \sum_{n=1}^\infty x^n = \f1{1-x},
\] so $S$ converges.

Likewise when $-1 < x < 0$, \[
\sum_{n=1}^\infty |a_n x^n| < \sum_{n=1}^\infty |x|^n = \f1{1-|x|},
\] so $S$ converges absolutely.

$S$ trivially converges when $x=0$, so it converges when $-1 < x < 1$. Now we only need to check $x=1$ and $x=-1$.

We know there are infinitely many primes, so $\smlm_{n=1}^\infty a_n \to \infty$. And when $x = 1$, $S = \smlm_{n=1}^\infty a_n$, so $S$ cannot converge.

Note that all prime number except 2 are odd. So when $x = -1$, $S = -1 + 1 + 1 + 1 + \cdots = -1 + \smlm_1^\infty 1$, which clearly diverges to $\infty$.

Therefore, $S$ converges exactly when $-1 < x < 1$.
% }}}

% {{{ Q3
\newquestion{3}

\begin{questionbody}
Let $\sum a_n x^n$ be a power series with radius of convergence $R$. Show that if the closed interval $[-K, K]$ lies inside the interval $(-R, R)$ then the function $f$ given by \[
f(x) = \sum_{n=0}^\infty a_n x^n
\] is bounded on the interval $[-K, K]$.
\end{questionbody}

% For all $\varepsilon > 0$, we want $\delta > 0$ such that $|x-c| < \delta \implies |f(x) - f(c)| < \varepsilon$.
% \begin{align*}
% |f(x) - f(c)| &= \l| \smlm_{n=0}^\infty a_n (x^n - c^n) \r|\\[1ex]
% &\le \smlm_{n=0}^\infty \l| a_n (x^n - c^n) \r|
% \end{align*}

It was proven in lectures that any convergent power series is continuous in its radius of convergence. Therefore $f(x)$ is continuous in $(-R, R)$ and therefore also in $[-K, K]$.

Since $f(x)$ is continuous in $[-K, K]$, by the extreme value theorem, it is bounded on the interval and attains its bounds.
% }}}

% {{{ Q4
\newquestion{4}

\begin{questionbody}
By considering the ratio of successive terms, show that the sequence $\ds\l( \f{e^n n!}{n^n} \r)$ is increasing. Plot a graph of the first 10 or so terms.

Plot another graph of the squares $\ds{\l( \f{e^n n!}{n^n} \r)}^2$. How fast do you think they are growing?
\end{questionbody}

The ratio of successive terms is \begin{align*}
& \f{\e^{n+1}\, (n+1)!\, n^n}{\e^n\, n!\, {(n+1)}^{n+1}}\\[1ex]
&= \f{\e\, (n+1)\, n^n}{{(n+1)}^{n+1}}\\[1ex]
&= \e {\l(\f{n}{n+1}\r)}^n\\[1ex]
&= \e {\l(\f{n+1}n\r)}^{-n}\\[1ex]
&= \e {\l({\l(1 + \f1n\r)}^n\r)}^{-1}\\[1ex]
&\to \e {\l(\e\r)}^{-1}\\[1ex]
&= 1
\end{align*}

The ratio tends to 1. In particular, ${\l(1 + \df1n\r)}^n$ tends to $\e$ from below, so ${\l(1 + \df1n\r)}^{-n}$ tends to $\df1\e$ from above. Therefore the ratio of successive terms tends to 1 from above and the ratio is therefore always $\ge 1$, so the terms of the sequence are increasing.

\begin{figure}[hbtp]
    \centering
    \includegraphics[scale=0.4]{Q4a}
    \caption{The first 10 terms of $\l(\df{\e^n n!}{n^n}\r)$}
\end{figure}

\begin{figure}[hbtp]
    \centering
    \includegraphics[scale=0.4]{Q4b}
    \caption{The first 10 terms of ${\l(\df{\e^n n!}{n^n}\r)}^2$}
\end{figure}

The sequence ${\l(\df{\e^n n!}{n^n}\r)}^2$ looks linear. Experimentally it has a gradient of 6.288747, which looks suspiciously like $2\pi$. In fact, the line $y = 2\pi x + 1$ fits it almost exactly.

\begin{figure}[hbtp]
    \centering
    \includegraphics[scale=0.4]{Q4c}
    \caption{${\l(\df{\e^n n!}{n^n}\r)}^2$ in blue and $y = 2\pi x + 1$ in red}
\end{figure}
% }}}

\end{document}
