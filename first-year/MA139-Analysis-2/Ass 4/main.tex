% vim: set foldmethod=marker foldlevel=0:

\documentclass[a4paper]{article}
\usepackage[UKenglish]{babel}

\usepackage{preamble}

\fancyhead[L]{MA139 Assignment 4}
\title{MA139 Analysis 2, Assignment 4}
\colorlet{questionbodycolor}{cyan!50}

\begin{document}

\maketitle

\setlength{\parindent}{0em}
\setlength{\parskip}{1em}

% {{{ Q1
\question{1}

\begin{questionbody}
Let $F : \R \to \R$ be defined by \[
F(x) = \l( 1 - \f12 x + \f1{12} x^2 \r) \e^x.
\]
\end{questionbody}

\subsection{~} % 1.a

\begin{questionbody}
Use Taylor's theorem with remainder to show that \[
F(x) \ge 1 + \f12 x + \f1{12} x^2 \quad\text{ for } x \ge 0.
\]
\end{questionbody}

To apply Taylor's theorem with remainder to $F$, we need the first few derivatives of $F$.
\begin{align*}
F'(x) &= \l( 1 - \f12 x + \f1{12} x^2 \r) \e^x + \l( - \f12 + \f16 x \r) \e^x \\[1ex]
&= F(x) + \l( - \f12 + \f16 x \r) \e^x \\[1ex]
F''(x) &= F'(x) + \l( - \f12 + \f16 x \r) \e^x + \f16 \e^x \\[1ex]
&= F'(x) + \l( - \f13 + \f16 x \r) \e^x \\[1ex]
F^{(3)}(x) &= F''(x) + \l( - \f13 + \f16 x \r) \e^x + \f16 \e^x \\[1ex]
&= F''(x) + \l( - \f16 + \f16 x \r) \e^x \\[1ex]
F^{(4)}(x) &= F^{(3)}(x) + \l( - \f16 + \f16 x \r) \e^x + \f16 \e^x \\[1ex]
&= F^{(3)}(x) + \f16 x \e^x
\end{align*}
\begin{align*}
F^{(5)} &= F^{(4)}(x) + \f16 x \e^x + \f16 \e^x \\[1ex]
&= F^{(4)}(x) + \l( \f16 + \f16 x \r) \e^x \\[1ex]
F^{(n)}(x) &= F^{(n-1)}(x) + \f{n - 4 + x}6 \e^x
\end{align*}

We want to prove that $F^{(4)}(x) \ge 0$ for all $x \ge 0$. We have \begin{align*}
F^{(4)}(x) &= F^{(3)} + \f16 x \e^x \\[1ex]
&= F''(x) + \l( - \f16 + \f16 x \r) \e^x + \f16 x \e^x \\[1ex]
&= F''(x) + \l( - \f16 + \f13 x \r) \e^x \\[1ex]
&= F'(x) + \l( - \f13 + \f16 x - \f16 + \f13 x \r) \e^x \\[1ex]
&= F'(x) + \l( - \f12 + \f12 x \r) \e^x \\[1ex]
&= F(x) + \l( - \f12 + \f16 x - \f12 + \f12 x \r) \e^x \\[1ex]
&= F(x) + \l( - 1 + \f23 x \r) \e^x \\[1ex]
&= \l( 1 - \f12 x + \f1{12} x^2 - 1 + \f23 x \r) \e^x \\[1ex]
&= \f1{12} \e^x \l( 2 x^2 + x \r)
\end{align*}
Clearly $\df1{12} \e^x > 0$ for all $x$ and $2 x^2 + x \ge 0$ for all $x \ge 0$, therefore $F^{(4)}(x) \ge 0$ for all $x \ge 0$.

Now applying Taylor's theorem with Lagrange remainder around 0, we get \begin{align*}
F(x) &= F(0) + F'(0) x + \f{F''(0) x^2}2 + \f{F^{(3)}(0) x^3}6 + \f{F^{(4)}(t) x^4}{24} \\[1ex]
&= 1 + \f12 x + \f16 \f{x^2}2 + 0 x^3 + \f{F^{(4)}(t) x^4}{24} \\[1ex]
&= 1 + \f12 x + \f1{12} x^2 + \f1{24} x^4 F^{(4)}(t)
\end{align*}
for some $t$ between $0$ and $x$. Assuming $x \ge 0$, then $0 \le t \le x$.

Since $F^{(4)}(t) \ge 0$ for all $t \ge 0$, we get $\ds F(x) \ge 1 + \f12 x + \f1{12} x^2$ for all $x \ge 0$, as required.

\subsection{~} % 1.b

\begin{questionbody}
Show that for all $x$, \[
1 - \f12 x + \f1{12} x^2 \ge 0.
\]
\end{questionbody}

We shall treat $\ds \f1{12} x^2 - \f12 x + 1$ as a quadratic equation in $x$. Then we can see its discriminant \enquote{$b^2 - 4ac$} to be $\ds \f14 - \f4{12} = -\f1{12} < 0$. The discriminant is negative, which means the quadratic has no real roots.

Since the quadratic equation has no real roots, and the $x^2$ coefficient is positive, we can conclude that $\ds 1 - \f12 x + \f1{12} x^2 > 0\ \fa x \in \R$.

\subsection{~} % 1.c

\begin{questionbody}
Deduce that for $x \ge 0$, \[
\e^x \ge \f{1 + \f12 x + \f1{12} x^2}{1 - \f12 x + \f1{12} x^2}
\] and hence that $\e \ge \df{19}7$.
\end{questionbody}

In part \textbf{(a)}, we showed that for all $x \ge 0$, \[
\e^x \l( 1 - \f12 x + \f1{12} x^2 \r) \ge 1 + \f12 x + \f1{12} x^2
\] The quadratic in brackets on the LHS has discriminant $\f14 - \f13 < 0$, and positive coefficient of $x^2$, so that term in brackets is always strictly positive, so we can divide by it.

Therefore, for $x \ge 0$, \[ \e^x \ge \f{1 + \f12 x + \f1{12} x^2}{1 - \f12 x + \f1{12} x^2} \]

Plugging in $x=1$ gives \begin{align*}
\e &\ge \f{1 + \f12 + \f1{12}}{1 - \f12 + \f1{12}} \\[1ex]
&= \f{\f{18}{12} + \f1{12}}{\f6{12} + \f1{12}} \\[1ex]
&= \f{19}7
\end{align*}
Therefore $\e \ge \df{19}7$ as required.

% }}}

% {{{ Q2
\newquestion{2}

\begin{questionbody}
Suppose $f, g \colon [a, b] \to \R$ are integrable and agree except at finitely many points in the interval. Show that \[
\int_a^b f = \int_a^b g.
\]
\end{questionbody}

We know that $f(x) = g(x) \ \fa x \in [a,b] \setminus \{\enum c1n\}$.

% TODO: Make this argument rigorous?

Since $f$ and $g$ are integrable, there exist partitions $P$ and $Q$ of $[a,b]$ such that for any $\varepsilon > 0$, \begin{align*}
U(f, P) - L(f, P) &< \varepsilon \\
U(g, Q) - L(g, Q) &< \varepsilon
\end{align*}

Since $f$ and $g$ only disagree at finitely many points $\enum c1n$, for each $c_i$, either $f$ or $g$ (or both) must be discontinuous at $c_i$. Therefore either $P$ or $Q$ must \enquote*{cut out the bad bit} at $c_i$. Therefore we can take a common refinement $R$ of $P$ and $Q$, which will \enquote*{cut out the bad bits} at all $\enum c1n$.

Say for each $c_i$ we choose $\delta_i$ such that $[c_i - \delta_i, c_i + \delta_i]$ is one of the intervals in $R$, evidently the one containing $c_i$. Then for any $\varepsilon$, we can choose all $\delta_i$ small enough such that the rectangles in the upper and lower sums will have arbitrarily small area.

Let's say the sum of all the rectangles containing each $c_i$ for the upper sum of $f$ is $\Gamma_f$ and the sum of all the rectangles containing each $c_i$ for the lower sum of $f$ is $\gamma_f$. Let's also define $\Gamma_g$ and $\gamma_g$ similarly for $g$.

Let $S$ be the partition $R$ but excluding each of the intervals containing $\enum c1n$. Then \begin{align*}
U(f, R) &= U(f, S) + \Gamma_f \\
U(g, R) &= U(g, S) + \Gamma_g \\
L(f, R) &= L(f, S) + \gamma_f \\
L(g, R) &= L(g, S) + \gamma_g
\end{align*}

Since $f$ and $g$ agree at all points in $S$, $U(f, S) = U(g, S)$.

We can choose all the $\delta_i$ accordingly to make $\Gamma_f = \Gamma_g = \Gamma$ and $\gamma_f = \gamma_g = \gamma$ and make both arbitrarily small, so \begin{align*}
U(f, R) &= U(f, S) + \Gamma = U(g, R) \\
L(f, R) &= L(f, S) + \gamma = L(g, R)
\end{align*}

Since the upper sums and lower sums for $f$ and $g$ both agree on the partition $S$, we can conclude that $\ds \int_a^b f = \int_a^b g$.

% }}}

% {{{ Q3
\newquestion{3}

\begin{questionbody}
Find \[
\lim_{n \to \infty} \f1n \smlm_{k=1}^n \log \l( 1 + \f kn \r).
\]

Be careful to explain what facts you use from the course.
\end{questionbody}

This can be viewed as $\llim{n \to \infty} U(f, P_n)$ where $f(x) = \log(1 + x)$ and $P_n$ is the partition of $[0,1]$ into $n$ equal intervals, since the area of each rectangle is the width $\f1n$ times the height $f(x_i)$, and since $f$ is increasing and the sum always takes $f(x_i)$ on the right hand side of the interval, we get the upper sum.

% TODO: Elaborate on HW8 Q4?

Note that $f$ is continuous and so by Homework~8 Question~4, this upper sum converges to the integral \begin{align*}
\intlim 01 {\log(1 + x)} x &= {\big[ (1+x) \log (1+x) - x \big]}_0^1 \\[1ex]
&= 2 \log 2 - 1 - \log 1 - 0 \\[1ex]
&= 2 \log 2 - 1 % Correct answer as per SageMath
\end{align*}

% }}}

\end{document}
