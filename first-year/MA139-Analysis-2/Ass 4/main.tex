% vim: set foldmethod=marker foldlevel=0:

\documentclass[a4paper]{article}
\usepackage[UKenglish]{babel}

\usepackage{preamble}

\fancyhead[L]{MA139 Assignment 4}
\title{MA139 Analysis 2, Assignment 4}

\begin{document}

\maketitle

\setlength{\parindent}{0em}
\setlength{\parskip}{1em}

% {{{ Q1
\question{1}

Let $F : \R \to \R$ be defined as $$F(x) = \l( 1 - \f12 x + \f1{12} x^2 \r) \e^x$$

\subsection{~}

To apply Taylor's theorem with remainder to $F$, we need the first few derivatives of $F$. \begin{align*}
F'(x) &= \l( 1 - \f12 x + \f1{12} x^2 \r) \e^x + \l( - \f12 + \f16 x \r) \e^x\\[1ex]
&= F(x) + \l( - \f12 + \f16 x \r) \e^x\\[1ex]
F''(x) &= F'(x) + \l( - \f12 + \f16 x \r) \e^x + \f16 \e^x\\[1ex]
&= F'(x) + \l( - \f13 + \f16 x \r) \e^x\\[1ex]
F^{(3)}(x) &= F''(x) + \l( - \f13 + \f16 x \r) \e^x + \f16 \e^x\\[1ex]
&= F''(x) + \l( - \f16 + \f16 x \r) \e^x\\[1ex]
F^{(4)}(x) &= F^{(3)}(x) + \l( - \f16 + \f16 x \r) \e^x + \f16 \e^x\\[1ex]
&= F^{(3)}(x) + \f16 x \e^x
\end{align*}
\begin{align*}
F^{(5)} &= F^{(4)}(x) + \f16 x \e^x + \f16 \e^x\\[1ex]
&= F^{(4)}(x) + \l( \f16 + \f16 x \r) \e^x\\[1ex]
F^{(n)}(x) &= F^{(n-1)}(x) + \f{n - 4 + x}6 \e^x
\end{align*}

Now applying Taylor's theorem with Lagrange remainder around 0, we get \begin{align*}
F(x) &= F(0) + F'(0) x + \f{F''(0) x^2}2 + \f{F^{(3)}(0) x^3}6 + \f{F^{(4)}(t) x^4}{24}\\[1ex]
&= 1 + \f12 x + \f16 \f{x^2}2 + 0 x^3 + \f{F^{(4)}(t) x^4}{24}\\[1ex]
&= 1 + \f12 x + \f1{12} x^2 + \f1{24} \l( F^{(3)}(t) + \f16 t \e^t \r)
\end{align*}
for some $t$ between $0$ and $x$. Assuming $x \ge 0$, then $0 \le t \le x$.

% TODO: Prove $F^{(n)}(t) \ge 0$ for by induction?

Therefore $\ds F(x) \ge 1 + \f12 x + \f1{12} x^2$ for all $x \ge 0$.

\subsection{~}

We shall treat $\ds \f1{12} x^2 - \f12 x + 1$ as a quadratic equation in $x$. Then we can see its discriminant \enquote{$b^2 - 4ac$} to be $\ds \f14 - \f4{12} = -\f1{12} < 0$. The discriminant is negative, which means the quadratic has no real roots.

Since the quadratic equation has no real roots, and the $x^2$ coefficient is positive, we can conclude that $\ds 1 - \f12 x + \f1{12} x^2 > 0\ \fa x \in \R$.

\subsection{~}

% TODO: Prove this

Therefore, for $x \ge 0$, $$\e^x \ge \f{1 + \f12 x + \f1{12} x^2}{1 - \f12 x + \f1{12} x^2}$$

Plugging in $x=1$ gives \begin{align*}
\e &\ge \f{1 + \f12 + \f1{12}}{1 - \f12 + \f1{12}}\\[1ex]
&= \f{\f{18}{12} + \f1{12}}{\f6{12} + \f1{12}}\\[1ex]
&= \f{19}7
\end{align*}
Therefore $\e \ge \df{19}7$ as required.

% }}}

% {{{ Q2
\newquestion{2}

Let $f, g : [a, b] \to \R$ be Riemann integrable functions which agree except at finitely many points in the interval, so $f(x) = g(x) \ \fa x \in \R \setminus \{\enum c1n\}$.

% TODO: Show integrals are the same

% }}}

% {{{ Q3
\newquestion{3}

We want to find $\ds \llim{n \to \infty} \f1n \smlm_{k=1}^n \log \l( 1 + \f kn \r)$.
% NOTE: hint is HW8 Q4

% }}}

\end{document}
