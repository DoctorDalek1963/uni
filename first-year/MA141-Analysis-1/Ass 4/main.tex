% vim: set foldmethod=marker foldlevel=0:

\documentclass[a4paper]{article}
\usepackage[UKenglish]{babel}

\usepackage{preamble}

\fancyhead[L]{MA141 Assignment 4}
\title{MA141 Analysis 1, Assignment 4}
\colorlet{questionbodycolor}{blue!50}

\begin{document}

\maketitle

\setlength{\parindent}{0em}
\setlength{\parskip}{1em}

% {{{ Q5
\question{5}

\begin{questionbody}
Suppose that $f \colon \R \to \R$ and $g \colon \R \to \R$ are continuous functions, and that we have $f(x) = g(x)$ for every $x \in \Q$. Use sequential continuity to show that $f = g$ everywhere.
\end{questionbody}

For every real number $x$, we can define a sequence $(a_n)$ as $a_n = \df{\lfloor x \cdot 10^n \rfloor}{10^n}$, starting at $n=0$. This is a way to generate decimal truncations of $x$. For example if $x = \pi$, then $a_0 = 3,\ a_1 = 3.1,\ a_2 = 3.14,\ \ldots,\ a_{10} = 3.1415926535,\ \ldots$ It is clear that $a_n \to x$ as $n \to \infty$.

We can use this process of generating a sequence for any real number to fill in the gaps of $f$ and $g$. Let $x \in \R$ and define $(a_n)$ as above to be the sequence converging to $x$. All terms of $a_n$ are rational, so $f(a_0) = g(a_0),\ f(a_1) = g(a_1),\ \ldots$ Since $f$ and $g$ are continuous, $f(a_n) \to f(x)$ and $g(a_n) \to g(x)$, and since $f(a_n) = g(a_n)\ \fa n$, we must conclude that $f(x) = g(x)\ \fa x \in \R$.
% }}}

% {{{ Q7
\question{7}

\begin{questionbody}
Let $f \colon (-\infty, 0] \to \R$ and $g \colon [0, \infty) \to \R$ both be continuous on their entire domain. Show that the function \[
h(x) = \begin{cases}
	f(x) & x \le 0\\
	g(x) & x > 0
\end{cases} \]
is continuous at $x = 0$ (and hence on $\R$) if and only if $f(0) = g(0)$.
\end{questionbody}

If $h$ is continuous at 0, then $\fa \varepsilon > 0, \exists\ \delta > 0$ such that $|x| < \delta \implies |h(x)| < \varepsilon$.

I just don't know what to do with this question, sorry.
% }}}

% {{{ Q11
\newquestion{11}

\begin{questionbody}
Show that any continuous function $f \colon [a, b] \to [a, b]$ has a fixed point, i.e. there exists an $x^* \in [a, b]$ such that $f(x^*) = x^*$.

Hint: consider the function $g(x) = f(x) - x$ and use the Intermediate Value Theorem.

Give an example to show that the conclusion is not true if $f \colon (a, b) \to (a, b)$.
\end{questionbody}

Let $g(x) = f(x) - x$. Then we have three cases, either $g(a) < g(b)$, $g(a) > g(b)$, or $g(a) = g(b)$. We only get the final case if $f(x) = x$, in which case every point is a fixed point.

In the case of $g(a) < g(b)$, we know $g(a) < 0 < g(b)$
% TODO: Why is this true?
so by the Intermediate Value Theorem, $g(c) = 0$ for some $c \in (a, b)$. Therefore $f(c) = c$, so $c$ is a fixed point of $f$.

Likewise for the case of $g(a) > g(b)$, we can show $g(a) > 0 > g(b)$ in the same way, so we can find a fixed point using the same logic.

Now let $f : (a, b) \to (a, b)$. The example $f(x) = x^2$ would have fixed points at $x=0$ and $x=1$, but these are not in the domain, so $f(0)$ and $f(1)$ are not defined. Therefore $f$ has no fixed point and shows that we can avoid fixed points in this case.
% }}}

% {{{ Q19
\question{19}

\begin{questionbody}
Suppose that $f \colon [0, \infty) \to \R$ is continuous and that $f(x) \to L$ as $x \to \infty$. Show that $f$ is bounded above and below on $[0, \infty)$. Show that $f$ need not attain both its upper and lower bound.
\end{questionbody}

We know that any convergent sequence is bounded above and below, so we can just define the sequence $(a_n)$ as $a_n = f(n)$. Then $a_n \to L$ as $n \to \infty$ and since $(a_n)$ is bounded above and below, $f$ must also be bounded above and below.
% FIXME: This proof is definitely wrong

The function $f(x) = 1 - \df{1}{1+x}$ is defined on $[0, \infty)$ and its lower bound is $0$, which is achieves at $f(0) = 0$, but its upper bound is $1$, which it never reaches. $f(x) \to 1$ as $x \to \infty$, but $f$ never actually attains its upper bound.
% }}}

\end{document}
