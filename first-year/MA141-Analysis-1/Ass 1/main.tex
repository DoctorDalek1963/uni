% vim: set foldmethod=marker foldlevel=0:

\documentclass[a4paper]{article}
\usepackage[UKenglish]{babel}

\usepackage{preamble}

\fancyhead[L]{MA141 Assignment 1}
\title{MA141 Analysis 1, Assignment 1}
\colorlet{questionbodycolor}{blue!50}

\begin{document}

\maketitle

\setlength{\parindent}{0em}
\setlength{\parskip}{1em}

% {{{ Q1
\question{1}

\begin{questionbody}
Use induction to prove Bernoulli’s inequality: if $x > -1$ then for every $n \in N$, $(1 + x)^n \ge 1 + nx$. (Where do you use the fact that $x > -1$?)
\end{questionbody}

Bernoulli's inequality states: if $x > -1$ then for every $n \in \mathbb{N}$, $(1 + x)^n \ge 1 + nx$. We will prove this by induction.

The base case is $n = 0$. We get $(1 + x)^0 = 1$ and $1 + 0 \times x = 1$. Clearly $1 \ge 1$, so the base case holds.

Now assume that we know the inequality holds for some $n = k$, so $(1 + x)^k \ge 1 + kx$. Then
\begin{align*}
	(1 + x)^{k+1} &= \underbrace{(1 + x)^k}_{\ge 1 + kx} \underbrace{(1 + x)}_{> 0}\\[1ex]
				  &\ge (1 + kx) (1 + x) \qquad\text{since } 1 + x > 0 \\[1ex]
				  &= 1 + kx + x + kx^2\\[1ex]
				  &= 1 + (k+1)x + kx^2\\[1ex]
				  &\ge 1 + (k+1)x \qquad\quad\text{since }kx^2 \ge 0
\end{align*}
% }}}

% {{{ Q2
% \newquestion{2}
%
% We know two rules:
% \begin{enumerate}
% 	\item $x < y,\ y < z \implies x < z$
% 	\item $0 < a,\ x < y \implies ax < ay$
% \end{enumerate}
%
% We start with the assumptions that $0 < a < b$ and $0 < c < d$. Then we can conclude that $0 < ac < bc$ and $0 < bc < bd$, both by rule 1. We can then focus these inequalities to get $ac < bc$ and $bc < bd$, and then we can use rule 2 to conclude that $ac < bd$, which is what we wanted.
% }}}

% {{{ Q5
\newquestion{5}

\begin{questionbody}
Identify the greatest lower bound and least upper bound for each of the following sets, and prove that they are indeed the GLB and LUB; say whether these bounds are elements of the set.
\end{questionbody}

For each part, I shall use $S$ to refer to the set in question.

\subsection{$\{x : 0 \le x \le 1\}$}

The greatest lower bound is $0$, which is in the set. Suppose we have some other lower bound $\ell > 0$. We know that $0 \in S$ and $0 < \ell$, so $\ell$ cannot be a lower bound.

Likewise, the least upper bound is $1$, which is in the set. Suppose we have some other upper bound $\ell < 1$. We know that $1 \in S$ and $\ell < 1$, so $\ell$ cannot be an upper bound.

\subsection{$\{x : 0 < x < 1\}$}

The greatest lower bound is $0$, which is not in the set. Suppose we have some other lower bound $\ell > 0$. We know from the Archimedean property of real numbers that for any real number $\varepsilon > 0$, we can find a natural number $n$ such that $0 < \frac{1}{n} < \varepsilon$. Thus, we can find an $n$ such that $0 < \frac{1}{n} < \ell$, so $\frac{1}{n}$ is less than $\ell$ but also in $S$. That means that $\ell$ cannot be a lower bound.

The least upper bound is $1$, which is not in the set. Suppose we have some other upper bound $\ell < 1$ and a real number $\varepsilon > 0$. If $\varepsilon$ is sufficiently small, then $\ell + \varepsilon < 1$, so $\ell + \varepsilon \in S$. That means that $\ell$ cannot be an upper bound, since $\ell + \varepsilon$ is an upper bound $> \ell$. Therefore there cannot exist an upper bound $\ell < 1$, so $1$ is the least upper bound.

\subsection{$\left\{ 1 + \dfrac{1}{n} : n \in \mathbb{N} \right\}$}

We can enumerate this set as something like $$\left\{ 1 + 1, 1 + \frac12, 1 + \frac13, 1 + \frac14, 1 + \frac15, \ldots \right\}$$

The greatest lower bound is $1$, which is not in the set. $1$ is a lower bound since $\frac1n > 0\ \forall\ n \in \mathbb N$, so $1 + \frac1n > 1\ \forall\ n \in \mathbb N$. Suppose we have some other lower bound $\ell > 1$. By the Archimedean property of real numbers, we can find a natural number $k$ such that $\frac1k < \ell - 1$. Therefore $1 + \frac1k < \ell$, and since $k$ is a natural number, we know that $1 + \frac1k \in S$. Therefore $1 + \frac1k$ is a lower bound which is smaller than $\ell$, so $\ell$ cannot be a lower bound.

The least upper bound is $2$, which is in the set. The first element of the set is $2$, and every other element is $1 + \frac1n$, where $n > 1$. We can show that this is always less than $2$ when $n > 1$.
\begin{align*}
	n &> 1\\
	\implies 1 &> \frac1n\\
	\implies 2 &> 1 + \frac1n
\end{align*}
Therefore $2$ is an upper bound and since it's in the set, it is also the least upper bound.

\subsection{$\left\{ 2 - \dfrac{1}{n} : n \in \mathbb{N} \right\}$}

We can enumerate this set as something like $$\left\{ 2 - 1, 2 - \frac12, 2 - \frac13, 2 - \frac14, \ldots \right\}$$

The greatest lower bound is $1$, which is in the set. $1$ is a lower bound since $\frac1n \le 1\ \forall\ n \in \mathbb N$, so $2 - \frac1n \ge 1$. Suppose we have some lower bound $\ell > 1$. $\ell$ cannot be a lower bound since $1 \in S$ and $1 < \ell$. Therefore we cannot have a lower bound $> 1$, so $1$ is the greatest lower bound.

The least upper bound is $2$, which is not in the set. $2$ is an upper bound since $\frac1n > 0\ \forall\ n \in \mathbb N$, so $2 - \frac1n < 2$. Suppose we have some other upper bound $\ell < 2$. The Archimedean property of real numbers tells us that we can find a natural number $k$ such that $0 < \frac1k < 2 - \ell$. Therefore $\ell + \frac1k < 2$, so $\ell + \frac1k$ is an upper bound of $S$ which is $> \ell$, so $\ell$ cannot be an upper bound. Therefore $2$ is the least upper bound.

\subsection{$\left\{ 1 + \dfrac{(-1)^n}{n} : n \in \mathbb{N} \right\}$}

We can enumerate this set as something like $$\left\{ 1 - \frac11, 1 + \frac12, 1 - \frac13, 1 + \frac14, 1 - \frac15, 1 + \frac16, \ldots \right\}$$

The greatest lower bound is $0$, which is in the set. $0$ is a lower bound since $0 < \frac1n \le 1\ \forall\ n \in \mathbb N$, so $0 \le 1 \pm \frac1n \le 2$. Therefore every element of $S$ is $\ge 0$. We cannot have a lower bound $\ell > 0$, since $0 \in S$. So any $\ell > 0$ cannot be a lower bound, so $0$ must be the greatest lower bound.

The least upper bound is $\frac32$, which is in the set. We can discount all the odd $n$ values from the set, since they result in $1 - \frac1n$, which will always be $< 1$. Therefore to find the upper bound, we only have to focus on the even values of $n$, which result is $1 + \frac1n$. These values are $1 + \frac12, 1 + \frac14, 1 + \frac16, \ldots$ and it should be clear to see that the largest of these is $1 + \frac12 = \frac32$. Therefore $\frac32$ is an upper bound. We cannot have another upper bound $\ell < \frac32$ because $\frac32 \in S$, so $\ell$ could never be an upper bound. Therefore $\frac32$ is the least upper bound.

\subsection{$\left\{ q < 0 : q^2 < 4, q \in \mathbb{Q} \right\}$}

We can rewrite this set as something like $\left\{ q \in \mathbb Q : -2 < q < 0 \right\}$.

The greatest lower bound is $-2$, which is not in the set. This is a lower bound because the condition $q^2 < 4$ is equivalent to $-2 < q < 2$, so we need $-2 < q\ \forall\ q \in S$. Suppose we have some lower bound $\ell > -2$. The Archimedean property of real numbers tells us that we can find a natural number $n$ such that $\frac1n < \ell + 2$, therefore $-2 < \ell - \frac1n$. Since $\ell$ and $\frac1n$ and both rational, their difference is rational. Therefore $\ell - \frac1n \in S$, but $\ell - \frac1n < \ell$, so $\ell$ cannot be a lower bound. Therefore $-2$ is the greatest lower bound.

The least upper bound is $0$, which is not in the set. This is an upper bound because the definition of $S$ directly tells us that $q < 0\ \forall\ q \in S$. Suppose we have an upper bound $-2 < \ell < 0$. The Archimedean property of real numbers tells us that we can find a natural number $n$ such that $0 < \frac1n < -\ell$. Therefore $\ell < -\frac1n < 0$. Since $\ell > -2$, $-\frac1n > -2$, so $-\frac1n$ is in $S$, but it's bigger than $\ell$. Therefore $\ell$ cannot be an upper bound, so $0$ is the least upper bound.
% }}}

% {{{ Q7
\newquestion{7}

\begin{questionbody}
The integer part (or \enquote{floor}) of a rational number $x$, written $\lfloor x \rfloor$, is defined as \[
\lfloor x \rfloor = \text{largest integer } n \in \Z \text{ such that } n \le x;
\]

Use the Least Upper Bound Property to show that this quantity exists and satisfies \begin{equation}
x - 1 < \lfloor x \rfloor \le x.
\label{eqn:Q7-target}
\end{equation}

Hint: Consider the set $S = \l\{ m \in \Z \colon m \le x \r\}$. Show that it has a least upper bound $r$,
and use Lemma 1.6 with $t = r − 1$ to find $n \in S$ that satisfies the requirements for $\lfloor x \rfloor$ in \eqref{eqn:Q7-target}.
\end{questionbody}

We will consider the set $S = \l\{ m \in \Z \colon m \le x \r\}$. This is clearly bounded above by $x$, so by the \textit{Least Upper Bound Axiom}, we know that $S$ has a least upper bound $r = \sup S$. In the case of $x \in \mathbb Z$, we can see that $r = x$.

Then Lemma 1.6 tells us that $r = \sup S$ if and only if $r$ is an upper bound for $S$ and for every $t < r$, there exists $s \in S$ such that $s > t$.

We already know that $r = \sup S$ by the \textit{Least Upper Bound Axiom}. That means we also know that for every $t < r, \exists s \in S$ such that $s > t$. For the sake of satisfying equation $(1)$, we will choose $t = r - 1$. Therefore we know that there exists some element $n \in S$ such that $r - 1 < n$.

Since $n \in S$, we also know that $n \le x$. Therefore $r - 1 < n \le x$. Call this element $n = \lfloor x \rfloor$ and we can conclude that $r - 1 < \lfloor x \rfloor \le x$.

This isn't quite equation \eqref{eqn:Q7-target}, but I'm not sure how to finish off the argument. It's intuitive to me that $r = \lfloor x \rfloor = \sup S$ and $x - 1 < \lfloor x \rfloor$, but I don't know how to formalise those ideas into a proper argument.

We could set $t = x - 1$ and then prove that $x - 1 < r$, but that doesn't really help me prove anything, and the problem sheet suggests $t = r - 1$ anyway, so I'm not sure how to finish this argument.
% }}}

% {{{ Q9
\newquestion{9}
\renewcommand{\thesubsection}{Q\arabic{section}~(\roman{subsection})}

\begin{questionbody}
By following the argument of Proposition 1.7 we can show that for any $q \in \N$ and $y > 0$ there exists $x \in \R$ such that $x^q = y$. Take $y > 1$ and consider the set
\[ S = \l\{ x \in \mathbb R \colon x \ge 0 \text{ with } x^q < y \r\} \]
\end{questionbody}

\subsection{~} % 9.i

\begin{questionbody}
Show that $S$ is non-empty and bounded above. It follows from the Least Upper Bound Axiom that $S$ has a supremum: set $r = \sup S$, and note (why?) that $r \ge 1$.
\end{questionbody}

$S$ is non-empty, since $0 \in S$, and it is bounded above since $y$ is finite, so there will eventually be some $x$ such that $x^q > y$. That $x$ will be greater than the upper bound of $S$, so $S$ must be bounded above.

Thus, from the \textit{Least Upper Bound Axiom}, we know that $S$ has a supremum, $r = \sup S$. $1$ will always be $\in S$, since $1^q < y$ for any $q$ when $y > 1$. Thus, $r \ge 1$.

\subsection{~} % 9.ii

\begin{questionbody}
Use the binomial expansion to show that if $x \ge 1$ and $0 < \varepsilon < 1$ then $(x + \varepsilon)^q \le x^q (1 + 2^q \varepsilon)$.

Hint: $(1 + 1)^q = 2^q$. If you cannot do this part of the question, you can still use the result to try part \textbf{(iii)}.
\end{questionbody}

The binomial expansion of $(x + \varepsilon)^q$ gives \begin{align*}
	(x + \varepsilon)^q &= \sum_{k=0}^q \binom{q}{k} x^{q-k} \varepsilon^k\\[1ex]
						&= x^q + q x^{q-1} \varepsilon + \binom{q}{2} x^{q-2} \varepsilon^2 + \cdots + q x \varepsilon^{q-1} + \varepsilon^q
\end{align*}

We can then factor this to get \begin{align*}
	(x + \varepsilon)^q &= x^q \sum_{k=0}^q \binom{q}{k} \frac1{x^k} \varepsilon^k\\[1ex]
						&= x^q \sum_{k=0}^q \binom{q}{k} \left( \frac{\varepsilon}{x} \right)^k\\[1ex]
						&= x^q \left( 1 + \frac{\varepsilon}{x} \right)^q\\[1ex]
						% &\le x^q \left( 1 + \frac{\varepsilon}{1} \right)^q
\end{align*}

Somehow we conclude that $(x + \varepsilon)^q \le x^q \left( 1 + 2^q \varepsilon \right)$.

\subsection{~} % 9.iii

\begin{questionbody}
Suppose that $r^q < y$. Use part \textbf{(ii)} to show that $(r + \varepsilon)^q < y$ for some sufficiently small $\varepsilon > 0$, and hence deduce a contradiction with the fact that $r$ is an upper bound for $S$.
\end{questionbody}

Suppose that $r^q < y$ and let $0 < \varepsilon < 1$.

By part $(ii)$, we know $(r + \varepsilon)^q \le r^q (1 + 2^q \varepsilon)$. Thus if $r^q < y$, then $(r + \varepsilon)^q \le r^q (1 + 2^q \varepsilon) < y (1 + 2^q \varepsilon)$.

For sufficiently small $\varepsilon$, $(1 + 2^q \varepsilon) \approx 1$, but I don't know how to make the jump and show that $(r + \varepsilon)^q < y$.

Since $(r + \varepsilon)^q < y$, $r + \varepsilon \in S$. But $\varepsilon > 0$, so $r + \varepsilon > r$. Thus, we have found an element of $S$ which is greater than $r$. That's a contradiction, since $r = \sup S$. Therefore, we know that $r^q < y$ must be false.

\subsection{~} % 9.iv

\begin{questionbody}
Suppose that $r^q > y$. Use Bernoulli’s Inequality from \textbf{Q1} to show that ${(r - \varepsilon)^q} > y$ for some sufficiently small $\varepsilon > 0$, and hence deduce a contradiction with the fact that $r$ is the least upper bound for $S$.
\end{questionbody}

Suppose that $r^q > y$ and let $0 < \varepsilon < 1$.

\textit{Bernoulli's Inequality} tells us that if $x > -1$, then $\forall\ n \in \mathbb N$, $(1 + x)^n \ge 1 + nx$. Since $0 < \varepsilon < 1$, we know that $-\varepsilon > -1$ and $q \in \mathbb N$. Therefore Bernoulli's Inequality tells us that $(1 - \varepsilon)^q \ge 1 - q\varepsilon$.

I have no idea how to complete this argument, but I know it ends by showing that $(r - \varepsilon)^q > y$.

Since $\varepsilon > 0$, $r - \varepsilon < r$ but $r - \varepsilon \notin S$. Additionally, $r \notin S$ since $r^q > y$, but every $s \in S$ requires $s^q < y$. Thus, we have found a better upper bound for $S$. $r - \varepsilon$ is an upper bound for $S$ which is smaller that $r$, so $r$ cannot be the supremum of $S$. This is a contradiction, therefore $r^q > y$ must be false.

Thus, since $r^q \not < y$ and $r^q \not > y$, we must conclude that $r^q = y$. \hfill $\square$
% }}}

\end{document}
