\documentclass[a4paper]{article}
\usepackage[UKenglish]{babel}

\usepackage{preamble}
\usepackage{array}

\title{MA151 Algebra 1, Assignment 4}
\author{Dyson Dyson}
\date{}

\begin{document}

\maketitle

\setlength{\parindent}{0em}
\setlength{\parskip}{1em}

\question{1}

Let $R$ be a non-zero ring. Let $a \sim b$ be the relation \enquote*{$a$ is an associate of $b$}, meaning there exists a unit $v \in R$ such that $av = b$.

\subsection{~}

An equivalence relation is reflexive, transitive, and symmetric.

For reflexivity, if $a \sim b$, then $\exists v \in R$ such that $av = b$ and $v$ is a unit. Then $a v v^{-1} = b v^{-1} \implies b v^{-1} = a$, so $b \sim a$.

Now for transitivity, suppose $a \sim b$ and $b \sim c$, so $\exists v, u \in R$ such that $av = b$ and $bu = c$. Then $(av) u = bu = c \implies a (vu) = c$, so $a \sim c$.

And for symmetry, $a 1 = a$, so $a \sim a$.

Therefore this is an equivalence relation.

\subsection{~}

Suppose $R = \Z$. The only units in $\Z$ are $\{1, -1\}$, so $a \sim b$ if and only if $a = b$ or $a = -b$. Therefore $0$ is equivalent to nothing, and every positive integer $x$ gets the equivalence class $[x]_\sim = \{x, -x\}$.

\question{2}

Let $R$ be a ring and let $a \in R$.

\subsection{~}

Suppose $R$ is commutative, and let $aR = \{ar : r \in R\}$. For $aR$ to be an ideal of $R$, we need $(aR, +)$ to be a subgroup of $(R, +)$, and we need $xy \in aR$ and $yx \in aR$ for all $x \in R,\ y \in aR$. Since $R$ is commutative, we only need to worry about one of these.

First, the ABC test for subgroups. The identity in $(R, +)$ is just $0$, which is trivially in $aR$. The sum of two terms $ar_1$ and $ar_2$ is $a(r_1 + r_2)$. Clearly $r_1 + r_2 \in R$, so $a(r_1 + r_2) \in aR$. The inverse of an element $ar$ is just $-ar = a(-r)$, and $-r \in R$, so $-ar \in aR$. Therefore $(aR, +)$ is a subgroup of $(R, +)$.

Now consider an arbitrary element $ar \in aR$ and an arbitrary element $x \in R$. Their product is $arx = a(rx)$, and since $rx \in R$, $a(rx) \in aR$. Therefore $aR$ is an ideal of $R$.

\subsection{~}

Now if we allow $R$ to be non-commutative, we could choose $R = GL_2(\R)$ and $a = \begin{pmatrix}1 & 1\\ 1 & 0\end{pmatrix}$. Then right-multiplying an element of $aR$ by an element of $R$ would keep the result in $aR$, but left-multiplying wouldn't necessarily. Therefore $aR$ is not an ideal of $R$ in this case.

\question{3}

\subsection{~}

$$\l(\Z / 7\Z\r)^* = \{1, 2, 3, 4, 5, 6\}$$

\subsection{~}

$$\l(\Z / 8\Z\r)^* = \{1, 3, 5, 7\}$$

\subsection{~}

We shall just draw the Cayley tables for these groups.

First, $\l(\Z / 7\Z\r)^*$,
\[
\setlength{\extrarowheight}{3pt}% local setting
\begin{array}{l|*{6}{l}}
	\times_7 & 1 & 2 & 3 & 4 & 5 & 6 \\
\hline
	1 & 1 & 2 & 3 & 4 & 5 & 6 \\
	2 & 2 & 4 & 6 & 1 & 3 & 5 \\
	3 & 3 & 6 & 2 & 5 & 1 & 4 \\
	4 & 4 & 1 & 5 & 2 & 6 & 3 \\
	5 & 5 & 3 & 1 & 6 & 4 & 2 \\
	6 & 6 & 5 & 4 & 3 & 2 & 1
\end{array}
\]

And then, $\l(\Z / 8\Z\r)^*$,
\[
\setlength{\extrarowheight}{3pt}% local setting
\begin{array}{l|*{7}{l}}
	\times_8 & 1 & 3 & 5 & 7 \\
\hline
	1 & 1 & 3 & 5 & 7 \\
	3 & 3 & 1 & 7 & 5 \\
	5 & 5 & 7 & 1 & 3 \\
	7 & 7 & 5 & 3 & 1 \\
\end{array}
\]

Just from looking at these tables, we can deduce that $\l( \l(\Z / 7 \Z\r)^*, \times_7 \r) \cong C_6$ and $\l( \l(\Z / 8 \Z\r)^*, \times_8 \r) \cong K_4$. But $K_4$ is not cyclic, so $\l( \l(\Z / 8 \Z\r)^*, \times_8 \r)$ is not cyclic.

\question{4}

Let $R = M_{2 \times 2}(\Q)$. We will show that the only ideals of $R$ are $\{0\}$ and $R$.

Let $\bf 0 = \begin{pmatrix}0 & 0\\ 0 & 0\end{pmatrix}$. If an ideal of $R$ did not contain $\bf 0$, then it wouldn't be a subgroup under addition because it wouldn't have an additive identity. Therefore every ideal needs $\bf 0$. Also note that $\{ \bf 0 \}$ is itself an ideal of $R$, since multiplying by anything from $R$ just results in $\bf 0$ again.

Now suppose we have some ideal $I$ containing $\bf 0$ and some $X \ne \bf 0$. Since $(I, +)$ is a group, it must also contain all integer multiples of $X$. And since $mX \in I$ and $Xm \in I$ for all $m \in R$, $I$ must expand to include all of $R$.

To see this, we can imagine an arbitrary "target" matrix $t \in R$, then find the matrix $m$ such that $Xm = t$. Let $X = \begin{pmatrix}x & y\\ z & w\end{pmatrix}$, $m = \begin{pmatrix}p & q\\ r & s\end{pmatrix}$, and $t = \begin{pmatrix}a & b\\ c & d\end{pmatrix}$. Then \begin{align*}
	\begin{pmatrix}x & y\\ z & w\end{pmatrix} \begin{pmatrix}p & q\\ r & s\end{pmatrix} &= \begin{pmatrix}a & b\\ c & d\end{pmatrix}\\[1ex]
	px + ry &= a\\
	qx + sy &= b\\
	pz + rw &= c\\
	qz + sw &= d
\end{align*}
Since $x, y, z, w, a, b, c, d$ are all known, these equations can always be solved for $p, q, r, s$. Therefore $\forall\, t \in R$, $\exists\, m \in R$ such that $Xm = t$. Therefore $I$ must contain all elements of $R$, so $I = R$.

Therefore the only ideals of $R$ are $\{ \bf 0 \}$ and $R$.

\question{5}

Let $R = \R[x]$ and let $I = \l\{ f(x) \in \R[x] : f(0) = 0 \r\}$.

\subsection{~}

Clearly $I \ne R$, since there exists polynomials in $f(x) \in\R[x]$ where $f(0) \ne 0$. Take $f(x) = x^2 + 1$, for instance. In this case, $f(0) = 1$. Therefore $I \ne R$.

For $I$ to be an ideal of $R$, we need $(I, +)$ to be a subgroup of $(R, +)$, for which we will use the ABC test, and we need $ir \in I$ and $ri \in I$ for all $r \in R, i \in I$, but multiplication is commutative here, so we only need to worry about one of these.

First, the ABC test for subgroups. The identity in $(R, +)$ is just $0$, which is trivially in $I$. The sum of two polynomials with zero constant term is another polynomial with zero constant term, so the sum of two elements in $I$ is another element in $I$. And the inverse of a polynomial with zero constant term is the negative version of that polynomial, which also has zero constant term, so $(I, +)$ has inverses. Therefore $(I, +)$ is a subgroup of $(R, +)$.

Now consider an arbitrary polynomial $r$ from $\R[x]$ and an arbitrary polynomial $i$ from $I$. To find the constant term of their product, we just find the product of their constant terms. Since $i$ has a constant term of $0$, $ri$ and $ir$ both have a constant term of $0$, so are both members of $I$.

Therefore $I$ is an ideal of $R$.

\subsection{~}

Suppose $J$ is an ideal of $R$ with $I \subsetneq J$. Since $I$ definitionally includes all polynomials with zero constant term, $J$ must include at least one polynomial with non-zero constant term. Without loss of generality, assume we have some $j(x) \in J$ where $j(0) = a$ and $a \ne 0$. Then for $J$ to be an ideal of $R$, we need $r(x) j(x) \in J$ and $j(x) r(x) \in J$ for all $x \in R$, although multiplication is commutative here, so we only need to worry about one of these.

Since $r(x)$ could be any element from $\R[x]$, we will end up generating all of $\R[x]$. We know that $J$ already contains every combination of real coefficients for powers of $x$, but we only know that it contains constant term $a$. But we can obtain any constant term $b$ by multiplying by some particular $r(x)$ with $r(0) = \frac ba$, since $j(0) = a$ and $a \ne 0$. Then $r(0) j(0) = b$, so $J$ must contain polynomials that cover all real constant terms.

Thus, $J$ must contain every polynomial from $\R[x]$, so $J = R$.

\question{6}

\subsection{~}

Consider $f(x) = x^3 + x^2 + x + 1$. $f(x)$ is not irreducible over $\Q$ since $x^3 + x^2 + x + 1 = (x + 1) (x^2 + 1)$.

\subsection{~}

Consider $f(x) = x^4 + 1$. We can use Eisenstein's criterion to show that $f(x)$ is irreducible over $\Q$, recalling the fact that $f(x)$ is irreducible if and only if $f(x+1)$ is irreducible. In this case $f(x+1) = x^4 + 4x^3 + 6x^2 + 4x + 2$.

Now we will choose our prime $p=2$. $p$ divides all the coefficients, excluding the coefficient of the term with the highest degree. $p \nmid 1$, and $p^2 \nmid 2$. Therefore $f(x+1)$ fulfils Eisenstein's criterion and is therefore irreducible over $\Q$. Therefore $f(x)$ is irreducible over $\Q$.

But $x^4 + 1 = \l(x^2 + \sqrt2 x + 1\r) \l(x^2 - \sqrt2 x + 1\r)$, so $f(x)$ is not irreducible over $\R$.

\subsection{~}

Consider $f(x) = x^2 + x + 4$. If $f(x)$ were not irreducible over $\Z / 11 \Z$, then we could write $x^2 + x + 4 = (ax + b) (cx + d) = acx^2 + (ad + bc)x + bd$, which gives the following system of equations, \begin{align*}
	ac &\stackrel{11}{\ \equiv\ } 1\\
	ad + bc &\stackrel{11}{\ \equiv\ } 1\\
	bd &\stackrel{11}{\ \equiv\ } 4\\
\end{align*}

And then I get stuck.
% TODO: What now?

\subsection{~}

Consider $f(x) = x^4 + 1$. The question says this is not irreducible over $\Z / 5 \Z$, but I don't know why. I can't find a root modulo 5, so it has no linear factors, but factoring into two quadratics doesn't seem to work either because I get two simultaneous equations mod 5 and nothing satisfies both.

\end{document}
