% vim: set foldmethod=marker foldlevel=0:

\documentclass[a4paper]{article}
\usepackage[UKenglish]{babel}

\usepackage{preamble}

\usepackage{graphicx}
\graphicspath{ {./imgs/} }

\fancyhead[L]{MA150 Assignment 2}
\title{MA150 Algebra 2, Assignment 2}
\colorlet{questionbodycolor}{magenta!50}

\begin{document}

\maketitle

\setlength{\parindent}{0em}
\setlength{\parskip}{1em}

% {{{ Q5
\question{5}

\begin{questionbody}
Let $P = \begin{pmatrix}1\\ 2\\ 3\end{pmatrix}$ and $Q = \begin{pmatrix}-2\\ 1\\ 5\end{pmatrix} \in \R^3$ and let $L$ be the (unique, infinite) line that passes through them both.

Describe $L$ in two ways: first, parametrically as $$L = \l\{ \ul v + \lambda \ul w : \lambda \in \R \r\}$$
where $\ul v$ is a point of $L$ and $\ul w$ is a vector parallel to $L$, and second implicitly by two equations of the form $ax + by + cz = d$ (for suitable values $a, b, c, d \in \R$).
\end{questionbody}

We can parametrise $L$ as $\overrightarrow{P\,} + \lambda \overrightarrow{PQ\,}$ and $\overrightarrow{PQ\,} = \overrightarrow{Q\,} - \overrightarrow{P\,} = \begin{pmatrix}-3\\ -1\\ 2\end{pmatrix}$ so $L$ can be parametrised as $$L = \l\{ \begin{pmatrix}1\\ 2\\ 3\end{pmatrix} + \lambda \begin{pmatrix}-3\\ -1\\ 2\end{pmatrix} : \lambda \in \R \r\}$$

To describe $L$ in terms of two implicit equations, we need two planes. We will start with the plane $\Pi_1$ through $P$, $Q$, and the origin. The vectors $\overrightarrow{P\,}$ and $\overrightarrow{Q\,}$ will both be in $\Pi_1$ so a normal vector is $$\ul n = \begin{pmatrix}1\\ 2\\ 3\end{pmatrix} \times \begin{pmatrix}-2\\ 1\\ 5\end{pmatrix} = \begin{pmatrix}7\\ -11\\ 5\end{pmatrix}$$

Therefore any point on $\Pi_1$ will satisfy $$\begin{pmatrix}x\\ y\\ z\end{pmatrix} \cdot \begin{pmatrix}7\\ -11\\ 5\end{pmatrix} = 0$$
So we get the equation $7x - 11y + 5z = 0$.

Note that the point $R = \begin{pmatrix}1\\ 0\\ 0\end{pmatrix}$ does not satisfy this equation and thus is not on $\Pi_1$. So we can define $\Pi_2$ as the plane containing $P$, $Q$, and $R$.

Two vectors in $\Pi_2$ are $\overrightarrow{RP\,} = \begin{pmatrix}0\\ 2\\ 3\end{pmatrix}$ and $\overrightarrow{RQ\,} = \begin{pmatrix}-3\\ 1\\ 5\end{pmatrix}$. Then we can find a normal vector $$\ul n = \begin{pmatrix}0\\ 2\\ 3\end{pmatrix} \times \begin{pmatrix}-3\\ 1\\ 5\end{pmatrix} = \begin{pmatrix}7\\ -9\\ 6\end{pmatrix}$$

$R$ is a point on $\Pi_2$ and $\overrightarrow{R\,} \cdot \ul n = 7$. Therefore any point on $\Pi_2$ will satisfy $$\begin{pmatrix}x\\ y\\ z\end{pmatrix} \cdot \begin{pmatrix}7\\ -9\\ 6\end{pmatrix} = 7$$
So we get the equation $7x - 9y + 6z = 7$.

Therefore the line $L$ can be described by the pair of equations \begin{align*}
7x - 11y + 5z &= 0\\
7x - 9y - 6z &= 7
\end{align*}

We can check and indeed, $P$ and $Q$ both satisfy both of these equations.
% }}}

% {{{ Q6
\question{6}

\begin{questionbody}
Compute the RREF of $$A = \begin{pmatrix}
1 & -3 & 0 & 1\\
0 & 0 & 1 & 2\\
-2 & 6 & 1 & 0
\end{pmatrix}$$
\end{questionbody}

\begin{align*}
A_{13}(2) \implies &\begin{pmatrix}
	1 & -3 & 0 & 1\\
	0 & 0 & 1 & 2\\
	0 & 0 & 1 & 2
\end{pmatrix}\\[1ex]
A_{23}(-1) \implies &\begin{pmatrix}
	1 & -3 & 0 & 1\\
	0 & 0 & 1 & 2\\
	0 & 0 & 0 & 0
\end{pmatrix}
\end{align*}
% }}}

% {{{ Q7
\newquestion{7}

\begin{questionbody}
For each of the following matrices $A$, determine whether $A$ in invertible, and if it is, compute $A^{-1}$.
\end{questionbody}

\subsection{~} % 7.a

\begin{questionbody}
$$A = \begin{pmatrix}1 & 2 & 3\\ 4 & 5 & 6\end{pmatrix}$$
\end{questionbody}

Only square matrices can be invertible, so $A$ is not invertible.

\subsection{~} % 7.b

\begin{questionbody}
$$A = \begin{pmatrix}2 & 4\\ 6 & 8\end{pmatrix}$$
\end{questionbody}

\begin{align*}
(A\,|\,I) = &\begin{apmatrix}{2}{2}
	2 & 4 & 1 & 0\\[0.8ex]
	6 & 8 & 0 & 1
\end{apmatrix}\\[1ex]
M_1\l(\f12\r) \implies &\begin{apmatrix}{2}{2}
	1 & 2 & \f12 & 0\\[0.8ex]
	6 & 8 & 0 & 1
\end{apmatrix}\\[1ex]
A_{12} \l(-6\r) \implies &\begin{apmatrix}{2}{2}
	1 & 2 & \f12 & 0\\[0.8ex]
	0 & -4 & -3 & 1
\end{apmatrix}\\[1ex]
M_2 \l(-\f14\r) \implies &\begin{apmatrix}{2}{2}
	1 & 2 & \f12 & 0\\[0.8ex]
	0 & 1 & \f34 & -\f14
\end{apmatrix}\\[1ex]
A_{21} \l(-2\r) \implies &\begin{apmatrix}{2}{2}
	1 & 0 & -1 & \f12\\[0.8ex]
	0 & 1 & \f34 & -\f14
\end{apmatrix}
\end{align*}
Therefore $A^{-1} = A_{21}(-2)\, M_2\l(-\f14\r)\, A_{12}(-6)\, M_1\l(\f12\r) = \begin{pmatrix}-1 & \f12\\[0.8ex] \f34 & -\f14\end{pmatrix}$.

\subsection{~} % 7.c

\begin{questionbody}
$$A = \begin{pmatrix}
1 & -2 & 0\\
2 & -3 & 0\\
0 & 0 & 1
\end{pmatrix}$$
\end{questionbody}

\begin{align*}
(A\,|\,I) = &\begin{apmatrix}{3}{3}
	1 & -2 & 0 & 1 & 0 & 0\\
	2 & -3 & 0 & 0 & 1 & 0\\
	0 & 0 & 1 & 0 & 0 & 1
\end{apmatrix}\\[1ex]
A_{12}(-2) \implies &\begin{apmatrix}{3}{3}
	1 & -2 & 0 & 1 & 0 & 0\\
	0 & 1 & 0 & -2 & 1 & 0\\
	0 & 0 & 1 & 0 & 0 & 1
\end{apmatrix}\\[1ex]
A_{21}(2) \implies &\begin{apmatrix}{3}{3}
	1 & 0 & 0 & -3 & 2 & 0\\
	0 & 1 & 0 & -2 & 1 & 0\\
	0 & 0 & 1 & 0 & 0 & 1
\end{apmatrix}
\end{align*}
Therefore $A^{-1} = A_{21}(2)\, A_{12}(-2) = \begin{pmatrix}-3 & 2 & 0\\ -2 & 1 & 0\\ 0 & 0 & 1\end{pmatrix}$.
% }}}

% {{{ Q8
\question{8}

\begin{questionbody}
Consider the system of equations \[
\arraycolsep=2pt
\begin{array}{rcrcrcrcrcrcr}
      & & 6x_2 &+& 2x_3 & &      &-& x_5  & &     &=& 1\\
      & &      & & 4x_3 & &      &+& x_5  & &     &=& -1\\
-2x_1 &+& x_2  & &      &+& 4x_4 &+& x_5  &-& x_6 &=& 0\\
-3x_1 & &      &+& x_3  &+& 6x_4 &+& 2x_5 & &     &=& 4
\end{array}\]
\end{questionbody}

\subsection{~} % 8.a

\begin{questionbody}
Write down the augmented matrix for this system.
\end{questionbody}

$$\begin{apmatrix}{6}{1}
0 & 6 & 2 & 0 & -1 & 0 & 1\\
0 & 0 & 4 & 0 & 1 & 0 & -1\\
-2 & 1 & 0 & 4 & 1 & -1 & 0\\
-3 & 0 & 1 & 6 & 2 & 0 & 4
\end{apmatrix}$$

% {{{ 8.b, RREF
\newpage
\subsection{~}

\begin{questionbody}
Compute the RREF of the augmented matrix.
\end{questionbody}

% NOTE: Be careful with formatting here. An align* environment won't break across pages, so we need to break manually in a way that looks good.
\begin{align*}
&\begin{apmatrix}{6}{1}
	0 & 6 & 2 & 0 & -1 & 0 & 1\\[0.8ex]
	0 & 0 & 4 & 0 & 1 & 0 & -1\\[0.8ex]
	-2 & 1 & 0 & 4 & 1 & -1 & 0\\[0.8ex]
	-3 & 0 & 1 & 6 & 2 & 0 & 4
\end{apmatrix}\\[1ex]
S_{13} \implies &\begin{apmatrix}{6}{1}
	-2 & 1 & 0 & 4 & 1 & -1 & 0\\[0.8ex]
	0 & 0 & 4 & 0 & 1 & 0 & -1\\[0.8ex]
	0 & 6 & 2 & 0 & -1 & 0 & 1\\[0.8ex]
	-3 & 0 & 1 & 6 & 2 & 0 & 4
\end{apmatrix}\\[1ex]
M_1\l(-\f12\r) \implies &\begin{apmatrix}{6}{1}
	1 & -\f12 & 0 & -2 & -\f12 & \f12 & 0\\[0.8ex]
	0 & 0 & 4 & 0 & 1 & 0 & -1\\[0.8ex]
	0 & 6 & 2 & 0 & -1 & 0 & 1\\[0.8ex]
	-3 & 0 & 1 & 6 & 2 & 0 & 4
\end{apmatrix}\\[1ex]
A_{14}(3) \implies &\begin{apmatrix}{6}{1}
	1 & -\f12 & 0 & -2 & -\f12 & \f12 & 0\\[0.8ex]
	0 & 0 & 4 & 0 & 1 & 0 & -1\\[0.8ex]
	0 & 6 & 2 & 0 & -1 & 0 & 1\\[0.8ex]
	0 & -\f32 & 1 & 0 & \f12 & \f32 & 4
\end{apmatrix}\\[1ex]
S_{24} \implies &\begin{apmatrix}{6}{1}
	1 & -\f12 & 0 & -2 & -\f12 & \f12 & 0\\[0.8ex]
	0 & -\f32 & 1 & 0 & \f12 & \f32 & 4\\[0.8ex]
	0 & 6 & 2 & 0 & -1 & 0 & 1\\[0.8ex]
	0 & 0 & 4 & 0 & 1 & 0 & -1
\end{apmatrix}\\[1ex]
M_2\l(-\f23\r) \implies &\begin{apmatrix}{6}{1}
	1 & -\f12 & 0 & -2 & -\f12 & \f12 & 0\\[0.8ex]
	0 & 1 & -\f23 & 0 & -\f13 & -1 & -\f83\\[0.8ex]
	0 & 6 & 2 & 0 & -1 & 0 & 1\\[0.8ex]
	0 & 0 & 4 & 0 & 1 & 0 & -1
\end{apmatrix}\\[1ex]
A_{21}\l(\f12\r) \implies &\begin{apmatrix}{6}{1}
	1 & 0 & -\f13 & -2 & -\f23 & 0 & -\f43\\[0.8ex]
	0 & 1 & -\f23 & 0 & -\f13 & -1 & -\f83\\[0.8ex]
	0 & 6 & 2 & 0 & -1 & 0 & 1\\[0.8ex]
	0 & 0 & 4 & 0 & 1 & 0 & -1
\end{apmatrix}\\[1ex]
A_{23}\l(-6\r) \implies &\begin{apmatrix}{6}{1}
	1 & 0 & -\f13 & -2 & -\f23 & 0 & -\f43\\[0.8ex]
	0 & 1 & -\f23 & 0 & -\f13 & -1 & -\f83\\[0.8ex]
	0 & 0 & 6 & 0 & 1 & 6 & 17\\[0.8ex]
	0 & 0 & 4 & 0 & 1 & 0 & -1
\end{apmatrix}
\end{align*}
\begin{align*}
M_3\l(\f16\r) \implies &\begin{apmatrix}{6}{1}
	1 & 0 & -\f13 & -2 & -\f23 & 0 & -\f43\\[0.8ex]
	0 & 1 & -\f23 & 0 & -\f13 & -1 & -\f83\\[0.8ex]
	0 & 0 & 1 & 0 & \f16 & 1 & \f{17}6\\[0.8ex]
	0 & 0 & 4 & 0 & 1 & 0 & -1
\end{apmatrix}\\[1ex]
A_{34}\l(-4\r) \implies &\begin{apmatrix}{6}{1}
	1 & 0 & -\f13 & -2 & -\f23 & 0 & -\f43\\[0.8ex]
	0 & 1 & -\f23 & 0 & -\f13 & -1 & -\f83\\[0.8ex]
	0 & 0 & 1 & 0 & \f16 & 1 & \f{17}6\\[0.8ex]
	0 & 0 & 0 & 0 & \f13 & -4 & -\f{37}3
\end{apmatrix}\\[1ex]
A_{31}\l(\f13\r) \implies &\begin{apmatrix}{6}{1}
	1 & 0 & 0 & -2 & -\f{11}{18} & \f13 & -\f7{18}\\[0.8ex]
	0 & 1 & -\f23 & 0 & -\f13 & -1 & -\f83\\[0.8ex]
	0 & 0 & 1 & 0 & \f16 & 1 & \f{17}6\\[0.8ex]
	0 & 0 & 0 & 0 & \f13 & -4 & -\f{37}3
\end{apmatrix}\\[1ex]
A_{32}\l(\f23\r) \implies &\begin{apmatrix}{6}{1}
	1 & 0 & 0 & -2 & -\f{11}{18} & \f13 & -\f7{18}\\[0.8ex]
	0 & 1 & 0 & 0 & -\f29 & -\f13 & -\f79\\[0.8ex]
	0 & 0 & 1 & 0 & \f16 & 1 & \f{17}6\\[0.8ex]
	0 & 0 & 0 & 0 & \f13 & -4 & -\f{37}3
\end{apmatrix}\\[1ex]
M_4(3) \implies &\begin{apmatrix}{6}{1}
	1 & 0 & 0 & -2 & -\f{11}{18} & \f13 & -\f7{18}\\[0.8ex]
	0 & 1 & 0 & 0 & -\f29 & -\f13 & -\f79\\[0.8ex]
	0 & 0 & 1 & 0 & \f16 & 1 & \f{17}6\\[0.8ex]
	0 & 0 & 0 & 0 & 1 & -12 & -37
\end{apmatrix}\\[1ex]
A_{43}\l(-\f16\r) \implies &\begin{apmatrix}{6}{1}
	1 & 0 & 0 & -2 & -\f{11}{18} & \f13 & -\f7{18}\\[0.8ex]
	0 & 1 & 0 & 0 & -\f29 & -\f13 & -\f79\\[0.8ex]
	0 & 0 & 1 & 0 & 0 & 3 & 9\\[0.8ex]
	0 & 0 & 0 & 0 & 1 & -12 & -37
\end{apmatrix}\\[1ex]
A_{42}\l(\f29\r) \implies &\begin{apmatrix}{6}{1}
	1 & 0 & 0 & -2 & -\f{11}{18} & \f13 & -\f7{18}\\[0.8ex]
	0 & 1 & 0 & 0 & 0 & -3 & -9\\[0.8ex]
	0 & 0 & 1 & 0 & 0 & 3 & 9\\[0.8ex]
	0 & 0 & 0 & 0 & 1 & -12 & -37
\end{apmatrix}\\[1ex]
A_{41}\l(\f{11}{18}\r) \implies &\begin{apmatrix}{6}{1}
	1 & 0 & 0 & -2 & 0 & -7 & -23\\[0.8ex]
	0 & 1 & 0 & 0 & 0 & -3 & -9\\[0.8ex]
	0 & 0 & 1 & 0 & 0 & 3 & 9\\[0.8ex]
	0 & 0 & 0 & 0 & 1 & -12 & -37
\end{apmatrix}
\end{align*}
% }}}

\newpage
\subsection{~} % 8.c

\begin{questionbody}
In terms of this RREF (without actually identifying a solution) explain why the
system is consistent.
\end{questionbody}

The system is consistent because the row reduced echelon form has no zero rows.

\subsection{~} % 8.d

\begin{questionbody}
Give the general solution of this system of equations.
\end{questionbody}

Choose parameters $\lambda, \mu \in \R$. Then \begin{align*}
x_1 &= -23 + 2\lambda + 7\mu\\
x_2 &= -9 + 3\mu\\
x_3 &= 9 - 3\mu\\
x_4 &= \lambda\\
x_5 &= -37 + 12\mu\\
x_6 &= \mu
\end{align*}
% }}}

\end{document}
