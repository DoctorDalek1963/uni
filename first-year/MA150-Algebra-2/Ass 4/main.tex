% vim: set foldmethod=marker foldlevel=0:

\documentclass[a4paper]{article}
\usepackage[UKenglish]{babel}

\usepackage{preamble}

\fancyhead[L]{MA150 Assignment 4}
\title{MA150 Algebra 2, Assignment 4}
\colorlet{questionbodycolor}{magenta!50}

\begin{document}

\maketitle

\setlength{\parindent}{0em}
\setlength{\parskip}{1em}

% {{{ Q4
\question{4}

\begin{questionbody}
Let $V$ be a Euclidean space with inner product $\angb{\cdot, \cdot}$. Suppose $\enum w1n$ is an orthonormal basis of $V$. Of course $\fa v \in V$, $\exists\ \lambda_i \in \R$ such that $v = \sum \lambda_i w_i$. Show that in fact $\lambda_i = \angb{v, w_i}$.
\end{questionbody}

For any fixed $i \le n$ \begin{align*}
v &= \smlm_{j=1}^n \lambda_j w_j\\[1ex]
\angb{v, w_i} &= \angb{\smlm_{j=1}^n \lambda_j w_j, w_i}\\[1ex]
&= \smlm_{j=1}^n \lambda_j \angb{w_j, w_i}\\[1ex]
&= \lambda_1 \angb{w_1, w_i} + \cdots + \lambda_i \angb{w_i, w_i} + \cdots + \lambda_n \angb{w_n, w_i}\\[1ex]
&= \lambda_i
\end{align*}
Since all $\enum w1n$ are orthonormal, so all but one of the inner products are zero.

Therefore $\angb{v, w_i} = \lambda_i$ for all $i = 1,\ \ldots,\ n$, as required.
% }}}

% {{{ Q5
\newquestion{5}

\begin{questionbody}
Let $V = \R^3$ equipped with the usual inner (dot) product. Let $$v_1 = \begin{pmatrix}1\\ 1\\ 0\end{pmatrix}, \quad v_2 = \begin{pmatrix}2\\ 0\\ 1\end{pmatrix}, \quad v_3 = \begin{pmatrix}1\\ 1\\ 2\end{pmatrix}$$
\end{questionbody}

\subsection{~} % 5.a

\begin{questionbody}
Apply the Gram-Schmidt orthogonalisation process to $v_1, v_2, v_3$ to construct an orthonormal basis $w_1, w_2, w_3$.
\end{questionbody}

\begin{align*}
w_1 &= \f{v_1}{\| v_1 \|} = \f1{\sqrt 2} \begin{pmatrix}1\\ 1\\ 0\end{pmatrix}\\[1ex]
u_2 &= v_2 - (v_2 \cdot w_1) w_1\\[1ex]
&= \begin{pmatrix}2\\ 0\\ 1\end{pmatrix} - \f2{\sqrt2} \f1{\sqrt2} \begin{pmatrix}1\\ 1\\ 0\end{pmatrix} = \begin{pmatrix}1\\ -1\\ 0\end{pmatrix}\\[1ex]
w_2 &= \f{u_2}{\| u_2 \|} = \f1{\sqrt2} \begin{pmatrix}1\\ -1\\ 0\end{pmatrix}\\[1ex]
u_3 &= v_3 - (v_3 \cdot w_1) w_1 - (v_3 \cdot w_2) w_2\\[1ex]
&= \begin{pmatrix}1\\ 1\\ 2\end{pmatrix} - \f2{\sqrt2} \f1{\sqrt2} \begin{pmatrix}1\\ 1\\ 0\end{pmatrix} - 0 \begin{pmatrix}1\\ -1\\ 0\end{pmatrix} = \begin{pmatrix}0\\ 0\\ 2\end{pmatrix}\\[1ex]
w_3 &= \f{u_3}{\| u_3 \|} = \begin{pmatrix}0\\ 0\\ 1\end{pmatrix}
\end{align*}

Therefore our orthonormal basis $w_1, w_2, w_3$ is $\ds \f1{\sqrt2} \begin{pmatrix}1\\ 1\\ 0\end{pmatrix}, \f1{\sqrt2} \begin{pmatrix}1\\ -1\\ 1\end{pmatrix}, \begin{pmatrix}0\\ 0\\ 1\end{pmatrix}$

\newpage
\subsection{~} % 5.b

\begin{questionbody}
Consider $v = \begin{pmatrix}0\\ 0\\ 1\end{pmatrix}$. Find $\lambda_1, \lambda_2, \lambda_3$ such that $v = \lambda_1 w_1 + \lambda_2 w_2 + \lambda_3 w_3$.
\end{questionbody}

Clearly $v = w_3$ so if $v = \lambda_1 w_1 + \lambda_2 w_2 + \lambda_3 w_3$ then, since $w_1, w_2, w_3$ are linearly independent, $\lambda_1 = \lambda_2 = 0$ and $\lambda_3 = 1$.

% }}}

% {{{ Q6
\newquestion{6}

\begin{questionbody}
Consider the symmetric matrix $$A = \begin{pmatrix} a & \f b2 \\ \f b2 & c \end{pmatrix}$$
and define a function \begin{align*}
\varphi : \R^2 &\to \R\\[1ex]
v = \begin{pmatrix} x \\ y \end{pmatrix} &\mapsto v^T A v = \begin{pmatrix} x & y \end{pmatrix} \begin{pmatrix} a & \f b2 \\ \f b2 & c \end{pmatrix} \begin{pmatrix} x \\ y \end{pmatrix}
\end{align*}

Notice that $\varphi(\ul 0) = 0$. This question determines precise conditions for which $\varphi(v) = 0$ for all $v \ne \ul 0$.

Notice that $\varphi(\lambda v) = \lambda^2 \varphi(v)$ so we only need to consider $v \in \R^2$ of the form $$e_1 = \begin{pmatrix} 1 \\ 0 \end{pmatrix} \quad\text{and}\quad v_x = \begin{pmatrix} x \\ 1 \end{pmatrix}\ \fa x \in \R$$
\end{questionbody}

\subsection{~} % 6.a

\begin{questionbody}
If $a \le 0$ find a vector $v \ne \ul 0$ with $\varphi(v) \le \ul 0$.
\end{questionbody}

$\varphi(e_1) = \begin{pmatrix}1 & 0\end{pmatrix} \begin{pmatrix}a \\ \f b2\end{pmatrix} = a$ so if $a \le 0$ then $\varphi(e_1) \le 0$.

\subsection{~} % 6.b

\begin{questionbody}
Express $\varphi(v_x)$ as a polynomial in $x$.
\end{questionbody}

\begin{align*}
\varphi(v_x) &= \begin{pmatrix}x & 1\end{pmatrix} \begin{pmatrix}a & \f b2 \\ \f b2 & c\end{pmatrix} \begin{pmatrix}x \\ 1\end{pmatrix}\\[1ex]
&= \begin{pmatrix}x & 1\end{pmatrix} \begin{pmatrix}ax + \f b2 \\ \f b2 x + c\end{pmatrix}\\[1ex]
&= \l(ax + \f b2\r)x + \f b2 x + c\\[1ex]
&= ax^2 + bx + c
\end{align*}

\clearpage
\subsection{~} % 6.c

\begin{questionbody}
Suppose $a > 0$. Prove that $\varphi(v_x) > 0$ for all $x \in \R$ if and only if $b^4 - 4ac < 0$. Recall that $\det A = ac - \f{b^2}4$, so that this condition is exactly the same as $\det A > 0$.
\end{questionbody}

Suppose $a > 0$ and $b^2 - 4ac < 0$. That means $ax^2 + bx + c = 0$ has no roots and since $a > 0$, $ax^2 + bx + c > 0$, so $\varphi(v_x) > 0\ \fa x \in \R$.

Conversely, suppose $\varphi(v_x) > 0$. Therefore $ax^2 + bx + c > 0\ \fa x \in \R$, which means this quadratic has no roots and therefore has a negative discriminant, so $b^2 - 4ac < 0$.

% }}}

% {{{ Q7
\newquestion{7}

\begin{questionbody}
Continue with the notation and matrix $A$ from the previous question. For any vectors $v, w \in \R^2$ define $\angb{v,w} = v^T A w$.
\end{questionbody}

\subsection{~} % 7.a

\begin{questionbody}
Show that $\angb{v, w} = \angb{w, v}$ for any $v, w \in \R^2$.
\end{questionbody}

The transpose of a scalar is the same scalar, so \begin{align*}
\quad\qquad\qquad \angb{v, w} &= \angb{v, w}^T\\
&= \l( v^T A w \r)^T\\
&= w^T A^T \l( v^T \r)^T\\
&= w^T A^T v\\
&= w^T A v \qqt{since }A^T = A\\
&= \angb{w, v}
\end{align*}

\subsection{~} % 7.b

\begin{questionbody}
Show that for any $v_1, v_2, w \in \R^2$ and $\lambda_1, \lambda_2 \in \R$,
$$\angb{\lambda_1 v_1 + \lambda_2 v_2, w} = \lambda_1 \angb{v_1, w} + \lambda_2 \angb{v_2, w}$$
\end{questionbody}

\begin{align*}
\angb{\lambda_1 v_1 + \lambda_2 v_2, w} &= \l(\lambda_1 v_1 + \lambda_2 v_2\r)^T A w\\[1ex]
&= \l(\lambda_1 v_1^T + \lambda_2 v_2^T\r) A w\\[1ex]
&= \lambda_1 v_1^T A w + \lambda_2 v_2^T A w\\[1ex]
&= \lambda_1 \angb{v_1, w} + \lambda_2 \angb{v_2, w}
\end{align*}

\subsection{~} % 7.c

\begin{questionbody}
Suppose $a > 0$ and $\det A > 0$. Show that $\angb{v, v} \ge 0$ for any $v \in V$, and that $\angb{v, v} = 0$ if and only if $v = \ul 0$.
\end{questionbody}

Since $a > 0$ and $\det A > 0$, $b^4 - 4ac < 0$. We know that $\angb{v, v} = v^T A v$ so $\angb{v, v}$ is equivalent to $\varphi(v)$ from Question~6, and we know that $\varphi(v) \ge 0\ \fa v \in \R^2$ when $a > 0$ and $\det A > 0$, with equality only when $v=0$.

Essentially, the desired result follows immediately from Question~6 and the observation that $\angb{v, v} = \varphi(v)$.

\subsection{~} % 7.d

\begin{questionbody}
Which of the following matrices $A$ determine an inner product on $V = \R^2$ by the formula $v^T A w$ above?
$$
\begin{pmatrix}-2 & 1\\ 1 & -2\end{pmatrix}\hspace{1em}
\begin{pmatrix}5 & -3\\ -3 & 2\end{pmatrix}\hspace{1em}
\begin{pmatrix}5 & 3\\ 3 & -2\end{pmatrix}\hspace{1em}
\begin{pmatrix}2 & -1\\ -1 & 1\end{pmatrix}\hspace{1em}
\begin{pmatrix}1 & -1\\ -1 & 1\end{pmatrix}
$$
\end{questionbody}

For a matrix $A$ to determine an inner product as above, we need $a > 0$ and $\det A > 0$, so only $\begin{pmatrix}5 & -3\\ -3 & 2\end{pmatrix}$ and $\begin{pmatrix}2 & -1\\ -1 & 1\end{pmatrix}$ determine inner products

% }}}

\end{document}
