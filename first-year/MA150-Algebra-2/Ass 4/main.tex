% vim: set foldmethod=marker foldlevel=0:

\documentclass[a4paper]{article}
\usepackage[UKenglish]{babel}

\usepackage{preamble}

\fancyhead[L]{MA150 Assignment 4}
\title{MA150 Algebra 2, Assignment 4}

\begin{document}

\maketitle

\setlength{\parindent}{0em}
\setlength{\parskip}{1em}

% {{{ Q4
\question{4}

Let $V$ be a Euclidean space with inner product. Suppose $\enum w1n$ is an orthonormal basis of $V$. Of course $\fa v \in V$, $\exists\ \lambda_i \in \R$ such that $v = \sum \lambda_i w_i$.

We want to show that $\lambda_i = \angb{v, w_i}$. For any fixed $i \le n$ \begin{align*}
v &= \smlm_{j=1}^n \lambda_j w_j\\[1ex]
\angb{v, w_i} &= \angb{\smlm_{j=1}^n \lambda_j w_j, w_i}\\[1ex]
&= \smlm_{j=1}^n \lambda_j \angb{w_j, w_i}\\[1ex]
&= \lambda_1 \angb{w_1, w_i} + \cdots + \lambda_i \angb{w_i, w_i} + \cdots + \lambda_n \angb{w_n, w_i}\\[1ex]
&= \lambda_i
\end{align*}
Since all $\enum w1n$ are orthonormal, so all but one of the inner products are zero.

Therefore $\angb{v, w_i} = \lambda_i$ for all $i = 1,\ \ldots,\ n$, as required.
% }}}

% {{{ Q5
\newquestion{5}

Let $V = \R^3$ and let $$v_1 = \begin{pmatrix}1\\ 1\\ 0\end{pmatrix}, \quad v_2 = \begin{pmatrix}2\\ 0\\ 1\end{pmatrix}, \quad v_3 = \begin{pmatrix}1\\ 1\\ 2\end{pmatrix}$$

\subsection{~}

We will apply Gram-Schmidt to $v_1, v_2, v_3$ to get an orthonormal basis $w_1, w_2, w_3$.

\begin{align*}
w_1 &= \f{v_1}{\| v_1 \|} = \f1{\sqrt 2} \begin{pmatrix}1\\ 1\\ 0\end{pmatrix}\\[1ex]
u_2 &= v_2 - (v_2 \cdot w_1) w_1\\[1ex]
&= \begin{pmatrix}2\\ 0\\ 1\end{pmatrix} - \f2{\sqrt2} \f1{\sqrt2} \begin{pmatrix}1\\ 1\\ 0\end{pmatrix} = \begin{pmatrix}1\\ -1\\ 0\end{pmatrix}\\[1ex]
w_2 &= \f{u_2}{\| u_2 \|} = \f1{\sqrt2} \begin{pmatrix}1\\ -1\\ 0\end{pmatrix}\\[1ex]
u_3 &= v_3 - (v_3 \cdot w_1) w_1 - (v_3 \cdot w_2) w_2\\[1ex]
&= \begin{pmatrix}1\\ 1\\ 2\end{pmatrix} - \f2{\sqrt2} \f1{\sqrt2} \begin{pmatrix}1\\ 1\\ 0\end{pmatrix} - 0 \begin{pmatrix}1\\ -1\\ 0\end{pmatrix} = \begin{pmatrix}0\\ 0\\ 2\end{pmatrix}\\[1ex]
w_3 &= \f{u_3}{\| u_3 \|} = \begin{pmatrix}0\\ 0\\ 1\end{pmatrix}
\end{align*}

Therefore our orthonormal basis $w_1, w_2, w_3$ is $\ds \f1{\sqrt2} \begin{pmatrix}1\\ 1\\ 0\end{pmatrix}, \f1{\sqrt2} \begin{pmatrix}1\\ -1\\ 1\end{pmatrix}, \begin{pmatrix}0\\ 0\\ 1\end{pmatrix}$

\subsection{~}

Consider $v = \begin{pmatrix}0\\ 0\\ 1\end{pmatrix}$. Clearly $v = w_3$ so if $v = \lambda_1 w_1 + \lambda_2 w_2 + \lambda_3 w_3$ then $\lambda_1 = \lambda_2 = 0$ and $\lambda_3 = 1$.

% }}}

% {{{ Q6
\newquestion{6}

Let $$A = \begin{pmatrix} a & \f b2 \\ \f b2 & c \end{pmatrix}$$
And define \begin{align*}
\varphi : \R^2 &\to \R\\[1ex]
v = \begin{pmatrix} x \\ y \end{pmatrix} &\mapsto v^T A v = \begin{pmatrix} x & y \end{pmatrix} \begin{pmatrix} a & \f b2 \\ \f b2 & c \end{pmatrix} \begin{pmatrix} x \\ y \end{pmatrix}
\end{align*}

Note that $\varphi(\ul 0) = 0$ and $\varphi(\lambda v) = \lambda^2 \varphi(v)$ so we only need to consider $v \in \R^2$ of the form $$e_1 = \begin{pmatrix} 1 \\ 0 \end{pmatrix} \quad\text{and}\quad v_x = \begin{pmatrix} x \\ a \end{pmatrix}\ \fa x \in \R$$

\subsection{~}
\subsection{~}
\subsection{~}

% }}}

% {{{ Q7
\newquestion{7}

\subsection{~}
\subsection{~}
\subsection{~}
\subsection{~}

% }}}

\end{document}
