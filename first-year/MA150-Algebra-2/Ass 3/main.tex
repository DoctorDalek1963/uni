% vim: set foldmethod=marker foldlevel=0:

\documentclass[a4paper]{article}
\usepackage[UKenglish]{babel}

\usepackage{preamble}

\fancyhead[L]{MA150 Assignment 3}
\title{MA150 Algebra 2, Assignment 3}
\colorlet{questionbodycolor}{magenta!50}

\begin{document}

\maketitle

\setlength{\parindent}{0em}
\setlength{\parskip}{1em}

% {{{ Q6
\question{6}

\begin{questionbody}
\begin{equation}
W = (x + 2y - 3z = 0) \subset \R^3
\label{eqn:Q6-W}
\end{equation}
\end{questionbody}

\subsection{~} % 6.a

\begin{questionbody}
Show that $W \ne \R^3$, and explain why that implies that $\dim W < 3$.
\end{questionbody}

The vector $\begin{pmatrix}1 \\ 1 \\ -1\end{pmatrix}$ is not in $W$ since it doesn't satisfy the equation. In particular, $1(1) + 2(1) - 3(-1) = 6 \ne 0$. Therefore $W \ne \R^3$.

We know from lectures that the dimension of a subspace is less than or equal to the dimension of the parent space, and they have the same dimension if and only if they are equal. Since $W \subset \R^3$, $\dim W \le \dim \R^3$. The dimension of $\R^3$ is 3 (since the standard basis of $\R^3$ has 3 elements). Therefore $\dim W \le 3$. But $W \ne \R^3$, so $\dim W < 3$.

\subsection{~} % 6.b

\begin{questionbody}
Find a basis of $W$ and find $\dim W$.
\end{questionbody}

We can rearrange equation~(\ref{eqn:Q6-W}) to get $x = 3z - 2y$. Then we can introduce parameters $\lambda$ and $\mu$ and conclude that any point in $W$ can be written as \[
\begin{pmatrix}3 \mu - 2 \lambda \\ \lambda \\ \mu\end{pmatrix}
= \lambda \begin{pmatrix}-2 \\ 1 \\ 0\end{pmatrix}
+ \mu \begin{pmatrix}3 \\ 0 \\ 1\end{pmatrix}
\] Therefore $\l\{ \begin{pmatrix}- 2 \\ 1 \\ 0\end{pmatrix}, \begin{pmatrix}3 \\ 0 \\ 1\end{pmatrix} \r\}$ is a basis of $W$.

Call the elements of this basis $\{ w_1, w_2 \}$ for convenience. Plugging $w_1$ into equation~(\ref{eqn:Q6-W}) gives $1(-2) + 2(1)  - 3(0) = 0$ as required, and plugging $w_2$ into equation~(\ref{eqn:Q6-W}) gives $1(3) + 2(0) - 3(1) = 0$ as required. Therefore $w_1, w_2 \in W$.

For $w_1$ and $w_2$ to be independent, we need to show that $\lambda w_1 + \mu w_2 = 0_W$ if and only if $\lambda = \mu = 0$. That linear independence equation expands to \[
\begin{pmatrix}3 \mu - 2 \lambda \\ \lambda \\ \mu\end{pmatrix}
= \begin{pmatrix}0 \\ 0 \\ 0\end{pmatrix}
\] The second components of the vectors imply $\lambda = 0$, and the third components imply $\mu = 0$. Therefore $w_1$ and $w_2$ are linearly independent.

$w_1$ and $w_2$ must span $W$ since any linear combination is of the form $\begin{pmatrix}3 \mu - 2 \lambda \\ \lambda \\ \mu\end{pmatrix}$ and we showed before that that is equivalent to equation~(\ref{eqn:Q6-W}), which is the definition of $W$.

Since we have a basis of $W$ with 2 elements, we know that $\dim W = 2$.

% }}}

% {{{ Q7
\newquestion{7}

\begin{questionbody}
Let $V = {\R[x]}_{\le 3}$ be the vector space of polynomials in $x$ of degree at most $3$, and let $W = \R^2$. Consider the linear map $\varphi \colon V \to W$ determined on the basis $1, x, x^2, x^3$ by
\[
\varphi(1) = \begin{pmatrix}1 \\ 0\end{pmatrix},
\quad \varphi(x) = \begin{pmatrix}-1 \\ 1\end{pmatrix},
\quad \varphi(x^2) = \begin{pmatrix}1 \\ -2\end{pmatrix},
\quad \varphi(x^3) = \begin{pmatrix}-1 \\ 3\end{pmatrix}
\]
\end{questionbody}

\subsection{~} % 7.a

\begin{questionbody}
Compute $\varphi(2x^3 - 3x + 2)$.
\end{questionbody}

\begin{align*}
\varphi(2x^3 - 3x + 2) &= \varphi(2x^3) + \varphi(-3x) + \varphi(2) \\[1ex]
&= 2 \varphi(x^3) - 3 \varphi(x) + 2 \varphi(1) \\[1ex]
&= 2 \begin{pmatrix}-1 \\ 3\end{pmatrix} - 3 \begin{pmatrix}-1 \\ 1\end{pmatrix} + 2 \begin{pmatrix}1 \\ 0\end{pmatrix} \\[1ex]
&= \begin{pmatrix}-2 \\ 6\end{pmatrix} + \begin{pmatrix}3 \\ -3\end{pmatrix} + \begin{pmatrix}2 \\ 0\end{pmatrix} \\[1ex]
&= \begin{pmatrix}3 \\ 3\end{pmatrix}
\end{align*}

\subsection{~} % 7.b

\begin{questionbody}
Consider the linear map $\psi \colon V \to W$ where \[
\psi = \begin{pmatrix} f(-1) \\[1ex] \ds \dd fx (-1) \end{pmatrix}
\] Show that $\psi = \varphi$.
\end{questionbody}

By proposition 5.17, two linear maps are equal if their domains and codomains are equal and they agree on the elements of a basis of the domain. $\varphi$ and $\psi$ are both defined on $\varphi, \psi \colon {\R[x]}_{\le 3} \to \R^2$. Then we just have to check that $\varphi$ and $\psi$ agree on some basis of the domain, and it makes sense to use $\{ 1, x, x^2, x^3 \}$.

\begin{gather*}
\psi(1) = \begin{pmatrix}1 \\ 0\end{pmatrix}, \quad
\psi(x) = \begin{pmatrix}-1 \\ 1\end{pmatrix}, \quad \\[1ex]
\psi(x^2) = \begin{pmatrix}{(-1)}^2 \\ 2(-1)\end{pmatrix} = \begin{pmatrix}1 \\ -2\end{pmatrix}, \quad
\psi(x^3) = \begin{pmatrix}{(-1)}^3 \\ 3{(-1)}^2\end{pmatrix} = \begin{pmatrix}-1 \\ 3\end{pmatrix}
\end{gather*}

Since $\psi$ and $\varphi$ agree on a basis, $\psi = \varphi$.

\subsection{~} % 7.c

\begin{questionbody}
Compute $\Im \varphi$.
\end{questionbody}

To find the image of a linear transformation, we can write it as a matrix and take the column span of its row reduced echelon form. $\varphi$ is $L_M$ where \[ M = \begin{pmatrix} 1 & -1 & 1 & -1 \\ 0 & 1 & -2 & 3 \end{pmatrix} \]

Finding $\rref(M)$ only takes one step, $A_{21}(1)$. \[
\rref(M) = \begin{pmatrix} 1 & 0 & -1 & 2 \\ 0 & 1 & -2 & 3 \end{pmatrix}
\] Then $\op{Colspan}(\rref(M)) = \l\{ \begin{pmatrix}1 \\ 0\end{pmatrix}, \begin{pmatrix}0 \\ 1\end{pmatrix} \r\}$, so $\Im \varphi = \R^2$.

\subsection{~} % 7.d

\begin{questionbody}
Compute $\dim \ker \varphi$.
\end{questionbody}

$\varphi$ is defined on the domain $V = {\R[x]}_{\le 3}$, which has dimension 4. Also $\Im \varphi = \R^2$, so $\dim \Im \varphi = 2$. Therefore by the Rank-Nullity Theorem, \[
\dim \ker \varphi = \dim V - \dim \Im \varphi = 4 - 2 = 2
\]

% }}}

% {{{ Q8
\newquestion{8}

\begin{questionbody}
Let $V = {\R[x]}_{\le 2}$ be the vector space of polynomials in $x$ of degree at most $2$.
\end{questionbody}

\subsection{~} % 8.a

\begin{questionbody}
For any fixed $a \in \R$, prove that $x \mapsto x+a$ is an isomorphism $\pi \colon V \to V$. That is, $\pi$ is the linear map defined by $\pi(x^i) = {(x + a)}^i$ on the basis $1, x, x^2$ of $V$.
\end{questionbody}

An isomorphism of vector spaces is just a bijective linear map. We shall first prove that $\pi$ is a linear map.

We expect $\pi(\lambda x^i) = \lambda \pi(x^i)$.
\begin{align*}
\pi(\lambda x^i) &= \pi\l( {\l( \lambda^{\f1i} x \r)}^i \r) \\[1ex]
&= {\l( \lambda^{\f1i} x + a \r)}^i \\[1ex]
\end{align*}

% TODO: This question sucks and I don't understand it so I'm giving up

\subsection{~} % 8.b

\begin{questionbody}
Write the matrix of $\pi$ with respect to the basis $1, x, x^2$ of $V$.
\end{questionbody}

\[ L_\pi = \begin{pmatrix} 1 & a & a^2 \\ 0 & 1 & 2a \\ 0 & 0 & 1 \end{pmatrix} \]

% }}}

% {{{ Q9
\newquestion{9}

\begin{questionbody}
Consider $V = \l\{ f \colon \R \to \R \colon f \text{ is differentiable} \r\}$ which is a (very large) vector space under the usual operations $\lambda f + \mu g$.
\end{questionbody}

\subsection{~} % 9.a

\begin{questionbody}
Let $W = \l\langle \cos(x), \cos(2x) \r\rangle$ which is a subspace of $V$. What is $\dim W$?
\end{questionbody}

$\cos(x)$ and $\cos(2x)$ are linearly independent and span $W$ by definition, so $\{\cos(x), \cos(2x)\}$ is a basis for $W$. The dimension of a vector space is equal to the number of vectors in a basis, so $\dim W = 2$.

\subsection{~} % 9.b

\begin{questionbody}
Let $\cal U = \{ f \in W \colon f(10) = 0 \}$, which is a subspace of $W$. What is $\dim \cal U$?
\end{questionbody}

We want functions of the form $\lambda \cos(x) + \mu \cos(2x)$ for some $\lambda, \mu \in \R$ where $\lambda \cos(10) + \mu \cos(20) = 0$. That means we need \[ \lambda = \f{-\mu \cos(20)}{\cos(10)} \]

Therefore every element of $\cal U$ is of the form \[
\mu \l(\f{-\cos(20)}{\cos(10)} \cos(x) + \cos(2x) \r)
\] and therefore $\l\{ \df{-\cos(20)}{\cos(10)} \cos(x) + \cos(2x) \r\}$ is a basis of $\cal U$. Since this basis has 1 element, $\dim \cal U = 1$.

\subsection{~} % 9.c

\begin{questionbody}
Let $\cal U_2 = \{ f \in W \colon f(10) = 1 \}$. Is $\cal U_2$ a subspace of $W$?
\end{questionbody}

We want functions of the form $\lambda \cos(x) + \mu \cos(2x)$ for some $\lambda, \mu \in \R$ where $\lambda \cos(10) + \mu \cos(20) = 1$.

To be a subspace of $W$, $\cal U_2$ must be a non-empty subset (this is trivially true), and must be closed under the operations of $W$. So if we have some $\alpha \cos(x) + \beta \cos(2x) \in \cal U_2$, then we want $\lambda \l( \alpha \cos(x) + \beta \cos(2x) \r) \in \cal U_2$ for any $\lambda \in \R$.

But $\alpha \cos(10) + \beta \cos(20) = 1$ by definition of $\cal U_2$, and $\lambda \l( \alpha \cos(x) + \beta \cos(2x) \r) = \lambda \ne 1$. Therefore $\cal U_2$ is not closed under scalar multiplication and therefore is not a subspace of $W$.

% }}}

% {{{ Q10
\newquestion{10}

\begin{questionbody}
Consider $L_A \colon \R^4 \to \R^3$ (i.e. $\ul v \mapsto A \ul v$) for the matrix \[
A = \begin{pmatrix} 1 & 2 & 3 & 4 \\ 0 & 0 & a & b \\ 0 & 0 & c & d \end{pmatrix}
\] where the values for $a, b, c, d \in \R$ are not known.
\end{questionbody}

\subsection{~} % 10.a

\begin{questionbody}
State the Rank--Nullity Theorem. % chktex 8
\end{questionbody}

Let $\varphi \colon V \to W$ be a linear map. Then $\dim \Im \varphi + \dim \ker \varphi = \dim V$.

\subsection{~} % 10.b

\begin{questionbody}
Provide values for $a, b, c, d$ so that $\dim \op{Colspan} A = 1$. What are $\dim \Im L_A$ and $\dim \ker L_A$ in your example?
\end{questionbody}

$a = b = c = d = 0$ gives $\dim \op{Colspan} A = 1$. Then \[
\dim \Im L_A = \dim \op{Colspan} A = 1
\] and then by the Rank--Nullity Theorem, $\dim \ker L_A = 4 - 1 = 3$. % chktex 8

\subsection{~} % 10.c

\begin{questionbody}
Provide values for $a, b, c, d$ so that $\dim \op{Colspan} A = 2$. What are $\dim \Im L_A$ and $\dim \ker L_A$ in your example?
\end{questionbody}

$a = 1$, $b = c = d = 0$ gives $\dim \op{Colspan} A = 2$. Then $\dim \Im L_A = 2$ and $\dim \ker L_A = 2$.

\subsection{~} % 10.d

\begin{questionbody}
Provide values for $a, b, c, d$ so that $\dim \op{Colspan} A = 3$. What are $\dim \Im L_A$ and $\dim \ker L_A$ in your example?
\end{questionbody}

$a = 1$, $d = 1$, $b = c = 0$ gives $\dim \op{Colspan} A = 3$. Then $\dim \Im L_A = 3$ and $\dim \ker L_A = 1$.

% }}}

\end{document}
