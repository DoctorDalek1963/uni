% vim: set foldmethod=marker foldlevel=0:

\documentclass[a4paper]{article}
\usepackage[UKenglish]{babel}

\usepackage{preamble}

\usepackage{graphicx}
\graphicspath{ {./imgs/} }

\title{CS147 Discrete Maths and its Applications 2, Assignment 1}
\author{Dyson Dyson (ID: 5503449)}
\date{}

\begin{document}

\maketitle

\setlength{\parindent}{0em}
\setlength{\parskip}{1em}

% {{{ Q1
\question{1}

\subsection{~}

$8^n = \cal O(7^n)$ implies that $\exists\, c > 0, N > 0$ such that for all $n > N$, \begin{align*}
8^n &\le c 7^n\\
n \log 8 &\le \log c + n \log 7\\
n (\log 8 - \log 7) &\le \log c\\
n \log \f87 &\le \log c
\end{align*}
$\log \f87 > 1$, so whatever value of $c$ we choose, $n \log \f87$ will eventually be larger than $\log c$. Therefore $8^n$ is not $\cal O(7^n)$.

\subsection{~}

$n 2^{\f n2} = \Omega(n 2^n)$ implies that $\exists\, c > 0, N > 0$ such that for all $n > N$, \begin{align*}
n 2^{\f n2} &\ge c n 2^n\\
2^{\f n2} &\ge c 2^n\\
\log 2^{\f n2} &\ge \log (c 2^n)\\
\f n2 \log 2 &\ge \log c + n \log 2\\
0 &\ge \log c + \f n2 \log 2
\end{align*}
Everything on the right hand side is $>0$ and therefore not $\le 0$. Therefore $n 2^{\f n2}$ is not $\Omega(n 2^n)$.

\subsection{~}

$\log (n!) = \cal O(n \log n)$ implies that $\exists\, c > 0, N > 0$ such that for all $n > N$, \begin{align*}
\log (n!) &\le c n \log n\\
\log (n!) &\le \log(n^{cn})\\
0 &\le \log(n^{cn}) - \log(n!)\\
0 &\le \log\l(\f{n^{cn}}{n!}\r)
\end{align*}
We know that $n^n > n!$ for large $n$, so we see that $\df{(n^n)^c}{n!} > 1$, therefore the logarithm is greater than $0$, so $\log(n!)$ is indeed $\cal O(n \log n)$.
% }}}

% {{{ Q2
\question{2}

\subsection{~}

Let $T(n) = 4T\l(\f n2\r) + \Theta(\log n)$. We will not use the master theorem with $a=4$, $b=2$, and $d=?$.

\subsection{~}

Let $T(n) = 8T\l(\f n2\r) + \Theta(5^n)$. We will not use the master theorem with $a=8$, $b=2$, and $d=?$.
% }}}

% {{{ Q3
\question{3}

% }}}

% {{{ Q4
\question{4}

% }}}

% {{{ Q5
\question{5}

% }}}

\end{document}
