% vim: set foldmethod=marker foldlevel=0:

\documentclass[a4paper]{article}
\usepackage[UKenglish]{babel}

\usepackage{preamble}

\usepackage{graphicx}
\graphicspath{ {./imgs/} }

\fancyhead[L]{CS147 Assignment 1}
\title{CS147 Discrete Maths and its Applications 2, Assignment 1}
\author{Dyson Dyson (ID: 5503449)}

\begin{document}

\maketitle

\setlength{\parindent}{0em}
\setlength{\parskip}{1em}

% {{{ Q1
\question{1}

\subsection{~}

$8^n = \cal O(7^n)$ implies that $\exists\, c > 0, N > 0$ such that for all $n > N$, \begin{align*}
8^n &\le c 7^n\\
n \log 8 &\le \log c + n \log 7\\
n (\log 8 - \log 7) &\le \log c\\
n \log \f87 &\le \log c
\end{align*}
$\log \f87 > 1$, so whatever value of $c$ we choose, $n \log \f87$ will eventually be larger than $\log c$. Therefore $8^n$ is not $\cal O(7^n)$.

\subsection{~}

$n 2^{\f n2} = \Omega(n 2^n)$ implies that $\exists\, c > 0, N > 0$ such that for all $n > N$, \begin{align*}
n 2^{\f n2} &\ge c n 2^n\\
2^{\f n2} &\ge c 2^n\\
\log 2^{\f n2} &\ge \log (c 2^n)\\
\f n2 \log 2 &\ge \log c + n \log 2\\
0 &\ge \log c + \f n2 \log 2
\end{align*}
Everything on the right hand side is $>0$ and therefore not $\le 0$. Therefore $n 2^{\f n2}$ is not $\Omega(n 2^n)$.

\subsection{~}

$\log (n!) = \cal O(n \log n)$ implies that $\exists\, c > 0, N > 0$ such that for all $n > N$, \begin{align*}
\log (n!) &\le c n \log n\\
\log (n!) &\le \log(n^{cn})\\
0 &\le \log(n^{cn}) - \log(n!)\\
0 &\le \log\l(\f{n^{cn}}{n!}\r)
\end{align*}
We know that $n^n > n!$ for large $n$, so we see that $\df{(n^n)^c}{n!} > 1$, therefore the logarithm is greater than $0$, so $\log(n!)$ is indeed $\cal O(n \log n)$.
% }}}

% {{{ Q2
\newquestion{2}

\subsection{~}

Let $T(n) = 4T\l(\f n2\r) + \Theta(\log n)$. We know that $\log n = \Omega(1)$ and $\log n = \cal O(n)$. So let $T_1(n) = 4T_1\l(\f n2\r) + \Theta(1)$ and $T_2(n) = 4T_2\l(\f n2\r) + \Theta(n)$ and note that $T_1(n) \le T(n) \le T_2(n)$ for large enough $n$.

We can apply the master theorem to $T_1$ with $a=4$, $b=2$, and $d=0$. Then $\df a{b^d} > 1$ so $T_1(n) = \Theta\!\l(n^{\log_2 4}\r) = \Theta(n^2)$.

We can also apply the master theorem to $T_2$ with $a=4$, $b=2$, and $d=1$. Then $\df a{b^d} > 1$ so $T_2(n) = \Theta\!\l(n^{\log_2 4}\r) = \Theta(n^2)$.

Therefore $\Theta(n^2) \le T(n) \le \Theta(n^2)$ so $T(n) = \Theta(n^2)$.

\subsection{~}

Let $T(n) = 8T\l(\f n2\r) + \Theta(5^n)$. We know that $5^n = \Omega(n^d)$ for all $d > 0$. We cannot place a polynomial upper bound on an exponential function. Let $T_1(n) = 8T_1\l(\f n2\r) + \Theta(n^d)$ for some large $d$. Then $T_1(n) \le T(n)$ for large enough $n$.

We can apply the master theorem to $T_1$ with $a=8$, $b=2$, and large $d$. Then $\df a{b^d} < 1$ so $T_1(n) = \Theta(n^d)$. Therefore $T(n) \ge \Theta(n^d)$ for large $d$. Equivalently, $T(n) = \Omega(n^d)$ for large $d$.
% }}}

% {{{ Q3
\newquestion{3}

\begin{align*}
\bb P(A | B \cap C) &= \f{\bb P(A \cap (B \cap C))}{\bb P(B \cap C)}\\[1ex]
&= \f{0.1}{0.4}\\[1ex]
&= \f14\\[1ex]
\end{align*}
% }}}

% {{{ Q4
\question{4}

Let $X$ and $Y$ be discrete random variables distributed uniformly over $\{1,\ \ldots,\ n\}$. Either $X = Y$, $X > Y$, or $X < Y$, and $\bb P(X > Y \lor X < Y) = 1 - \bb P(X = Y)$. These are symmetric so $\bb P(X < Y) = \bb P(X > Y) = \f12 - \f12 \bb P(X = Y)$.

So $\bb P(X \le Y) = \bb P(X < Y) + \bb P(X = Y) = \f12  + \f12 \bb P(X = Y)$ and $\bb P(X = Y) = \f1n$. Therefore $$\bb P(X \le Y) = \f12 + \f1{2n} = \f{n + 1}{2n}$$
% }}}

% {{{ Q5
% \newquestion{5}
% TODO: How do I even approach this?

% }}}

\end{document}
